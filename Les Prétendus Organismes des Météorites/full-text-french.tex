\documentclass[a4paper, 12pt, oneside, french]{article}
\usepackage[utf8]{inputenc}
\usepackage[T1]{fontenc}
\usepackage[french]{babel}
\usepackage{ebgaramond}
\usepackage{booktabs}
\setlength{\emergencystretch}{15pt}
\usepackage{fancyhdr}
\usepackage{graphicx}
\graphicspath{ {./} }
\usepackage{float}
\begin{document}
\begin{titlepage} % Suppresses headers and footers on the title page
	\centering % Centre everything on the title page
	\scshape % Use small caps for all text on the title page

	%------------------------------------------------
	%	Title
	%------------------------------------------------
	
	\rule{\textwidth}{1.6pt}\vspace*{-\baselineskip}\vspace*{2pt} % Thick horizontal rule
	\rule{\textwidth}{0.4pt} % Thin horizontal rule
	
	\vspace{1.5\baselineskip} % Whitespace above the title
	
	{\LARGE Les Prétendus Organismes}

	\vspace{1.2\baselineskip}

	{\LARGE des Météorites}

	\vspace{1\baselineskip} % Whitespace above the title

	\rule{\textwidth}{0.4pt}\vspace*{-\baselineskip}\vspace{3.2pt} % Thin horizontal rule
	\rule{\textwidth}{1.6pt} % Thick horizontal rule
	
	\vspace{1\baselineskip} % Whitespace after the title block
	
	%------------------------------------------------
	%	Subtitle
	%------------------------------------------------
	
	{Par Carl Vogt\\ Président de L'Institut National Genevois} % Subtitle or further description
	
	\vspace*{1\baselineskip} % Whitespace under the subtitle
	
	%------------------------------------------------
	%	Editor(s)
	%------------------------------------------------
    \vspace*{\fill}

	{\small\scshape }

    {Imprimerie Centrale Genevoise, Rue du Rhone, 52\\ Genève 1882} % Subtitle or further description
    
    Internet Archive Online Edition  % Publication year
	
	{Utilisation non commerciale --- Partage dans les mêmes conditions 4.0 International} % Publisher
\end{titlepage}
\setlength{\parskip}{1mm plus1mm minus1mm}
\clearpage
\frenchspacing
\listoffigures
\clearpage
\section*{Les Prétendus Organismes des Météorites}
\pagestyle{fancy}
\fancyhf{}
\rhead{Les Prétendus Organismes des Météorites}
\cfoot{\thepage}
\paragraph{}
Vers la fin de 1880 parut en Allemagne un ouvrage in-4$^{\circ}$, qui ne pouvait manquer d'éveiller l'attention. Il était intitulé : Les Météorites (Chondrites) et leurs organismes, figurées et décrites par Otto Hahn, Docteur. Trente-deux planches avec cent-quarante-deux figures photographiées. Tubingue, 1880. Laupp, éditeur.

Je résume, en traduisant littéralement les paroles de l'auteur, les principaux résultats qu'il énonce.

\og Les Chondrites, roches composées de Feldspath-Olivine (Enstatite) sont constituées par un monde animal ; elles ne sont ni stratifiées, ni conglomérées, mais forment un feutre d'animaux, un tissu, dont toutes les mailles étaient jadis des êtres vivants, des animaux des types les plus inférieurs, des commencements d'une création. \fg (P. 3.)

\og Qu'on regarde les planches de mon ouvrage et l'on aura immédiatement la certitude qu'il ne s'agit pas de formes minérales, mais de formes organiques ; que nous devant nous des figures d'animaux du type le plus inférieur, appartenant à une création, qui pour la plus grande partie, trouvent leurs parents les plus proches sur notre terre ; --- quant aux crinoïde et aux coraux, c'est établi avec la certitude la plus absolue ; les spongiaires montrent au moins une telle ressemblance avec les formes de notre terre, comme elle existe entre des genres terrestres rapprochés. \fg (P. 7.)

\og Le lecteur qui regardera seulement d'une manière superficielle mes formes trouvera bientôt qu'elles fournissent une véritable histoire de développement. Toutes les transitions de l'éponge au corail, du corail au crinoïde sont là, de manière que l'on peut réellement être dans le doute, où l'on veut placer ces transitions, à moins d'en faire des genres nouveaux. \fg (P. 3.)

\og Sauf le seul travail de Gümbel dans le Bulletin de l'Académie de Munich, toutes les autres recherches faites jusqu'ici dans ce domaine ne peuvent être considérées comme des constatations scientifiques, autant par rapport à l'exactitude des observations, que par rapport aux conclusions basées sur ces observations, aux hypothèses non prouvées et aux a priori entièrement vides. \fg (P. 7.)

M. Hahn croit donc avoir fourni \og la preuve incontestable que les chondrites sont des restes d'animaux ayant vécu dans l'eau, que la météorite entière n'est formée que de restes de spongiaires, de coraux et de crinoïdes, métamorphosés, par pétrification, en enstatite. Il est vrai, dit-il, qu'il y a des petites places rares, où se trouvent des véritables cristaux, mais ces cristaux sont disposés de manière que cela ne peut avoir de l'influence sur la valeur de mes preuves réelles. \fg (P. 21.)

\og Si j'ai dit, continue-t-il, que les chondrites ne sont qu'un tissu d'animaux, un feutre animal, cela souffre cependant une exception. \fg

\og On trouve, en effet, dans cette roche à squelettes d'animaux des petites places à contours arrêtés, qui ont été probablement (mais pas nécessairement) roche dès le commencement. Ce sont des rares inclusions gris-bleues de 3 à 5 millimètres de diamètre sans forme déterminée et répétée, qui contiennent dans leur masse grise des cristaux évidents d'un minéral jaune-verdâtre, dont les coupes présentent tantôt des carrés ou des rhombes, tantôt des pentagones. Ce minéral peut être de l'augite ou de l'olivine. Mais cela ne renverse pas le fait, que des conformations organiques existent dans les couches à olivine et que ces conformations ont engendré la construction des corps planétaires, qu'elles ont composés et construits. \fg

\og Dans tous les cas, il y a inversion des rapports dans la roche chondritique vis-à-vis des couches sédimentaires de notre terre. Dans ces dernières, les organismes sont enfouis dans la masse rocheuse qui les entoure ; dans les premières, il n'y a que des organismes et la roche en est une collection. \fg (P. 35.)

\og Ces formes ne sont pas des formes minérales, \fg dit M. Hahn avec une certitude absolue. Mais sachant bien que des assertions semblables ne sont guère acceptées par le monde scientifique, sans preuves palpables, il cherche à les donner, en les groupant sous deux catégories, dites preuves positives et preuves négatives.

\og Pour fournir la preuve positive, qu'il s'agit d'un organisme végétal ou animal, je crois nécessaire de démontrer :
\begin{enumerate}
\item \og Une forme déterminée. \fg (Je ne sais autrement traduire le terme employé plusieurs fois par M. Hahn \og geschlossene Form \fg ; la traduction littérale \og forme fermée \fg n'ayant aucun sens).
\item \og Une forme qui se répète ;
\item \og Qui se répète dans les degrés du développement ;
\item \og Structure, savoir : cellules ou vaisseaux ;
\item \og Ressemblance avec des formes connues.
\end{enumerate}
\paragraph{}
\og Lorsque ces conditions sont remplies, il ne s'agit plus que de déterminer, si l'objet est une plante ou un animal ? Mes formes remplissent-elles ces conditions. \fg (P. 20.)

Il va sans dire que la réponse est affirmative.

De toutes ces conditions posées par M. Hahn, il n'y en a évidemment que deux qui puissent trancher la question sous certains points de vue ; les autres sont également applicables aux minéraux. Les cristaux ont des formes déterminées, qui se répètent toujours et encore bien mieux que les formes organiques, dans les diverses phases développement. Jusqu'à présent nous étions plutôt persuadé que c'était un privilège du grand nombre des types organiques, de changer de forme pendant les différentes phases de leur développement ; qu'abstraction faite des œufs, des germes et des semences, les formes larvaires, par exemple, étaient souvent fort différentes de celles des animaux définitifs, que les cotylédons des plantes ne ressemblaient souvent, en aucune manière, aux feuilles définitives, tandis que les formes de cristaux étaient éminemment stables. M. Hahn soutient que nous étions dans l'erreur. Soit, --- mais, en tout cas, les trois premières conditions qu'il pose ne disent absolument rien pour la distinction entre formes organiques et inorganiques.

La structure que M. Hahn invoque comme quatrième condition est sans doute prépondérante, à condition toutefois qu'elle persiste dans les parties animales ou végétales soumises à la pétrification. M. Hahn pose comme condition de cette structure la présence de cellules ou de vaisseaux. C'est fort bien, --- mais je voudrais savoir, quelles cellules et quels vaisseaux pourraient rester, lorsqu'une éponge subit la fossilisation ? On sait que les tissus de ces animaux sont composés de cellules éminemment délicates, qui diffluent avec la plus grande facilité, que tout ce qu'on peut retrouver dans une éponge pétrifiée consiste en spicules minéraux, calcaires ou siliceux, dans lesquels on ne peut voir ni cellules ni vaisseaux ! Et si la présence de cellules ou de vaisseaux est un caractère indispensable, que deviennent les coraux fossiles, où l'on ne voit absolument que des cristaux entourant des lacunes ?

Il ne reste donc, des cinq conditions posées par M. Hahn, que la dernière, la similitude avec des formes connues. Mais ici encore les incertitudes les plus grandes peuvent prendre place. Sont-ce les formes extérieures ? Sont-ce les détails de structure des formes ? Nous citerons, dans un autre mémoire, une foule de cas, où des conformations éminemment minérales, produites artificiellement ou par la nature, miment d'une manière parfaite les formes organiques et nous avons, d'un autre côté, dans les coraux, dans les cristaux intracellulaires des plantes, dans les otolithes des animaux, une quantité d'exemples de formes minérales, produites par les organismes.

Il faut donc s'adresser aux formes et structures spéciales comparées. Il faut pousser la comparaison jusque dans les détails les plus minimes en apparence, lorsqu'on veut nous prouver que tel objet, que nous avons sous les yeux, est une éponge, un corail ou un crinoïde. Nous laissons donc pour le moment de côté les preuves dites négatives, par lesquelles l'auteur veut nous démontrer que les objets figurés par lui ne peuvent être des formes minérales, --- elles sont à peu près de la même valeur que ses preuves positives. Nous nous adressons aux formes spéciales, qui par la ressemblance avec les formes connues et par leur structure identique doivent fournir la preuve incontestable que les chondrites sont formées par des organismes parents de ceux de la terre.

Nous passons successivement en revue ces prétendus organismes, en énumérant, avec les termes mêmes de l'ouvrage, les caractères que l'auteur attribue aux différents organismes qu'il croit avoir reconnus.

\og A. --- Spongiaires \fg

\og 1$^{\circ}$ \emph{Urania}. \fg

\og Corps en lobe arrondi avec point de fixation visible. \fg --- \og Plis causés par la contraction. \fg --- \og Contourné en spirale. \fg --- \og La structure consiste en une membrane externe, posée sur des couches lamellaires. \fg --- \og Couleur bleue. \fg --- \og Stratification évidente. On serait tenté de mettre l'exemplaire figuré parmi les coraux, s'il n'y avait la forme. \fg --- \og On croit voir l'indication d'un orifice buccal. \fg

\og D'après ce qui vient d'être dit, je considère l'\emph{Urania} comme une éponge fixée, qui se contracte en spirale, aspire et expulse l'eau comme nos éponges actuelles. \fg (P. 23 et 24.)

Tels sont les détails de structure, qui doivent nous convertir à l'opinion de M. Hahn. Les \emph{Urania} occupent, suivant lui, trois vingtièmes de la masse totale des météorites pierreuses ; elles sont figurées sur six planches comprenant trente et une figures.

Dans un ouvrage précédent du même auteur \emph{Die Urzelle}, l'\emph{Urania guilielmi}, dédiée à l'empereur Guillaume, était représentée comme un végétal à feuilles arrondies, enroulées dans leur jeune âge, muni de capsules portant des spores. En passant dans l'ouvrage actuel, l'\emph{Urania} a perdu ces capsules avec leurs spores ; elle est devenue une éponge. Il est vrai que nous n'apprenons point pourquoi ce changement de place si considérable a été opéré ; l'auteur ne dit pas un traître mot sur les raisons qui l'ont engagé à changer d'opinion. Quels sont les caractères que ce prétendu organisme a perdus ou gagnés pour être transporté d'un règne à l'autre ? Question inopportune à laquelle l'auteur ne répond pas.
\og 2$^{\circ}$ Eponges à spicules. \fg (Table 7.)

\og Je place la fig. 1 parmi les \emph{Astrospongia}. Les spicules sont régulièrement croisés. Fig. b est une charpente spiculaire irrégulière avec une cavité faiblement indiquée. \fg (P. 24.)

Les prétendus spicules ressemblent, à s'y méprendre, à des cristaux linéaires dispersés dans une masse homogène, tels qu'on les voit dans la première lave venue. On voit en quelques endroits une tendance peu marquée vers un arrangement stellaire, très commun chez les cristaux, insolite chez les spicules des éponges, dont les formes connues sont tout à fait différentes (1).

L'auteur ne peut avoir comparé ses /emph{Urania} et ses Astrosponges avec les spongiaires vivants et fossiles ; il ne peut avoir étudié la structure de ces derniers, car il serait impossible qu'avec ces connaissances acquises il aurait pu vouloir faire croire aux connaisseurs, que les notions et figures données par lui ont le moindre rapport avec la structure et les caractères microscopiques des spongiaires. M. Hahn doit ignorer entièrement les belles recherches de M. Zittel sur les spongiaires fossiles. (Mém. de l'Acad. de Munich. Vol. 12 et 13 ; \emph{Handbuch der Palaeontologie, Vol. 1}), car avec cette connaissance il n'aurait pu nous présenter, comme des spongiaires évidents, des coupes à contours arrondis, entourés d'une membrane (sic!), ayant une structure ou finement striée ou lamellaire, également inconnues chez les spongiaires vivants et fossiles. Nous connaissons, il est vrai, une quantité de spongiaires fossiles, où la disposition des canaux présente un arrangement rayonné, visible déjà à l'œil nu ou à la loupe (\emph{Aulocopium}, les Ventriculitides) ; mais dans toutes ces éponges, les spicules, soit libres, soit formant par leur réunion un squelette réticulé à mailles très régulières, sont toujours reconnaissables par les grossissements avec lesquels M. Hahn a travaillé. Chez les prétendus spongiaires des météorites, il n'existe aucune trace de ce squelette caractéristique. Nous connaissons aussi par les recherches de M. Zittel, les conditions sous lesquelles, par la pseudomorphose des éponges siliceuses en calcaire et celle des éponges calcaires en silice, la structure intime peut se perdre entièrement ou en partie ; mais dans ces cas, l'indication des canaux disparait également et il ne reste que des masses amorphes sans structure apparente, qu'on appelait autrefois \og pétrosponges \fg mais qui ont été rayés entièrement de la classification depuis que M. Zittel a fait connaître leur véritable structure primitive.

Conclusions : les prétendus spongiaires des météorites n'ont ni la forme, ni la structure des spongiaires connus.

\og B. --- Les Coraux \fg

\og Ici, nous avons des formes terrestres si bien conservées, qu'il ne peut rester aucun doute. \fg

\og Tab. 8 montre une forme modèle ; Tab. 9 fait voir la structure canaliculaire ; canaux bourgeonnants manifestes qui relient entre eux les tubes (car ce sont des tubes). Il s'y ajoute la courbure des canaux qu'on ne peut absolument pas confondre avec des cassures lamellaires, les orifices manifestes des tubes et le point de fixation évident... Les canaux bourgeonnants sont distants de 0,003 mm. Certainement tout ce que l'on peut demander pour la structure d'un \emph{Favosites}. \fg

\og Tab. 11. Ici, chaque observateur reconnaîtra facilement l'image de formes vivantes de coraux, d'autant plus que figure 1 indique la forme en calice (cavité). Le même objet montre des cloisons manifestes dans les tubes. \fg (Malheureusement, je ne réussis pas de voir dans cette figure ni indice d'une cavité, ni tubes, ni cloisons transverses.)

Dans d'autres figures : \og Structure lamellaire évidente. \fg

Dans d'autres : \og Coraux tubulaires évidents. On distingue clairement dans l'original : Matière interstitielle vitreuse, paroi tubulaire noire, matière jaune remplissant les tubes, qui aussi quelquefois devient noire. Cette forme revient à cent reprises et dans toutes les chondrites. \fg (P. 25 et 26.)

Les coraux constituent, suivant l'auteur, un vingtième de la masse totale.

En étudiant attentivement les trente figures des prétendus coraux distribuées sur neuf planches, on peut se convaincre d'abord, que toutes les figures représentant des échantillons entiers, montrent absolument la même forme générale que les Uranias, --- forme arrondie à contours nettement accusés, semblable à celle d'une feuille ronde ou ovalaire entière. La seule différence qui existe entre les prétendues éponges et les prétendus coraux est dans l'apparence des stries divergentes qui partent d'un point de départ excentriquement rapproché de bord et qui paraissent plus épaisses et mieux accusées dans les coraux. C'est comme on voit, la forme générales des chondres, --- la plupart des figures ne nous donnent absolument rien de plus que ce que nous ont fait connaitre depuis longtemps les auteurs qui se sont occupés des météorites. Nous trouvons, il est vrai, quelques rares figures montrant des stries rayonnantes depuis plusieurs points de départ. M. Gümbel déjà mentionné cette disposition exceptionnelle que j'ai constatée aussi sur plusieurs de mes coupes ; nous en voyons une autre, désignée sous le nom de \og corail en chaîne \fg (Kettenkoralle), où sur un espace arrondi clair se présentent quelques taches obscures à contours lavés et disposés irrégulièrement. Cette figure ressemble autant et plus peut-être à une peau tachetée d'un chat quelconque qu'à un corail. Mais l'auteur veut que cela soit un corail ; que ta volonté soit faite, seigneur!

La structure doit surtout ressortir de deux figures photographiées sous un fort grossissement, Tab. 9 et Tab. 15. Sur la première on voit des colonnettes à contours arrêtés droites, quelquefois un peu courbées; quelques-unes de ces colonnettes montrent des séries de points sombres alignés dans l'axe. Le même pointillage se voit sur quelques colonnettes de la quinzième planche, seulement cette même figure donne en même temps l'explication du phénomène, lequel, suivant M. Hahn, doit fournir la preuve de l'existence d'un canal axial au centre des colonnettes. On y voit en effet une colonnette ébréchée par intervalles presque réguliers sur l'un des bords et cassée transversalement en plusieurs morceaux, qui ressemble ainsi à un bâton d'engrenage. Les ébréchures comme les cassures sont remplies d'une matière incrustante noire. Qu'on se figure un bâton d'engrenage à enfoncements coniques usé, sur la surface portant les creux jusque vers le fond de ces creux et on aura l'image d'une colonnette marquée de points alignés dans l'axe, telle que la figure M. Hahn.

S'il est déjà étonnant que parmi ces nombreuses figures, comparées tantôt à des \emph{Favosites} du Silurien, tantôt à des coraux cratériformes, étoilés ou même à chaînons, il ne s'en trouve pas une seule qui montre une forme générales tant soit peu différente des prétendues Uranias, notre étonnement s'accroît encore si nous comparons les structures (non décrites, car M. Hahn ne donne pas des descriptions, mais représentées) avec celles que nous connaissons aux coraux vivants ou fossiles bien caractérisés. Bien téméraire, en effet, celui qui voudrait trouver, dans les figures de M. Hahn, quelque chose d'analogue à la figure que nous donnons de la partie d'une coupe d'une branche de Syringopora caliendrum Ehrenberg, qui nous a été prêtée obligeamment par notre collègue, M. Th. Studer, professeur à Berne, et que nous donnons de préférence à des coupes de Fungies, d'Astrées, de Madrépores, de Méandrines, de Tubipores et de \emph{Favosites} en notre possession, parce qu'elle résume pour ainsi dire les modifications de structure que l'on peut trouver sur les autres coraux. Cette coupe (fig. 1) montre en effet une branche du corail coupée en long. La coupe traverse des larges espaces entourés par un squelette plus épais et des fines pointes usées jusqu'à la transparence la plus complète.

\og La structure microscopique des squelettes des Madréporaires, dit M. Zittel (Palaeontologie, p. 206), est très uniformément fibro-cristalline. Les fibres minimes qui rayonnent en partant de centres de cristallisation, forment des dessins étoilés ou semblables à des plumes. \fg

Le squelette du polypier d'un Anthozoaire montre en effet toujours une structure microscopique qui, dans la plupart des cas, est franchement cristalline. Un tube, une branche de corail n'est pas simplement un morceau de calcaire solide, percé dans son axe par un canal central arrondi ou divisé par des cloisons, comme nous le présente M. Hahn ; cette branche est toujours composée d'une multitude de petites pièces cristallines, assemblées dans un ordre déterminé. Sur les coupes transversales des canaux ou des cellules des \emph{Favosites} et des Tubipores, on voit les têtes de ces pièces avancer vers l'intérieur ; sur les coupes longitudinales, elles paraissent disposées comme les barbillons d'une plume. Le bourgeon d'un canal (notre figure en montre un), n'eût-il qu'un dixième de millimètre d'épaisseur, fera toujours voir cette structure composée par la simple raison que le squelette est primitivement composé de spicules cristallins isolés les uns des autres, qui se réunissent seulement plus tard. Ces spicules épars se voient aisément dans la couche corticale des Gorgonides, des Isidés, dans la masse charnue des Alcyonaires. Dans les parties étalées du polypier, dans les lamelles de remplissage, dans les cloisons souvent si fines, ces pièces cristallines se rassemblent en étoiles, simulant quelquefois par leurs formes des corpuscules osseux ou montrent seulement un aspect réticulé, mais dans lequel les pièces minimes sont encore reconnaissables sous un fort grossissement. Nous donnons une figure (fig. 1a) de cette structure réticulée sous un grossissement de 500 diamètres. Cette structure ne disparait jamais, à moins qu'une cristallisation pétrifiante ait envahi le tout, squelette comme intervalles ; on la voit encore sur les coupes les plus minces, elle y apparaît même beaucoup mieux que sur des coupes un peu épaisses ; elle se montre, avec la dernière, dans les cloisons si minces des \emph{Favosites}.

Or, cette structure si caractéristique à éléments cristallins de forme variée, mais constante dans chaque espèce, fait entièrement défaut dans les prétendus coraux de M. Hahn, tirés des chondres. Nous avons sous nos yeux des coupes de chondres, qui représenteraient pour cet auteur des coraux ; l'objet est composé de baguettes ou de colonnettes solides, rayonnantes depuis un centre excentrique (point de fixation pour M. Hahn), quelquefois dichotomisées à angles très aigus, séparées les unes des autres par une masse incrustante opaque, laquelle s'est infiltrée dans les cassures transversales ou dans les ébréchures superficielles, simulant ainsi des cloisons transversales ou des séries longitudinales de trous et de rainures.

Il n'y a donc aucune similitude entre les prétendus coraux de M. Hahn et les véritables coraux, tels que nous les connaissons dans les différentes créations depuis les couches les plus anciennes de la terre. Il n'y a pas même de similitude quant aux formes extérieures, car les cellules tubiformes des Favositides sont distinctement polygonales et percées de trous sur leurs parois, et le polypier en entier est ou grossièrement branchu ou bien disposé en masse épaisse.

Nous arrivons à la dernière classe, représentée, suivant M. Hahn, dans les chondres des météorites et qui en forme, suivant cet auteur, à elle seule les seize vingtièmes de la masse totale. C'est la classe ou même, si l'on veut, l'embranchement des Échinodermes, représenté par les crinoïdes. Étudié de préférence par notre auteur, ce type n'a pas fourni moins de soixante-six figures. Ici, nous trouverons sans doute une plus ample moisson de faits et d'observations. La structure des crinoïdes est compliquée ; leurs formes sont très variées ; l'étude offre beaucoup de difficultés, sur lesquelles peut s'exercer la sagacité de l'observateur. Vu cette multitude d'exemplaires trouvés dans la seule météorite de Knyahinya, le fond de la mer planétaire, dont proviennent les aérolithes, doit avoir ressemblé aux forêts sous-marines à crinoïdes, que nous ont fait connaître les dragages des expéditions modernes.

\og C. --- Les Crinoïdes. \fg

\og Se trouvent depuis la forme la plus simple d'un bras articulé jusqu'aux crinoïdes complets avec tige (nous avons vainement cherché une tige sur les figures), avec calice, bras principaux et auxiliaires. La conservation est ordinairement bonne. La difficulté gît dans les mille directions des coupes qui donnent toujours des images différentes du même objet. Les corps oviformes, que l'on considérait comme des verres, sont des calices de crinoïdes. \fg --- \og Bras brisés par pression d'en haut. \fg --- \og Crinoïdes avec autant de bras qu'il vous plaira \fg (Mit einer beliebigen Anzahl von Armen). --- \og Crinoïde à cinq bras. \fg --- \og Structure réticulée sur quelques formes, qui s'accorde avec la structure de la Schreibersite dans les fers météoriques. \fg --- \og Différentes formes douteuses ; on ne sait si ce sont des éponges, des \emph{Urania} ou des coraux. \fg --- \og Rappelle le genre \emph{Comatula}. \fg

Je crois n'avoir rien omis de ce qui se rapporte aux observations sur les formes et les structures. Le reste doit se deviner sur les figures.

On conviendra que c'est bien maigre. Quelques assertions sans preuve aucune.

Comme je l'ai déjà fait pressentir à propos des éponges et des coraux, l'auteur ne présente aucune comparaison, même superficielle, avec la structure d'autres organismes vivants ou fossiles appartenant à la même classe. M. Hahn se contente de la ressemblance la plus grossière. Les objets qu'il figure ressemblent en effet à des crinoïdes comme une feuille de Sabal ou de Chamaerops ressemble à un éventail. Cela suffit.

Il y aurait long à dire, si l'on voulait entrer dans une critique détaillée des nombreuses figures photographiées par l'auteur. C'est ainsi que toutes les figures de la Table 29 seront prises, par tous les observateurs qui se sont occupés de recherches sur des coupes minces de roches, pour des assemblages de cristaux plus ou moins aciculaires, réunis sous la forme si commune d'astérisques groupés autour de différents centres, tels qu'on les voit, par exemple, dans les actinolithes. La plupart des figures de la planche suivante ne contrediront pas cette diagnose. D'autres figures, comme celles des Tab. 17 et 28, ne montrent aucune ressemblance, ni lointaine, ni grossière, avec une partie ou coupe d'un crinoïde ; sur d'autres figures, enfin (Tab. 19) des coupes de gros cristaux mal définis, à angles usés et traversés par des cassures dirigées dans tous les sens, nous sont hardiment octroyées comme des plaques du calice d'un crinoïde, dont les bras se résolvent immédiatement, sans transition, en une foule de rayons secondaires.

On peut appliquer à tous ces prétendus crinoïdes la même remarque que nous avons déjà faite à propos des coraux. Tous, en tant qu'ils sont entiers, ont exactement la même forme en feuilles arrondies, comme les coraux, comme les Uranias. On pourrait calquer les contours de l'éponge \emph{Urania} pour les appliquer sur un corail, sur un crinoïde, sans avoir besoin d'y faire la moindre retouche. Nous donnons une figure d'un crinoïde Hahnien (fig. 2), dessinée à la chambre claire sur une coupe mince de la météorite de Vouillé, dont M. Daubrée nous a permis de faire usage avec son obligeance habituelle. Cette figure est même beaucoup plus complète qu'aucune des figures photographiées en si grand nombre par M. Hahn, --- on y verra exactement la même forme en feuille arrondie. Nous ne sommes cependant pas tout à fait sûr, il est vrai, si notre détermination est juste, --- est-ce une \emph{Urania}, un corail, un crinoïde ? Nous en laissons volontiers le choix au lecteur, --- ce dont nous sommes sûr, c'est qu'en tout cas, c'est une coupe d'un chondre complet, dans lequel sont enchâssés par places des éclats de fer météorique.

Certes, aucune des figures données par M. Hahn ne correspond aux formes extérieures des crinoïdes, telles que nous les connaissons. L'ordonnance générale du corps correspond-elle mieux ? Il est permis d'en douter. Sauf un seul cas, aucun de ces crinoïdes météoritiques n'obéit à la loi générale, qui établit le nombre de cinq rayons pour les animaux de cette classe. Quelques rares Cystoïdes seulement offrent des exceptions à cette règle en ce qu'ils ont un nombre réduit de bras toujours peu développés, simples, sans ramifications, et si peu apparents qu'on a longtemps nié leur existence. Chez les crinoïdes de Knyahinya, au contraire, quel développement luxueux de bras, ramifiés à l'excès, en nombre aussi considérable que l'on voudra! Les quelques vrais crinoïdes fossiles à six bras (\emph{Hexacrinus}, \emph{Atocrinus}) sont si rares, si voisins de genres semblables, que la plupart des auteurs les considèrent comme des monstruosités. Mais ils ne peuvent se comparer en aucune façon à ces Briarées tombés sur la terre et qui probablement ont été précipités, parce qu'ils se trouvaient en rébellion ouverte contre la loi établie pour les créations terrestres.

La forme générales nous laisse en défaut, l'ordonnance des parties du corps nous échappe, --- il faut donc s'attacher à la structure intime, microscopique de ces êtres, dépourvus de tiges, de calices et munis d'un nombre infini de bras ramifiés à outrance, lesquels, à tout prendre, n'étaient que bras et auraient été très embarrassés, suivant toute apparence, pour loger les organes nécessaires pour la vie, si toutefois ils avaient été animés.

La structure microscopique des pièces calcaires du squelette des Echinodermes est facile à reconnaître. C'est un fait constant que toutes ces pièces, quelles qu'elles soient, plaques, piquants, articles des tiges, des bras, des cirrhes ou des pinnules, ont toujours une structure réticulée, à mailles trouées plus ou moins serrées, structure qui se manifeste dès l'apparition du squelette chez les jeunes et se maintient chez tous dans l'âge adulte. Toutes ces pièces du squelette sont construites sur le même type fondamental, parce qu'elles se forment par la réunion d'éléments constituants anguleux, primitivement isolés les uns des autres, mais qui se soudent par leurs proéminences. Les mailles peuvent être plus lâches ou plus serrées, mais elles ne manquent jamais, même dans les parties les plus dures du squelette.

Je donne comme exemple de cette structure la figure dessinée d'après nature et sous un faible grossissement, du \emph{Pentacrinus europaeus} (fig. 3), de la larve si connue de la comatule. On voit cette structure réticulée à mailles sur la tige, composée de cylindres articulés, sur les plaques principales et axillaires du calice, sur les bras encore peu développés. Je n'ai, du reste, qu'à rappeler les descriptions et les figures données par M. Carpenter (Embryogénie de l'Antedon) (\emph{Comatula}) et celles déjà anciennes de M. Valentin (Monographies des Echinodermes vivants et fossiles par Agassiz. Neuchâtel 1838-45. Echinus). M. Zittel résume cette structure très bien dans sa \emph{Paléontologie} (Vol. 1, p. 311-315). Cet auteur dit, en parlant des crinoïdes fossiles (p. 324) : \og Ils montrent presque toujours une conformation essentiellement cristalline, due à l'infiltration de spath calcaire, mais que détruit rarement la structure microscopique réticulée d'une manière complète. Celle-ci perd, en revanche, lorsque le calcaire est remplacé par de la silice. \fg

Or, rien, absolument rien de cette structure ne se voit dans les figures de M. Hahn. Ce qui lui plaît de désigner sous le titre de \og structure réticulée \fg (Tab. 30, fig. 6 ; Tab. 21, fig. 5) ne ressemble en aucune manière à la structure à mailles des pièces des Echinodermes, mais plutôt à des cristaux très-petits, coupés obliquement et disposés en gradins. M. Hahn y trouve lui-même une ressemblance \og étonnante \fg avec la Schreibersite des fers météoriques, laquelle deviendra peut-être aussi, l'imagination aidant, un organisme. En revanche, ni les bras d'aucun de ces prétendus crinoïdes, ni surtout les plaques colossales composant le soi-disant calice d'un de ces crinoïdes, figuré Tab. 19 et qui ne sont autre chose qu'un cristal traversé par des cassures remplies d'une substance opaque, ne montrent aucune trace de la structure caractéristique des pièces squelettaires des crinoïdes.

J'avoue franchement que ce manque absolu de recherches comparatives sur des animaux déterminés, vivants ou fossiles, cette absence complète des caractères de structure microscopique si connus, tels qu'ils se trouvent dans les pièces squelettaires d'une classe relativement si hautement organisée comme les Echinodermes, m'ont inspiré les premiers soupçons quant au bien-fondé des conclusions que M. Hahn tirait de ses observations si laborieuses.

Aussi paraît-il qu'un des défenseurs de M. Hahn, son ami M. Weinland, zoologiste, a complétement abandonné les \og soi-disant crinoïdes\fg de son ami \og qu'il ne peut suivre partout dans ses déterminations zoologiques. \fg (\emph{Das Ausland}, N$^{\circ}$ 26, 1881.)

Je parlais de mes doutes. Ils se sont accrus lorsque je voyais, qu'on me permette de le dire, la désinvolture avec laquelle M. Hahn transportait ses organismes, non-seulement d'une classe, mais même d'un règne organique à l'autre. Des objets qui lui paraissaient des coraux au moment où il arrangeait ses planches, devenaient pendant la rédaction du texte des crinoïdes ou des spongiaires, comme s'il n'y avait pas un abîme entre ces différents types, comme si leur structure n'était pas, comme nous l'avons démontré, foncièrement différente. L'\emph{Urania}, plante rapprochée des Floridées, ayant des organes de reproduction dessinés et décrits dans un ouvrage précédent (\emph{Urzelle}), avait tout d'un coup perdu ces organes et était devenue, en un tour de main, une éponge. Si, dans sa réponse aux critiques de M. Rzehak (\emph{Das Ausland}, N$^{\circ}$ 20) M. Weinland excuse son ami en disant \og qu'au commencement de notre siècle \og beaucoup d'observateurs capables prenaient encore les éponges pour des plantes, \fg il nous semble que cette excuse est pire que l'erreur, car un auteur actuel ne doit pas retomber dans les fautes commises il y a 80 ans! Un autre auteur aurait senti la nécessité, vis-à-vis d'un public scientifique, d'exposer les raisons qui l'ont engagé à modifier son appréciation, si ces raisons consistent dans des détails de structure nouvellement découverts, dans des études comparatives faites sur des algues et des éponges, etc. Ici, rien de semblable, sic volo, sic jubeo, stat pro ratione voluntas!

Je me trompe. M. Hahn motive ces transpositions par un des chapitres les plus curieux, qui ont été écrits de notre temps, où l'on ne sait ce que l'on doit admirer le plus, la parfaite ignorance de l'auteur des lois de l'évolution ou la hardiesse avec laquelle il expose ses vues dans des termes dignes de l'oracle de Delphes. Notre auteur démontre en effet \og le type unitaire de tous les organismes météoriques. \fg Eponges, coraux, crinoïdes ne sont qu'un type unitaire! Les formes se développent l'une de l'autre. Je cite textuellement : \og Il est certain que l'\emph{Urania} est la forme la plus simple. Mais cette forme est le point de départ des autres. \fg

\og Le lambeau sémicirculaire se divise en couches, les couches en tubes, les tubes se divisent en travers. Maintenant se forment des bras, réunis par un canal. Un calice se développe entre les bras et la tige de fixation et le crinoïde le plus simple est là! \fg En vérité, c'est excessivement simple que ça!

Il y a cependant un fond de vérité dans cette singulière énonciation. Tous ces organismes de M. Hahn procèdent en effet d'un même type, mais qui est loin d'être organique. Je reviendrai sur ce sujet en démontrant que le mot \og structure organique \fg dont M. Hahn et ses amis font un usage vraiment abusif, est un mot entièrement vide de sens, lorsqu'on l'emploie en général et en l'appliquant à toutes les formes sans exception et qu'il ne peut être employé qu'en l'appliquant à un objet déterminé et connu. On peut dire : telle structure est identique à celle des éponges, des coraux, des crinoïdes, donc elle est organique; on ne peut pas dire : tel objet a une structure organique ou inorganique, car d'un côté des corps engendrés par des organismes, comme les polypiers des coraux, ne sont composés que de cristaux et d'un autre côté des corps absolument inorganiques peuvent engendrer des formes impossibles à distinguer de formations organiques.

Or, comme je viens de le prouver, les prétendus organismes de M. Hahn n'ont aucunement la structure des animaux auxquels il les rapproche ; nous pouvons donc dire que la preuve positive n'est pas fournie.

A défaut de preuves positives, M. Hahn a cherché à accumuler un certain nombre de preuves dites négatives, qui peuvent se résumer de la manière suivante : les formes que j'ai décrites et figurées ne peuvent provenir de corps inorganiques, donc elles sont organiques.

Nous n'allons pas suivre M. Hahn dans ces généralités lesquelles, comme nous venons de le dire, n'ont aucune valeur en elles-mêmes ; nous examinerons, en étudiant les faits donnés par l'observation, les détails pour arriver après à des conclusions générales.

M. Hahn a examiné 19 Météorites. C'est celle de Knyahinya (9 Juin 1866) qui lui a fourni le matériel le plus considérable. Sa collection de 360 coupes fines doit être, si nous en croyons M. Weinland, la plus magnifique collection du monde. Nous le croirons volontiers. Sauf quelques rares exceptions, lesquelles du reste n'offrent aucun type nouveau, toutes les figures de l'ouvrage en question représentent des prétendus organismes de Knyahinya. Un seul fragment de cette chute a fourni cette multitude de formes, que M. Hahn évalue à plusieurs centaines. C'est bien du bonheur, sans doute, que de trouver dans un seul caillou tant de formes réunies ensemble. Nous autres paléontologistes terrestres ne sommes pas aussi fortunés.

Or, la méthode de recherches, suivie par M. Hahn et ses amis, est toujours la même très connus depuis longtemps ; on fait des coupes minces et on observe au microscope.

\og Je faisais, dit M. Hahn, à dessein les coupes dans trois épaisseurs ; peu translucides, pour avoir les corps inclus aussi complets que possible ; très minces, pour voir clairement la structure ; la plus grande partie de telle manière qu'on avait les deux vues à la fois. \fg

\og J'ajoute ici une remarque, que confirmera chacun qui s'est occupé de coupes minces de pétrifications. \fg

\og Ce n'est que dans des cas rares que la structure reste visible sur des coupes parfaitement transparentes et par conséquent très minces. L'observateur au microscope est au suprême degré réjoui par les belles formes et lignes qu'il voit à sa coupe semi-transparente. Dans sa joie, il veut faire encore mieux et s'attend, en continuant à user sa coupe, à voir une image parfaite. Mais lorsqu'il remet pour une seconde fois sa coupe sous le microscope, il n'y voit plus qu'une étendue presque sans structure, des formes à peine indiquées, incertaines dans les contours, qui ne laissent plus reconnaître sous le microscope ce qu'on apercevait un moment auparavant sous la loupe. Mais ce phénomène est en connexion avec la métamorphose des roches et des formes qui y sont incluses. La chose est du reste connue et n'a pas besoin de détails plus spéciaux \fg (P. 16 et 17.)

J'avoue que mon expérience va à fin contraire. Sur des coupes semi-transparentes, je ne vois que des choses confuses et ce n'est que sur des coupes très minces et très transparentes que je vois les détails de structure. Je reviendrai du reste sur ce sujet.

Dans mes études, entreprises dans le but de me convaincre de l'existence d'organismes dans les aérolithes, je devais nécessairement m'adresser aux chondrites et spécialement aux chondres mêmes, qui en forment la plus grande portion. Pour M. Hahn, les chondrites ne sont, comme nous l'avons dit, qu'un \og feutre d'organismes \fg et les cristaux y forment une exception des plus rares. M. Weinland ne va pas si loin. \og Les différentes chondrites, dit-il, sont très inégalement riches en conformations organiques ; quelques-unes en sont composées aux deux tiers et plus. \fg Et le troisième tiers de la masse ? Je pense que les deux amis se mettront d'accord sur ce tiers, organique pour l'un, inorganique évidemment pour l'autre. C'est un détail d'appréciation, sans doute ; mais comme il s'applique aux mêmes objets préparés par M. Hahn et que M. Weinland avait à sa disposition, il a son importance. Que deviennent en face de ce tiers les preuves négatives de M. Hahn, suivant lesquelles les formes de ce tiers ne peuvent pas être inorganiques ?

Il fallait donc s'adresser aux chondres. En parcourant les ouvrages, j'ai vu avec étonnement, que malgré l'avis de M. Hahn, relaté plus haut, la structure de ces corps a été déjà parfaitement reconnue par Gustave Rose, qui leur a donné le nom (Sur la constitution des météorites, 1862), par M. Daubrée (Comptes-rendus 1866), par M. Tschermak (dans ses communications nombreuses à l'Académie de Vienne) et par tant d'autres ; que M. Gümbel avait fait un résumé complet de ces connaissances (Académie de Munich, Bulletin 1878), cité du reste avec éloge par M. Hahn et que MM. Makowski et Tschermak avaient complété ces renseignements en dernier lieu à propos de l'aérolithe de Tieschitz. (Mém. Acad. Vienne 1878). Tandis que les figures, du reste très exactes de M. Gümbel, sont en effet insuffisantes, étant dessinées sous des grossissements trop faibles, celles données par M. Makowski et Tschermak font parfaitement voir les formes extérieures et la structure rayonnante des chondres, ainsi que les détails des inclusions et des incrustations. Je donne ici la description faite par M. Gümbel, pour ne pas être forcé de répéter dans la suite des choses connues.

\og Toutes les chondrites sont sans doute des roches en débris, composées d'esquilles minérales petites ou plus grandes, des chondres arrondis connus, presque toujours parfaitement conservés, mais souvent aussi cassés en morceaux et enfin de grains métalliques, fers météoriques, chrômés ou sulfurés. Tous ces fragments tiennent ensemble, mais ne sont pas collés par une substance intermédiaire, --- on ne trouve point de substances amorphes, vitreuses ou laviques. \fg (M. Tschermak a cependant trouvé de ces substances vitreuses dans la météorite d'Orvinio (Mém. Acad. Vienne. Vol. 20 1870), et on peut soulever la question, si la substance incrustante des colonnettes, dont nous parlerons, ne s'est pas trouvée en état de fusion ou de demi-fusion, ce qui me paraît d'autant plus probable qu'elle a souvent un aspect boursouflé et qu'elle forme des inclusions entre les cristaux. Cette substance entre trop profondément partout dans les interstices les plus minces, pour qu'on puisse croire qu'elle provienne uniquement de l'écorce.) --- \og Ce n'est que dans l'écorce fondue et dans les incrustations noires semblables à l'écorce et qui pénètrent dans les cassures, que l'on rencontre une substance amorphe vitreuse, mais qui a été engendrée postérieurement pendant la chute de l'aérolithe à travers l'atmosphère. Les granules plus grands et difficilement fusibles sont encore conservés le plus souvent dans cette écorce sans être fondus. Les esquilles minérales ne montrent aucune trace d'usure ou de roulement ; elles sont à angles vifs et pointues. La surface des chondres n'est jamais lisse, comme elle devrait l'être, si ces globules étaient le résultat d'une usure par roulement ; elle est au contraire inégale, mamelonnée, âpre comme la surface d'une mûre ou taillée en facettes cristalloïdes. Beaucoup de ces chondres sont allongés, avec un amincissement dans une direction donnée, comme cela arrive pour les grêlons. On rencontre souvent des pièces qui doivent être considérées comme des morceaux de chondres éclatés ou déchirés. Exceptionnellement se voient des chondres joints ensemble comme des jumeaux ; plus souvent on en voit sur lesquels ou dans lesquels se trouvent des morceaux de fer météorique. A en juger d'après de nombreuses coupures minces, les chondres sont diversement composés. Le plus souvent on trouve une structure fibreuse rayonnante excentriquement, de façon que depuis un point situé dans la partie amincie et éloignée du centre rayonnent des faisceaux vers la périphérie. Les coupes dirigées suivant les plans les plus divers laissent toujours reconnaître dans la substance rayonnante une ordonnance en colonnettes, en aiguilles, en feuilles ou en lamelles ; on peut en conclure que les chondres sont en effet formés par des colonnettes fibreuses. En correspondance avec cette manière de voir, on aperçoit sur certaines coupes, dirigées en angle droit sur les fibres longitudinales, des champs irrégulièrement anguleux et excessivement petits, comme si le tout était composé de petits granules polyédriques. Quelquefois aussi les chondres présentent un aspect comme s'ils étaient composés de plusieurs systèmes rayonnant vers différentes directions. On dirait que le centre de rayonnement aurait changé pendant leur formation, ce qui produit sur certaines coupes une structure d'apparence confuse. La structure fibreuse devient obscure ver l'endroit de la périphérie où se trouve le point de réunion du faisceau rayonnant ; elle est ici remplacée par une structure d'agglomération granuleuse. Sur aucun des nombreux chondres coupés, si toutefois ils étaient entiers, je n'ai pu observer que les faisceaux se seraient prolongés immédiatement jusqu'au bord comme si leur point de réunion était situé en dehors du globule. Les colonnettes élégamment articulées en travers ne s'étendent pas, dans la plupart des cas, de la même manière dans toute la longueur du faisceau ; elles deviennent plus pointues, se ramifient et se terminent en faisant place à d'autres, de manière que les coupes en travers présentent des dessins variés à mailles réticulées. Les colonnettes sont formées, comme cela a déjà été dit, d'un noyau plus clair et d'une enveloppe plus foncée ; le premier est plus ou moins attaquable par les acides, tandis que l'enveloppe résiste davantage. \fg (Suivant mes observations, les colonnettes résistent à l'action de l'eau régale bouillante tandis qu'une partie de la substance servant d'enveloppe est dissoute déjà par l'acide chlorhydrique seul.) \og Les incrustations enveloppantes, qui dans la règle ne s'étendent que sur une petite partie des globules et paraissent composées de fer météorique sont très remarquables. Les mêmes incrustations unilatérales, visibles comme des traits courbés en arc se trouvent aussi dans l'intérieur des chondres et fournissent une forte preuve contre la supposition d'une genèse des chondres par usure d'un matériel quelconque. Toute la disposition de la structure rayonnante des chondres parles du reste d'une manière décisive contre cette supposition. Mais tous les chondres ne sont pas excentriquement rayonnants --- plusieurs, surtout les plus petits montrent une structure finement granulée, comme s'ils étaient composés d'une masse pulvérulente pétrie en boule. Mais même dans ce cas la conformation unilatérale des globules est indiquée par une condensation excentrique plus considérable des particules pulvérulentes. \fg (Gümbel l. c. p. 58 suiv.)

J'ai voulu cette description en entier parce qu'elle correspond assez, sauf les points indiqués, à mes observations propres et parce qu'elle ne donne que des faits observés sans opinion préconçue et sans autre explication plus ou moins hypothétique. Minéralogiste et géologue consommé, M. Gümbel est parti de l'étude de quelques météorites tombées en Bavière pour s'élever à des généralités qui trouvent partout une facile application.

Je dois citer ici un fait étrange. M. Gümbel a étudié aussi les météorites charbonneuses de Bokkeveld et de Kaba. \og J'espérais, dit-il (p. 71), que par le moyen de coupes minces je pourrais peut-être découvrir dans la masse charbonneuse une trace de structure organique. Cette masse montre dans les endroits rares où elle peut être rendue transparente, la structure membraneuse ou finement granuleuse que l'on rencontre ailleurs sur des substances semblables... \fg \og Je n'ai pu découvrir aucune indication de structure organique... \fg Il répète, en parlant de la météorite de Kaba : \og Aussi cette météorite charbonneuse, traitée par la méthode indiquée (traitement par le chlorate de potasse et par l'acide nitrique ensuite), ne montre aucune trace de structure organique. Peut-être réussira-t-on de trouver en employant le même procédé sur des masses plus considérables ou sur d'autres météorites charbonneuses, les preuves de l'existence d'êtres organiques sur des corps célestes en dehors de la terre. \fg (L. c. p 72.)

Dans son ardeur de trouver des partisans, M. Hahn cite cette phrase de la manière suivante : \og M. Gümbel termine ainsi sa description de la météorite de Kaba : \og Peut-être réussira-t-on de prouver l'existence d'êtres organiques sur des corps célestes en dehors de la terre. \fg J'espère, ajoute M. Hahn, que j'ai réussi! \fg

N'est-il pas étrange, que M. Hahn passe sous silence la restriction, fort sage du reste, que pose M. Gümbel en fondant ses espérances uniquement sur les météorites charbonneuses ?

Je passe à mes observations.

Outre une collection de plusieurs centaines de coupes fines de roches diverses formée depuis longtemps, le matériel à ma disposition m'a été fourni de la manière la plus obligeante par MM. de Hochstetter et Brezina (un bel échantillon entier de Knyahinya), par M. Daubrée (Péridot et Enstatite artificiels formés par fusion ; météorites de Vouillé et de Knyahinya), par M. de Marignac (une douzaine de chondrites d'origines diverses), et par M. Stanislas Meunier (Enstatite artificielle en givre). --- N'ayant pas l'intention de donner des descriptions de ces différentes météorites, je me bornerai à celle Knyahinya et subsidiairement à celle de Vouillé, qui fournissent matière suffisante pour le but que je me propose.

La première question que je devais me poser est celle-ci : La méthode de recherches, suivie exclusivement par M. Hahn et ses amis, est-elle exempte d'erreurs possibles ?

Réponse négative. En effet, les structures observables sur des organismes vivants et fossiles se maintiennent jusque dans les coupes les plus minces et deviennent même plus saisissables à mesure que la coupe devient plus transparente ; --- en revanche, les structures observées par M. Hahn ne sont visibles, dans la plupart des cas, comme il le dit lui-même, que sur des coupes semi-transparentes et disparaissent lorsqu'on pousse le travail plus loin. Il fallait donc rechercher à quoi tient cette différence fondamentale ; il fallait chercher, en outre, s'il n'était pas possible de contrôler les résultats fournis par l'observation microscopique des plaques minces, en employant d'autres moyens d'exploration.

Je prie de croire que je n'ai pas négligé la simple inspection de coupes minces et que les meilleurs instruments de Leitz, de Seibert et Krafft, de Verick et de Zeiss m'ont servi à tour de rôle. Je n'aurais pas mentionné ce détail, absolument insignifiant, parce que chacun a maintenant un bon microscope, si l'on n'avait pas appuyé d'une manière tout à fait particulière dans un article populaire sur l'excellence de l'instrument avec lequel M. Hahn fait ses observations.

Il ne fallait pas pousser bien loin l'examen des coupes faites dans le plan du rayonnement, pour reconnaître que les chondres étaient composés, comme l'a dit Gümbel, de colonnettes cristalloïdes, souvent simples, souvent aussi ramifiées, les branches partant dans ce dernier cas sous des angles très aigus et diminuant alors progressivement d'épaisseur depuis le point de départ excentrique vers la périphérie. Dans la plupart des cas, ces colonnettes étaient parfaitement droites, dans d'autres elles étaient légèrement courbées. M. Hahn revenant plusieurs fois, dans son livre, comme dans sa réponse à M. Rzehak (Ausland 1881. N$^{\circ}$ 26, p. 506) sur l'axiome, que des lignes courbes ne peuvent se présenter dans le règne minéral, je donnerai, dans un autre mémoire, les figures de quelques groupes et groupes de cristaux courbées, semblables aux frondes de certaines algues et qui se trouvent dans des laves et autres roches cristallines.

Ces colonnettes rayonnantes, ramifiées ou non, plus ou moins épaisses, présentent toujours des incrustations opaques, non transparentes sur les coupes les plus fines et persistantes en grande partie malgré l'action des acides. Cette matière incrustante et fortement adhérente, remplit tous les interstices des colonnettes et pénètre dans les séparations transversales très fréquentes et souvent régulières des colonnettes de manière à simuler des cloisons. Ces cloisons sont souvent distancées d'une manière tellement régulière qu'on croirait voir, à ne considérer qu'une seule colonnette, des filaments d'algues. On observe aussi que la substance opaque incrustante n'est pas partout d'égale épaisseur ; là où elle paraît moins opaque, on voit des rugosités, des petits creux, des trous même plus profonds qui pénètrent dans la substance parfaitement limpide des colonnettes et qui sont remplies par la substance opaque. La substance transparente des colonnettes est presque toujours rugueuse, comme rongée, marquée de mille accidents divers et toujours ces creux et guillochages sont revêtus de matière incrustante.

MM. Weinland et Hahn insistent beaucoup, soit sur la disposition quelquefois régulière de ces cloisons apparentes, soit sur leur nature comme cloisons. Ce ne sont pas des cassures, ce sont des cloisons ; une cassure forme une simple ligne, est \og un phénomène d'optique \fg ; ici, ce sont \og des cloisons corporelles \fg. J'avoue que je ne saisis pas la différence entre une cassure, dont les deux parois sont faiblement écartées et dont l'intervalle est remplie par une matière opaque et une cloison corporelle. Pour montrer que des cassures plus ou moins régulièrement distancées se trouvent dans des cristaux simulant des filaments d'algues, je donne la figure de pareils cristaux, trouvés dans une coupe mince de diorite, provenant de la rivière de Leith, près d'Edimbourg (fig. 4). Dans la plupart des cas, les bords de ces cassures se correspondent si exactement qu'on ne voit qu'une seule ligne ; dans d'autres, plus rares, on voit deux lignes parallèles ; l'intervalle est alors rempli par une substance vitreuse parfaitement claire et limpide. Que la substance de remplissage soit un peu opaque, on verra une cloison corporelle ayant une épaisseur mesurable. Je donnerai plus loin les preuves, soit par l'observation de coupes désagrégées, soit par l'analyse des morceaux résultant de l'action des acides, que telle est on effet l'explication réelle des cloisons \og ayant corps \fg.

Une seconde particularité sur laquelle insistent les organisateurs des chondrites repose sur le fait que les colonnettes sont des véritables tubes ronds, formés par une paroi opaque et par un contenu clair, un remplissage d'olivine ou d'enstatite. Suivant eux, la substance incrustante opaque serait donc le squelette primitif de l'animal, tandis que la substance claire des colonnettes formerait le moule des cavités, remplies autrefois par la substance molle et détruite de l'animal.

Nous posons en fait, que tout corps transparent, que ce soit un dodécaèdre ou un prisme allongé à pans rectilignes, paraîtra rond sous le microscope par la lumière transmise, lorsqu'il est entouré par une substance plus opaque. C'est un phénomène élémentaire et qui s'explique parfaitement par la disposition de la substance entourant, qui laisse passer au milieu une plus grande quantité de lumière que sur les bords, où elle se présente avec une épaisseur plus considérable. Des ombres diminuant graduellement vers un centre ou vers une ligne, augmentant graduellement vers le bord nous donnent l'impression d'une bosse ronde à surfaces courbes. Ceci arrive d'autant plus facilement, lorsque les pans des bords se rencontrent sous des angles émoussés. Or, de même comme les enstatites massives présentent des angles tellement émoussés qu'elles paraissent rondes, de même aussi les prismes allongés des enstatites paraissent arrondis et complétement ronds, lorsqu'ils sont entourés par une matière plus opaque comme par une gaine.

A ces difficultés, inhérentes à la nature des objets, s'en ajoute une autre. Dans la plupart des chondres, les colonnettes sont tellement exiguës et minces, qu'il devient matériellement impossible de faire des coupes n'ayant que l'épaisseur d'une seule colonnette. Toutes les coupes, même les plus minces, contiennent donc plusieurs couches superposées de colonnettes. On conçoit aisément que ces corps superposés, transparents, mais incrustés d'une matière opaque et dont les bords ne se correspondent pas dans leur superposition, doivent nécessairement produire des effets d'ombres fallacieux et souvent indéchiffrables. Un interstice opaque entre deux colonnettes sous-jacentes et placé dans l'axe médian de la colonnette qui se trouve dans le foyer de la lentille du microscope, donnera à cette colonnette une apparence comme si elle était percée par un canal longitudinal ; des cloisons placées un peu obliquement par rapport à l'axe de la colonnette, entre lesquelles se placent les ombres produites par des cloisons sous-jacentes, donneront à la colonnette l'air d'être disposée en chapelet. Avec la meilleure volonté du monde et malgré l'emploi des meilleurs instruments, on ne peut arriver à vaincre toutes ces difficultés ; je dirais même que plus on est exercé à l'observation microscopique, plus on se persuade que des certitudes ne peuvent être acquises.

J'ai essayé la lumière polarisée, dont on ne doit jamais négliger l'emploi, lorsqu'il s'agit de l'étude de minéraux ou de roches ; les résultats n'étaient pas assez concluants pour écarter tous les doutes. J'exposerai ces résultats plus tard dans leur ensemble.

J'étais donc arrivé à la conclusion que la méthode d'observation, exclusivement employée par M. Hahn et ses amis, savoir, l'étude microscopique de coupes minces, ne pouvait conduire à des résultats certains. M. Hahn voit toute la masse des chondrites composée d'organismes ; M. Weinland n'en voit que dans les deux tiers ; M. Rzehak (Ausland, 1881, N$^{\circ}$ 26) n'en voit pas du tout, et examen fait de tout, je devais me ranger de l'avis de ce dernier observateur.

Il fallait donc rechercher d'autres méthodes et d'autres comparaisons.

M. Gümbel avait déjà montré le chemin. Il a toujours eu soin de contrôler ses observations sur les coupes minces par des opérations microchimiques. En parlant de la météorite de Mauerkirchen (20 Nov. 1768) il dit (l. c. p. 19) : \og Après avoir traité le matériel finement écrasé (non pulvérisé) par de l'eau régale et de la potasse caustique, je vis que les parties métalliques et les esquilles jaunâtres (olivine) avaient disparu et que le reste consistait en morceaux blancs ou brunâtres qui se laissent facilement distinguer sous le microscope. Les fragments brunâtres sont fortement fendillés, rarement pourvus de traces de stries parallèles opaques ; ils sont transparents et vivement colorés de taches multicolores par la lumière polarisée. Ce sont sans doute des fragments d'un minéral du groupe des Augites. Les esquilles blanches, au contraire, sont souvent seulement translucides, en partie attaquées par les acides et ne montrent, à la lumière polarisée, que des couleurs mates disposées en taches qui par-ci et par-là indiquent une disposition par bandes. \fg Et en parlant de la météorite de Krâhenberg (5 Mai 1869) (l. c. p. 57). On voit dans une coupe mince traitée à l'acide chlorhydrique et tenant encore ensemble des lacunes nombreuses, plus ou moins grandes, indiquant la place des matériaux dissous par l'acide. En traitant cette coupe ensuite avec une solution de potasse caustique, on la désagrège en petits morceaux, granules et parties pulvérulentes, parmi lesquels les esquilles provenant des inclusions plus grandes se distinguent par leur plus grande consistance. Il est très remarquable que dans les morceaux ayant une structure réticulée à stries, quand même ils tiennent encore ensemble, les stries transparentes sont complétement détruites et que les lamelles intermédiaires opaques sont seules conservées et se présentent comme un squelette. On peut mettre ce fait hors de doute par l'examen à la lumière polarisée. \fg

J'ai suivi cette méthode. J'ai traité des coupes, j'ai traité des chondres écrasés, non pulvérisés et comme c'est la météorite de Knyahinya qui seule a fourni toutes les formes décrites et figurées par M. Hahn, j'ai choisi cette météorite pour mes expériences.

Après avoir écrasé les fragments en petits morceaux d'un millimètre de diamètre de diamètre environ, j'ai épuisé avec de l'acide chlorhydrique bouillant cette grenaille, dans laquelle beaucoup de chondres se laissaient encore voir presque intactes avec leurs surfaces hérissées de petites pointes cristallines. Il y a un dégagement assez tumultueux d'hydrogène sulfuré, preuve de la présence de pyrites ; le fer dissous colore l'acide en jaune verdâtre. J'obtiens un précipité léger nuageux, presque gélatineux, qui se dépose très lentement, tandis que de petites particules brillantes et incolores tombent rapidement au fond et forment comme une farine blanche qui recouvre les grains restés entiers au fond de l'éprouvette.

Examiné au microscope, le précipité léger et nuageux se présente comme une substance amorphe avec des granules extrêmement fins en poussière. C'est évidemment de la silice amorphe. Quelques trichites assez rares, très noirs et très fins se rencontrent disposés en touffes au milieu de cette masse, --- je les attribue à des parcelles de la croûte de fusion, dont une partie était encore attachée au fragment analysé. Le précipité blanc, lourd et farineux, est composé au contraire en totalité de petites pièces cristalloïdes, dont je donnerai la description plus loin.

Outre les pyrites et les métaux dissous, l'acide chlorhydrique a donc disjoint quelques particules extrêmes des colonnettes en dissolvant et en décomposant un silicate incrustant probablement riche en fer.

J'attaque avec de l'eau régale bouillante. Dégagement tumultueux d'acide nitreux ; l'acide est de nouveau coloré en jaune par du fer. L'eau régale a donc dissous un autre silicate ferrique plus résistant à l'attaque. Plus de précipité nuageux ; mais le précipité farineux a augmenté. Les grains restants sont d'un gris sale, hérissés d'aspérités.

J'examine ce précipité farineux au microscope après l'avoir préparé au baume.

Je vois immédiatement que sur la plupart des débris la matière opaque incrustante n'a pas complétement disparu. Il doit donc y avoir une substance, probablement un silicate, contenant du fer ou un autre métal, qui est insoluble dans les acides les plus forts. Mais la matière incrustante a partout considérablement diminué, et je trouve une quantité de petites pièces qui sont entièrement nettoyées et transparentes comme de l'eau, tandis que d'autres présentent une opacité plus grande.

Les piécettes isolées et transparentes sont prismatiques, allongées, à pans terminaux coupés verticalement dans quelques cas ; mais le plus souvent elles montrent à leurs extrémités des facettes sur lesquelles étaient sans doute articulées des pièces encore plus petites (fig. 5 et 12-15). Les côtés des prismes sont rugueux ; on y voit ordinairement des petites impressions ou des creux plus profonds, dans lesquels persiste encore un peu de matière opaque ; dans d'autres cas, ces pans sont parfaitement rectilignes, mais les angles sous lesquels ils se rencontrent paraissent arrondis. Des facettes semblables à celles des bouts se montrent aussi par-ci et par-là sur les côtés des prismes ; elles servaient sans doute à l'articulation de petits cristaux latéraux placés en bifurcation. De nombreuses fissures transversales et longitudinales se font surtout remarquer sur les pièces plus grosses (fig. 5) ; très souvent ces fissures transversales se montrent béantes, présentant une échancrure au bord, tandis que dans l'intérieur du morceau elles se présentent comme des \og cloisons ayant corps \fg ; on voit distinctement que ces fissures sont encore remplies par la substance incrustante qui colle ensemble les fragments séparés par la fissure. Il n'y a pas un seul morceau clair et transparent qui ne montrerait à l'évidence la structure cristalline. La masse constituante claire ne paraît cependant pas toujours entièrement homogène ; on y voit des dessins nuageux, quelquefois pointillés sans forme définie. Toutes ces petites pièces claires, légèrement colorées en jaune quelquefois, réfractent fortement la lumière ; leurs contours sont nettement accusés. Par la lumière polarisée elles montrent aux Nicols croisés les plus belles couleurs disposées par petites taches irrégulières.

Je réserve la description des morceaux plus composés offrant une structure réticulée et fibreuse, semblable à celles des chondres, pour plus tard.

Je divise le reste de la matière, traitée successivement par les deux acides indiqués en deux portions et je traite l'une de ces portions avec de la potasse caustique, tandis que j'attaque l'autre avec de l'acide sulfurique concentré.

L'acide sulfurique concentré n'a plus aucune action ; la potasse caustique au contraire décompose encore une partie. Il se forme la même substance presque gélatineuse, qui se dépose très lentement et le même précipité farineux, comme par l'action des acides employés en premier lieu. En dernier lieu, il reste un dépôt grisâtre de substance non décomposée, lequel peut-être aurait été réduit aussi, si j'avais continué la cuisson plus longtemps. Le précipité farineux est entièrement composé d'esquilles très fines, cristalloïdes, réfractant fortement la lumière et brillant, sous les Nicols croisés, d'une lumière blanche un peu bleuâtre. Le dépôt gris montre encore des restes des chondres tenant ensemble. Mais la matière incrustante est beaucoup raréfiée et ces pièces brillent, sous les Nicols croisés, des plus belles couleurs de l'arc-en-ciel. J'en ai dessiné une dans cet état (fig. 6). C'est une preuve de plus que l'apparition des couleurs de double réfraction au polariscope est simplement empêchée par la présence de la matière incrustante opaque.

Les petites esquilles et menus fragments, dans lesquels on peut réduire une coupe fine en l'usant jusqu'à la dernière limite, montrent absolument les mêmes formes, comme ceux produits par l'action des acides, avec cette différence, cependant, que les parties opaques de fer pyriteux et magnétique s'y trouvent encore et que la matière incrustante est conservée en entier. La plus grande partie de ces esquilles est composée de cristaux évidents, transparents, quelquefois colorés en jaune, réfractant fortement la lumière et se parant de belles couleurs par la lumière polarisée aux Nicols croisés. Ces cristaux sont toujours fissurés dans tous les sens et souvent désagrégés, de manière à montrer les fissures encore remplies de matière incrustante. Celle-ci pénètre aussi dans des petits trous ronds plus ou moins profonds, qui produisent, suivant le rapprochement ou l'éloignement du foyer, l'impression de bulles, de trous ou d'anneaux ; on voit souvent attachés à leurs extrémités des petits cristaux prismatiques ou pointus. Je donne un dessin d'un de ces cristaux (fig. 6). Outre ces cristaux, il y a aussi des fragments des masses fibreuses à colonnettes, comme dans les pièces désagrégées par les acides et sur lesquels je reviendrai.

Un premier point à constater ici, c'est que, contrairement à l'assertion de M. Hahn, la plus grande partie de la météorite de Knyahinya est manifestement composée de cristaux, réfractant la lumière et décomposant la lumière polarisée. \og Si (les chondrites) étaient des cristaux, dit M. Hahn (p. 23), et si la fissuration lamellaire était la cause de la structure, le minéral devrait nécessairement réfracter la lumière. Mais dans la plupart de ces inclusions ne se montre aucune réfraction, ni même de la polarisation d'agrégation! Elles ne peuvent donc être ni des minéraux simples ni des cristaux, encore moins pourrait-on expliquer la structure par des fissures lamellaires. Ce fait seul, la qualité optique, aurait dû conduire à l'interprétation juste. \fg

J'ai déjà dit que M. Hahn considère la présence de cristaux dans les météorites comme un fait très exceptionnel ; dans celle de Knyahinya ils doivent même faire complétement défaut suivant lui, car il attribue la totalité par vingtièmes à des organismes. Or, je maintiens que la même météorite de Knyahinya se décompose par l'action des acides, de la potasse et de l'usure mécanique en cristaux évidents, réfractant et décomposant la lumière et que ces cristaux et fragments de cristaux forment la plus grande masse des esquilles obtenues par les deux méthodes décrites. Ces cristaux, lorsqu'ils sont un peu plus gros, réunis et collés ensemble en groupes par la matière incrustante, se font du reste aisément remarquer dans les coupes fines, et je donne une figure d'un groupe semblable tiré de la météorite de Vouillé (fig. 8), où ils sont en général plus gros que dans celle de Knyahinya. J'ai cependant trouvé dans plusieurs coupes de la météorite préférée par M. Hahn des groupes semblables. Dans l'échantillon du Musée de Vienne que j'ai fait détailler, j'ai constaté, engagé au milieu de la masse, un chondre ovalaire, gros comme un petit pois, de la longueur d'un centimètre et large de sept millimètres, qui était entièrement composé de cristaux traversés par des fentes peu accusées, mais nombreuses, dans lesquelles on pouvait à peine apercevoir de la matière incrustante. Le chondre avait une couleur presque blanche, à peine grisâtre ; sa surface était raboteuse et sur la partie de cette surface, qui s'était dégagée lors du polissage de la gangue environnante, on remarquait des petites bosselures noires, semblables à des morceaux de scories. A la lumière polarisée, ces cristaux prenaient des couleurs passant d'un ton verdâtre, cadavérique, mais très lumineux, à des teintes jaunes brunâtres et brun-rougeâtres.

Ces groupes de cristaux fendillés, traversés par des \og cloisons ayant corps \fg, se présentent du reste dans les météorites absolument sous le même aspect que dans les enstatites artificielles obtenues par M. Daubrée au moyen de la fusion du péridot avec 15\% de fer doux et dont je suis redevable à l'obligeance de mon savant ami. Dans ces enstatites artificielles (fig. 9 et 10) le fer en excès a joué le même rôle que la matière incrustante dans les météorites ; il a rempli les interstices et les fissures. Autour de gros cristaux presque globulaires, qui souvent ont sauté lors de l'usure en laissant un vide obtusément angulaire, se trouvent des groupes de cristaux agglomérés. Or, c'est sur cette substance assez dure pour rayer le verre que j'ai observé un fait qui donnera, je pense, l'explication de nos assertions si diamétralement opposées de M. Hahn et de moi. Une coupe très fine de cette substance (fig. 9), transparente et usée jusqu'à la dernière limite, présente sous les Nicols croisés, les plus belles couleurs jaunes, bleues et rouges, disposées par taches. On ne pourrait trouver une meilleure substance pour faire la démonstration de l'action de la lumière polarisée. Du même morceau j'ai fait faire des coupes un peu plus épaisses, translucides ou semi-transparentes (fig. 10) ; elles ne montrent, sous les Nicols croisés, à côté de quelques cristaux fortement colorés, que par-ci et par-là quelques pâles taches colorées, à peine perceptibles. C'est exactement la même chose comme dans les météorites ; sur les coupes fines de Knyahinya comme de Vouillé, qui montrent des images telles que les a présentées M. Hahn, et qui sont donc usées jusqu'à sa limite seulement, je vois à peine quelques pâles taches colorées très petites ; sur les coupes usées entièrement et sur les fragments détachés je les vois répandues partout et brillantes dans tous les éclats. Il est donc évident que la superposition des cristaux munis de leurs incrustations opaques, empêche la perception des rayons colorés engendrés par la lumière polarisée.

Un autre exemple confirmera ce que je viens de dire. Une coupe mince de la météorite de Vouillé présente sur un de ses bords un chondre mesurant environ deux millimètres dans son plus grand diamètre et que j'ai représenté fig. 11. Cette coupe ferait sans doute les délices d'un observateur croyant aux organismes. Un noyau central, sur lequel on voit seulement un fin pointillage et une partie rendue moins claire par mille fines lignes croisées, est entouré par une frontière plus opaque, de laquelle partent en rayonnant des fines lignes présentant souvent des ramifications et qui se continuent jusqu'au bord, entouré par une ceinture en demi-cercle de substance complétement noire. La masse entièrement transparente de ce chondre est en outre traversée par quelques crevasses rayonnantes remplies également par la substance noire. Sur une place, la masse incrustante s'est détachée complétement et montre manifestement la forme d'un canal cylindrique. J'ai désigné ce canal par la lettre a dans la fig. 11 ; en l'observant sous un très fort grossissement, le bout central (b de la même figure) se montre bien sous la forme de l'orifice d'un canal taillé en biseau. Les fines lignes rayonnantes sont tellement minces, que les plus fortes lentilles à immersion les font seulement paraître comme un trait. C'est donc une Urania modèle, suivant les figures données par M. Hahn. Or, toute cette partie fibreuse, dans laquelle on ne voit aucune trace de cloisons transversales, montre sous les Nicols croisés des séries rayonnantes de taches presque carrées, infiniment petites, de couleurs rouge et bleue alternantes. Ici, dans cet objet, la matière incrustante est tellement mince qu'elle n'exerce pas d'influence quant à l'absorption des rayons polarisés. Un morceau détaché c donne, comme nous verrons plus tard, l'explication du dessin coloré fourni par le polariscope.

Je reviens à la météorite de Knyahinya traitée par les acides ou usée jusqu'à la réduction en esquilles. Je disais qu'outre les cristaux immédiatement reconnaissables, qui composent la majeure partie des fragments, il s'en trouve d'autres qui sont moins transparents et présentent cette structure à tubes ramifiés, à cloisons transverses \og ayant corps \fg, que M. Hahn considère comme décisive pour la nature organique des chondres. Je donne (fig. 12-15) quelques dessins de plusieurs fragments ; l'un (fig. 12) représente quelques pièces encore assez grosses, sur lesquelles sont posées quelques petites pièces presque cylindriques ou prismatiques à angles émoussés ; dans deux autres (fig. 13 et 14), chacun reconnaîtra aisément la structure des crinoïdes à bras ramifiés, telle que l'a représentée M. Hahn. Or, partout où ces éventails minimes tiennent encore ensemble, ou voit les pièces articulées, séparées par des cloisons \og ayant corps \fg comme arrondies par des ombres légères latérales ; mais là où les extrémités libres des colonnettes se présentent, elles sont à bords et angles vifs et nettement arrêtés. Examinés au polariscope, ces fragments à structure organique ne montrent aucune réaction aussi loin qu'ils font corps ; mais les extrémités libres présentent les couleurs des substances à double réfraction.

La composition par cristaux est plus manifeste dans d'autres fragments à structure lamellaire, comme j'en ai figuré un fig. 15. Les interstices sont remplis par la matière incrustante qui entre dans les fissures longitudinales et transversales, dans les creux et les pores des pièces claires qui semblent posséder une structure lamellaire prononcée, comme si des petites planchettes minces et longues étaient collées ensemble, en présentant souvent leurs faces étroites. Ces fragments ont comme ensemble la même couleur grisâtre que les précédentes ; ils ne présentent point de changements par les Nicols croisés ; mais leurs extrémités taillées en biseau ou en gradin, qui dépassent la matière incrustante, brillent des plus vives couleurs.

Enfin, il reste par l'action des acides des chondres non décomposés, globulaires, hérissés d'aspérités, grands comme des têtes d'épingles, que j'ai préparés au baume dans une cellule à parois épaisses d'un millimètre. Le corps de ces chondres est, cela va sans dire, absolument opaque sous le microscope, tandis qu'à la lumière directe ils présentent une couleur d'un gris-clair. Mais les aspérités, dont ils sont hérissés, sont pour la plupart transparentes, taillées en angles vifs et les Nicols croisés y font apparaître les taches colorées.

J'ai tenu à rapporter tous ces détails, fastidieux peut-être, parce qu'ils éclairent, il me semble, la question d'une manière positive. Grâce à l'analyse par les acides et par l'usure, je puis maintenant affirmer, sans crainte de contradiction sérieuse, que les morceaux de Knyahinya que j'ai examinés et qui sont aussi authentiques que l'échantillon, sur lequel M. Hahn a trouvé \og des centaines de structures organiques \fg, ne contiennent, outre les esquilles métalliques et les parties peu considérables pulvérulentes, que des cristaux, rien que des cristaux, diversement développés en grandeur, agencés, agglomérés, agglutinés de différentes manières. Je puis affirmer avec certitude, que toutes les soi-disant structures organiques sont produites par des cristaux appartenant au moins à une espèce, peut-être même à plusieurs espèces minérales à simple et double réfraction.

On pourrait soulever l'objection que les organismes ont été détruits par les acides et que les cristaux seuls ont résisté. Il est facile d'écarter cette objection par les raisons suivantes : 1$^{\circ}$ Des fragments à prétendue structure organique et des chondres presque entiers ont résisté aux acides, mais en dévoilant leur structure cristalline par la raréfaction de la substance incrustante ; 2$^{\circ}$ l'action mécanique du polissage jusqu'à la dernière limite a produit les mêmes effets.

Arrivé à ce point de mes recherches, je devais nécessairement me demander si des formes analogues ou identiques avec celles des chondres pouvaient être démontrées, soit dans des productions artificielles, soit dans des roches naturelles. Quant aux premières, je ne pouvais m'adresser qu'à MM. Daubrée et Stanislas Meunier, ces deux savants étant les seuls qui se soient occupés d'expériences relatives à la genèse des météorites. Je dois remercier ces messieurs qui ont mis à ma disposition, avec la plus grande amabilité, un matériel considérable.

J'ai déjà donné la description des enstatites artificielles, produites par M. Daubrée, par la fusion des péridots avec du fer doux. On peut comparer les dessins d'une coupe très fine de ce produit (fig. 9) et celui d'une autre moins mince (fig. 10), avec la reproduction (fig. 8) d'une partie de la météorite de Vouillé ; il est impossible de trouver des échantillons d'un même minéral plus ressemblants. M. Daubrée était donc parfaitement fondé de dire, que par son procédé de fusion, décrit déjà en 1866, il avait produit des formes et des agrégations semblables à celles qui se trouvent dans les météorites. Tout, forme, interstices remplis d'une matière incrustante, qualités optiques, tout se correspond exactement. Il n'y a de différence que dans la couleur ; les cristaux de la météorite de Vouillé sont un peu teintés en jaune, tandis que ceux du produit artificiel sont incolores. La couleur jaune est presque toujours produite par infiltration de fer ; en suivant ces taches, on arrive presque infailliblement à une esquille noire de fer météorique qu'elles entourent comme une auréole. Des groupes semblables de cristaux nous sont donnés par M. Hahn (Tab. 21, fig. 5 ; Tab. 22 fig. 1 et 2) comme des parties de crinoïdes.

Les produits de la fusion de la lherzolite avec du fer doux, obtenus par M. Daubrée, renseignent sur un fait invoqué avec beaucoup de force par MM. Hahn et Karsten (Natur. 1881, N$^{\circ}$ 16). J'ai déjà fait remarquer combien les formes microscopiques de ces produits dont je donne des dessins (fig. 16 à 18) sont curieuses. Des longs bâtonnets clairs, mais ornementés de la façon la plus diverse, circonscrivent des champs anguleux, occupés par une substance transparente, dans laquelle rayonnent des fibres brunes, extrêmement déliées, qui, sous une lentille à immersion, se présentent comme des lignes croisées ou comme des chapelets. Ces fibres rayonnent tantôt depuis un centre, tantôt elles forment des figures de plumes ; dans la plupart des cas, elles sont droites, mais on en remarque aussi qui présentent une faible courbure. Sous les Nicols croisés, ces champs avec leurs fibres ne montrent aucun changement, tandis que les bâtonnets brillent des couleurs les plus vives.

Je donne deux figures de ces bâtonnets, dessinés sous un grossissement de 500 diamètres (fig. 17 et 18). J'aurais pu en donner cinquante figures et davantage, car, examiné en détail, chacun de ces bâtonnets montre une structure différente et souvent même l'aspect de cette structure change plusieurs fois dans la longueur du bâtonnet. Ici, ce sont des fines hachures croisées ; là, des aspérités qui donnent au bâtonnet l'air d'être hérissé de poils ; dans un autre endroit, on voit des pièces en forme d'ancres ou de crampons placés sur ces bâtonnets ou des mamelons peu élevés en forme de stomates ou de pores cellulaires. M. Hahn et ses adhérents parlent toujours du \og manque de structure \fg dans les minéraux ; je ne connais pas de parties organiques, qui présentent une structure plus compliquée que ces bâtonnets produits artificiellement. On invoque également des pores, des trous sur les colonnettes des chondres, comme preuves évidentes que des canaux latéraux se séparaient dans ces endroits des canaux principaux, que M. Hahn attribue à des coraux, tandis que M. Karsten y voit plutôt des filaments d'algues d'une Hystérophyme (\emph{Leptomitus} ou \emph{Leptothrix}). (Natur. 1881, N$^{\circ}$ 16, p. 184.) \og C'est, en tout cas, dit M. Karsten, un corps organisé, car des cristaux véritables, qui se forment dans des solutions qui s'évaporent ou se refroidissent, sont homogènes et sans structure. \fg On n'a qu'à examiner mes deux dessins pour voir que des cristaux formés dans une masse fondue qui se refroidit peuvent présenter une structure des plus compliquées, qui se manifeste aussi par le polariscope. Le bâtonnet à pores, qui ressemblent en quelques endroits aux cicatrices des feuilles, telles qu'elles se présentent sur les troncs des fougères et des sigillariés, montre sous les Nicols croisés une série de mamelons accusés, au milieu desquels se dessine un espace clair comme un trou. Tous ces bâtonnets présentent, sous les Nicols croisés, les couleurs les plus éclatantes.

Si les formes cristallines, semblables à celles produites par M. Daubrée au moyen de la lherzolite fondue, sont assez rares dans les météorites, il ne faut pourtant pas en conclure qu'elles en soient complétement exclues. Je compte, en effet, parmi les passages de la structure chondritique ramifiée vers celle de la lherzolite, les formes suivantes, toutes observées dans la météorite de Knyahinya :
\begin{enumerate}
\item Des chondres à structure combinée, où au milieu d'une masse presque pulvérulente se dessinent des colonnettes très allongées, articulées, disposées en général comme les rayons d'une roue. J'ai observé un des chondres qui présentait sur une de ces moitiés six rayons très régulièrement espacés, sur l'autre moitié il y avait tout un groupe de cristaux colonnaires, en partie branchus, très serrés et tandis que tous ces rayons partaient d'un centre excentrique, mais pas trop rapproché du bord, on voyait près de ce centre un bâtonnet cristallin d'une longueur considérable, qui traversait tout le chondre de part en part. Sur le côté du grand chondre il y en avait un petit, formé de colonnettes extrêmement fines comme des traits, mais entremêlées de colonnettes rayonnantes plus considérables.
\item Des formes, assez semblables à des plumes. D'un axe central, sur lequel on voit des articulations, partent d'un côté des rayons complétement transparents, comme l'axe lui-même, disposés à des intervalles irréguliers, mais tous parallèles et formant avec l'axe un angle de 40 degrés environ. Les intervalles entre ces axes secondaires sont remplis par des fibres cristallines, disposées à angle droit, comme les barbillons d'une plume ramifiée. De l'autre côté, ces barbillons partent de l'axe lui-même et on y voit quelques espaces plus clairs sans direction fixe. Les barbillons se présentent de la même manière comme les formes fibreuses de l'enstatite artificielle.
\item Enfin, des groupes ressemblant si exactement aux enstatites produites par la fusion de la lherzolithe, qu'on pourrait les confondre ensemble (fig. 23). Des prismes allongés, fissurés à l'infini, disposés sur plusieurs rangs et se joignant ensemble sous des angles obtus, qui circonscrivent un espace presque rond et pourraient bien correspondre aux pans d'un dodécaèdre coupé, entourent un champ traversé par des gros cristaux longs sur la nature desquels on ne peut avoir aucun doute. Dans les espaces laissés entre ces cristaux se sont développées des fines fibres disposées en rayons, se croisant sous plusieurs angles formant des faisceaux. On n'a qu'à comparer les fig. 16 et 23 pour être frappé de la ressemblance du groupement de ces fibres entre les gros cristaux. La réaction sous les Nicols croisés est exactement la même. Il y a donc identité complète entre le produit artificiel et le produit naturel de cette même météorite de Knyahinya, dont les cristaux devaient être rigoureusement exclus. Je dois à la vérité de dire, que M. Hahn a photographié (Tab. 29, fig. 2), un groupement analogue de Knyahinya, où une étoile à six rayons, dont deux ne sont qu'indiqués, tandis que les quatre autres sont formés par des groupes de cristaux parallèles, est aussi entouré par des séries de cristaux allongés, --- mais les interstices entre les rayons sont, dans la figure de M. Hahn, remplis aussi par des cristaux plus gros, tandis que dans échantillon on y voit les fines fibres cristallines de la lherzolite. Pour M. Hahn, c'est un crinoïde vu d'en haut ; je ne pense pas que l'idée de la comparaison avec un crinoïde, vu de quel côté que cela soit, puisse venir à la vue de mon dessin.
\end{enumerate}
\paragraph{}
Quoi qu'il en soit, ces faits prouvent bien que même les formes si étranges de l'enstatite engendrée par la fusion de la lherzolithe, sont en connexion intime avec la constitution de certains chondres des météorites ; qu'il y a des passages gradués, entre ces différentes formes, sous lesquelles les cristaux se sont développés et groupés et qu'entre les assemblages irréguliers de gros cristaux, la disposition colonnaire et enfin celle dendritique ou fibrillaire nous ne pouvons statuer des différences tranchées.

Mais la ressemblance la plus complète avec les chondres articulés et ramifiés est offerte par les givres d'enstatite artificielle, produite par M. Stanislas Meunier dans les expériences qu'il a exposées dans les comptes-rendus (séance du 23 février 1880) et sur lesquels il a de nouveau appelé l'attention dans une récente communication à l'Académie des sciences (séance du 7 novembre 1881).

M. Meunier a insisté sur la ressemblance de ce givre avec les chondres ; M. Rzehak a rappelé cette ressemblance ; M. Hahn et ses amis ont fait la sourde oreille. M. Meunier avait peut-être le tort de ne pas appuyer ses assertions par des figures ; grâce à son obligeance, je suis à même d'y suppléer. Je donne des dessins faits sous un grossissement de 500 diamètres (fig. 19-21) et je pense que personne ne pourra contester, je ne dis pas la ressemblance, mais l'identité avec les figures de fragments de chondres traités par les acides. Ce sont les mêmes colonnettes, le même agencement, le même rayonnement en partant de pièces plus grosses pour former des branches toujours plus déliées, les mêmes cloisons apparentes transversales dans les unes comme dans les autres. Sur l'une de ces figures on constate comme des cicatrices rondes, provenant de branches cassées, qui partaient dans une direction un peu différente (fig. 20 a) ; sur l'autre se voit une ramification étonnante, unilatérale dans quelques endroits (fig. 19) ; une troisième figure enfin (fig. 21), montre le rayonnement depuis un point central, point de fixation de la tige du crinoïde pour M. Hahn (Tab. 29, fig. 4). La plupart des branches sont droites, mais quelques-unes sont manifestement courbées, ce qui, suivant M. Hahn, est un caractère absolu d'une conformation organique. M. Meunier peut se vanter d'avoir produit des organismes par le concours de substances minérales dans un tube, chauffé au rouge sombre! Les cloisons transversales, rigoureusement dessinées à la chambre claire, sont aussi équidistantes qu'elles peuvent l'être dans un filament d'algue ou dans un bras de crinoïde. Toutes les pièces constituant ces aigrettes rayonnantes sont solides, transparentes, sans aucune trace de structure intérieure, comme les piécettes qui sortent des aigrettes produites par la dissociation des chondres.

Les givres à ma disposition étaient des préparations, couvertes d'une lame en verre mince. Mais leur distribution sur différents niveaux démontre déjà que les colonnettes doivent rayonner dans tous les sens et former des flocons en boule. M. Meunier m'informe qu'en effet les givres sortent sous cette forme du tube où ils se sont constitués ; mais que ces flocons sont tellement délicats que la pression du couvre-objet suffit pour les aplatir complétement. J'ai reçu dernièrement un petit tube rempli de givre, tel qu'il sort de l'expérience, et j'ai pu me convaincre qu'il renferme des petits flocons globulaires, composés d'aigrettes rayonnant dans tous les sens.

Je pense que la démonstration est aussi complète que possible. Les chondres de Knyahinya, considérés comme des animaux par M. Hahn, mais débarrassés autant que possible de la matière incrustante, se montrant, comme le disait M. Meunier, composés exactement des mêmes éléments comme le givre d'enstatite artificielle.

Il est donc acquis au débat que des groupes de cristaux plus considérables, réunis par une matière incrustante, qui en remplit les interstices et les cassures de manière à produire des cloisons ayant corps et qui se trouvent à profusion dans la météorite de Knyahinya comme dans les autres chondrites, ont été artificiellement reproduits par M. Daubrée, tandis que les formes rayonnantes, ramifiées et articulées des chondres, ont été préparées par M. Meunier.

La seconde question qui se présentait est celle-ci : Se trouve-t-il dans des roches terrestres des formes semblables à celles qui se rencontrent dans les météorites ?

Si l'on réfléchit bien aux conséquences des expériences de M. Meunier, on doit se dire que les conditions particulières sous lesquelles se sont formés les givres d'enstatite artificielle ne pourraient guère se rencontrer que dans l'action des volcans. Aussi a-t-on comparé les chondres aux globules qui se rencontrent assez fréquemment dans les tufs volcaniques. Mais la différence est grande ; les tufs volcaniques sont engendrés par des cendres ou des rapillis, cimentés par l'eau et ces cendres elles-mêmes résultent de la pulvérisation des laves, c'est-à-dire de roches à demi-fondues, composées de cristaux préexistants et de masses vitreuses en proportions variables. Les tufs ne sont donc pas formés directement dans une atmosphère de vapeur d'eau surchauffée, mais ils sont le résultat d'un remaniement de substances fondues et pulvérisées ensuite. Les conditions de formation ne sont donc pas les mêmes.

Si donc il existe dans les chondres des formes semblables aux givres de M. Meunier, et, si ces formes doivent être attribuées à des causes analogues, nous ne pouvons cependant pas douter qu'il existe dans les météorites d'autres parties qui paraissent devoir leur origine à des causes semblables à celles mises en action par M. Daubrée, savoir, la fusion ou la demi-fusion dans un milieu opérant une réduction. Le gros chondre de Knyahinya, dont j'ai parlé plus haut, me semble un produit manifeste de cristallisations dans un milieu en fusion. Les cristaux qui le composent sont trop serrés les uns contre les autres pour qu'on puisse admettre une autre formation et quelques masses pulvérulentes en forme de culot, qui sont engagés au milieu du chondre, me paraissent aussi militer en faveur de cette opinion. La structure souvent bulleuse, boursoufflée, de la matière incrustante, les mille empreintes, éraillures et érosions des cristaux revêtues par cette matière, qui a pénétré dans les fissures et cassures les plus déliées, parlent aussi en faveur de la cristallisation dans un fluide igné. La surface d'une quantité de cristaux ressemble entièrement par ces accidents divers à celle des cristaux existant dans les laves et il est probable que ces accidents ont une origine analogue. Je ne suis donc pas éloigné d'admettre que le passage immédiat de l'état gazeux à l'état cristallin d'un côté et la fusion plus ou moins accomplie de l'autre, ont joué tous les deux leur rôle dans la formation de chondrites et que, suivant les cas, l'une ou l'autre de ces causes peut avoir engendré des effets plus dominants.

Ce qui me confirme dans cette opinion, c'est l'étude de ce chondre transparent et presque uni de la météorite de Vouillé, que j'ai cité plus haut comme un type modèle d'une Urania de M. Hahn et dont j'ai donné un dessin (fig. 11). Je disais déjà que ce chondre à lignes flexueuses extrêmement déliées montre, sous les Nicols croisés, un arrangement colonnaire ou sérial de petites taches bleues et rouges alternantes. Or, sur l'un des bouts de la préparation, quelques morceaux de ce chondre ont été détachés par le polissage. Ces morceaux (fig. 11, c.) ont été ébranlés par l'action mécanique, leurs joints sont devenus plus apparents et ils se montrent composés d'une succession de fines colonnettes articulées, traversées par des cloisons nombreuses et courant parallèlement sans ramifications. La structure indiquée par la lumière polarisée a donc été rendue manifeste par les chocs et les ébranlements mécaniques. Dans ce morceau aussi, les Nicols produisent un effet merveilleux. J'ai trouvé, sur une coupe de la météorite de Knyahinya, un fragment présentant absolument la même apparence.

Ces morceaux d'Uranias manifestes ressemblent, à s'y méprendre, à un fragment d'enstatite détaché aussi par l'action du polissage, d'une grosse masse que je trouve dans une coupe mince du fameux \og Schillerfels \fg de Baste dans le Harz. La partie dont ce morceau s'est détaché ne montre aucune trace d'une structure colonnaire ; on y voit des bandes minces d'un brun grisâtre, à bords indécis et un peu flexueuses. La masse entière présente une striation tout aussi fine que le chondre de Vouillé. Ni le polariscope, ni des grossissements plus forts n'enseignent quelque chose de plus sur la structure de cette masse. Mais le fragment détaché par le choc du polissage (fig. 22) laisse voir la structure colonnaire la plus évidente, entièrement semblable à celle du fragment du chondre de Vouillé et, disons-le de suite, aussi à celle d'un fragment de chondre de Knyahinya, dessiné (fig. 15). Ce sont les mêmes colonnettes droites, parallèles, divisées par de nombreuses cloisons fines, transversales et on peut seulement s'étonner que cette structure, si apparente sur le fragment, soit tellement voilée sur la masse, dont le morceau a été détaché. Or, quelle conclusion à tirer de ceci ? Si la météorite de Knyahinya n'est composée, comme le veut M. Hahn, que d'organismes manifestes, le chondre de Vouillé doit être un organisme aussi et l'enstatite du Schillerfels de Baste ne peut être autre chose qu'un organisme ; mais si ce dernier est une enstatite, à la formation de laquelle la vie organique n'a pris aucune part, les chondres de Vouillé et de Knyahinya doivent aussi être exclus du règne organique.

Une discussion assez animée sur cette ressemblance minéralogique des chondres avec des conformations semblables dans des roches terrestres s'est établie entre M. Rzehak, d'un côté, et MM. Hahn et Weinland, de l'autre, dans le journal Ausland, de 1881. M. Rzehak avait critiqué (n$^{\circ}$ 20) l'ouvrage de M. Hahn en appuyant sur le fait qu'on avait observé des chondres ayant plusieurs centres de rayonnement et que la structure \og favositoïde \fg n'était qu'une préformation ultérieure de la structure en colonnettes des autres chondres. --- \og J'ai pu observer, disait-il, cette structure sur un feldspath (?) dont les contours rectilignes sont assez reconnaissables ; les lamelles ou colonnettes ne sont pas disposées en rayonnant, mais sont particulièrement intéressantes par le fait que des inclusions vitreuses globulaires, disposées suivant l'axe longitudinal, s'y font remarquer; les petites inclusions on été prises sans doute pour des perforations analogues à celles qui se rencontrent sur les parois des tubes des prétendus \emph{Favosites}. Quelquefois ces gouttelettes isolées se confondent de manière à simuler un canal dans l'axe de la colonnette. Les perforations apparentes des parois se trouvent aussi dans des endroits où manquent les cloisons divisant le tube corailliaire. Les cloisons manquent du reste souvent, et là où elles sont développées, elles se font reconnaître tout simplement pour des cassures transversales par leur irrégularité. \fg

MM. Weinland et Hahn ont riposté dans le n$^{\circ}$ 26 du même journal. Le premier affirme la nature animale de quelques-uns de ces organismes, qu'il décrira prochainement ; le second répète en grande partie ce qu'il a dit dans son ouvrage en attestant que les structures observées par lui sont des tubes ronds, consistant en \og substance formant les parois et une masse de remplissage. \fg Nous avons démontré, je pense, que des cristaux transparents, enveloppés par une substance opaque incrustante, offrent sous le microscope absolument cette apparence de tubes ronds, formés par une paroi opaque et une masse claire de remplissage. M. Hahn insiste beaucoup sur les perforations et les canaux centraux des tubes. Ce qui nous confond à notre tour, c'est la manière dont M. Hahn détruit ses propres assertions. Les soi-disant perforations, analogues ou identiques avec les canaux bourgeonnants des \emph{Favosites}, qu'il nous présente (Tab. 9 et Tab. 15) dans ses photographies, sont des taches noires, alignées sur la partie incolore, sur la matière de remplissage des prétendus tubes. --- Or, comment un trou percé à travers la gaine opaque du tube et pénétrant dans l'intérieur de ce tube rempli de substance transparente, comment un trou pareil peut-il se présenter comme un trou noir ? Et, si la masse transparente est le remplissage seulement d'un tube, comment cette masse peut-elle présenter dans son axe un canal central d'apparence foncée ? Il devrait donc y avoir deux tubes emboîtés l'un dans l'autre --- chose absurde en elle-même, qui n'a pas besoin d'être réfutée.

Nous trouvons dans cette réplique de M. Hahn un aveu très-caractéristique. \og L'enstatite de la météorite de Bishopswille, qui est du minéral enstatite pur, est tellement conforme à l'enstatite du Texas, figuré Tab. 1, fig. 2 (donc une enstatite météorique avec une enstatite terrestre), que les deux images ne peuvent être distinguées. Si l'enstatite météorique a la même structure comme l'enstatite terrestre lorsqu'elle se présente seulement comme minéral, il s'ensuit, lorsque le minéral météorique présente des structures entièrement différentes, que ces structures doivent avoir une autre cause, qui n'est pas inhérente au minéral. \fg

\og Tout est vie! Feutre d'organismes, rien d'autre, \fg s'écriait M. Hahn dans son ouvrage, et ici, dans sa réplique, il nous tombe du ciel, à la lettre, une enstatite minérale dans la météorite de Bishopswille!

Nous avons montré les transitions qui mènent à travers les \og cent structures \fg de l'enstatite soi-disant organique de M. Hahn. Depuis les formes de l'enstatite et de la bronzite, telles qu'elles se trouvent ordinairement dans les roches, des modifications graduelles mènent vers la structure colonnaire simple, ramifiée, rayonnante et divisée par des cloisons. \og L'enstatite et la bronzite, dit Rosenbusch (Physiographie microscopique des minéraux importants en pétrographie. Stuttgardt, 1873, p. 253), ne se voient guère à l'état de cristaux, mais sous forme de grains cristallins à contours irréguliers, qui laissent reconnaître une striation très serrée... La surface des coupes fortement inclinées sur le plan de clivage principal ne se montre pas de la même manière finement striée, mais âpre en forme de gradins. Des plans de séparation transverse et des cassures ne sont pas rares. \fg

C'est dans cet état de grains cristallins que se montrent les groupes, formés artificiellement par M. Daubrée, au moyen de la fusion du péridot avec du fer doux et les groupes de cristaux plus gros dans les météorites de Knyahinya et de Vouillé ; l'accident arrivé à la plaque mince de Schillerfels de la Baste nous a montré que le fin striage, dont parle Rosenbusch, est dû à une structure colonnaire, exactement semblable à celle des chondres Knyahinya et de Vouillé, dont aussi une partie a été dissociée par le choc du polissage. Les givres d'enstatite, produits par M. Stanislas Meunier nous ont montré que les formes des chondres ramifiés et articulés n'ont rien d'organique, puisque ces mêmes formes peuvent être produites par la formation des enstatites dans une atmosphère chauffée au rouge ; ces givres nous ont montré, en outre, que ces formes rayonnées, branchues, articulées ne sont qu'un pas de plus dans la tendance de ces minéraux, de se subdiviser à l'infini, et cette tendance est confirmée par les enstatites artificielles produites par M. Daubrée au moyen de la fusion de la lherzolite avec le fer doux. On pourrait ajouter, en effet, encore quelques centaines de structures à celles décrites ou plutôt photographiées par M. Hahn, en dessinant et en décrivant un à un les bâtonnets et les rayonnements fins visibles dans cette singulière production artificielle.

Pour se rendre compte des apparences si diverses sous lesquelles se présentent les chondres dans les coupes fines, on n'a qu'à considérer le groupement des aigrettes composant ces globules, autour d'un point excentrique, duquel elles rayonnent vers la périphérie de l'ovoïde. La coupe frise-t-elle seulement la surface, où les dernières piécettes des colonnettes ramifiées se pressent les unes contre les autres, --- on aura l'aspect d'un corps finement réticulé. Des coupes dirigées convenablement, comme celle de la météorite de Vouillé, que j'ai figurée, montrent pour cette raison une zone corticale transparente, finement réticulée. De forts grossissements laissent voir, dans cette zone périphérique, comme l'a dit déjà M. Gümbel, les contours de ces cristaux infiniment petits, qui ont encore conservé leurs angles obtus et réagissent fortement sous le polariscope. --- Si, au contraire, la coupe passe par le point de départ des colonnettes, conformément au plan des aigrettes rayonnantes, on verra un soi-disant corail ou crinoïde à bras ramifiés. --- La coupe passe-t-elle par un plan presque tangentiel au point de départ des aigrettes ? L'image d'un corail à branches bourgeonnantes et rayonnantes dans tous les sens se présentera infailliblement. --- Enfin, si la coupe passe par le point de départ même, on verra un groupe de gros cristaux ou des pièces cristalloïdes, en arrangement irrégulier, séparées par des interstices, lesquels sont remplis par une matière incrustante plus ou moins opaque. Des coupes plus ou moins obliques présenteront, suivant la direction différente du plan de la coupe, toutes les figures intermédiaires imaginables.

Qu'on me permette une comparaison triviale, mais cependant assez juste. Qu'on prenne un balai formé de branches ramifiées de bouleau, tel qu'on en fait usage dans beaucoup de pays et qu'on le traite d'une manière analogue à celle dont on traite les chondres en faisant des tranches minces. En coupant ce balai suivant différents plans longitudinaux, transverses, obliques, près de l'extrémité des branches à la périphérie ou près de l'emmanchement, on pourra obtenir des images, grossières il est vrai, mais imitant assez bien les Uranias, les coraux, les crinoïdes, dont on veut nous gratifier aujourd'hui.

Cette manière de voir se confirme encore par l'aspect du givre d'enstatite artificiel, tel qu'il sort du tube dans lequel il s'est formé. M. Stanislas Meunier a eu la bonté de me communiquer quelques-uns de ces flocons globuliformes, conservés dans une petite éprouvette. Ce sont des petites sphères très légères, très cassantes, hérissées de petites pointes et grandes de un à deux millimètres environ. Elles présentent sous la loupe une structure rayonnante. Examinées sous le microscope, après les avoir montées dans une cellule à parois assez épaisses, pour que le couvre-objet ne les touche ni ne les écrase, on voit les aigrettes ramifiées parties dans tous les sens comme dans les chondres et en montant ou descendant le foyer, on peut se procurer des coupes optiques lesquelles, sauf les interstices beaucoup plus grands entre les colonnettes, ressemblent assez aux coupes réelles des chondres.

Je n'ai pas besoin d'insister plus longuement sur ces observations. Elles prouvent, je pense, d'une manière péremptoire, que toutes les conclusions si étranges, auxquelles est arrivé M. Hahn, reposent sur des appréciations erronées, engendrées par des recherches incomplètes, faites sans contrôle, sans comparaison sérieuse avec des organismes réels, vivants ou fossiles et sans critique reposant sur l'emploi de méthodes différentes d'exploration. Toute cette prétendue création animale contenue dans les chondres des météorites doit donc être reléguée dans le domaine des erreurs involontaires, dont pullule l'histoire de la science.

Dans un second mémoire nous prouverons, mon collègue M. Denis Monnier et moi, par des expériences sans réplique, que l'on peut produire à volonté les formes organiques essentielles, telles que tubes, tubes à cloisons, cellules à canaux poriques, etc., en employant, pour cette fabrication de formes déterminées, rien que des substances absolument inorganiques, telles que sels métalliques, silicates, etc. Nous prouverons que la forme de ces produits est constante en ce sens, que certains réactifs produisent des tubes cylindriques, creux à l'intérieur, remplis de dépôts granuleux au centre du tube, avec parois membraneuses et cloisons transversales, tandis que d'autres réactifs produisent des cellules à parois, à canaux poriques arrondis, droits ou flexueux, rayonnant depuis le centre et s'ouvrant, sur la périphérie de la cellule, avec des orifices béants. Nous démontrerons par ces expériences qu'il n'existe aucun caractère général de forme, qui puisse être invoqué comme distinctif entre les produits organiques et inorganiques, et nous exposerons en détail, avec appui par des figures photographiées, les résultats auxquels nous sommes arrivés et dont nous avons donné connaissance à la Section des sciences de l'Institut national genevois dans sa séance du 13 Décembre 1881.

Je crois, en résumé, que le présent mémoire justifie les propositions suivantes :
\begin{enumerate}
\item Les prétendus organismes des météorites (chondrites) n'existent pas, et ce que l'on a décrit et figuré comme tels est engendré par des conformations cristallines absolument inorganiques ;
\item Aucun de ces prétendus organismes n'a la structure microscopique propre aux organismes vrais, auxquels on les a associés ; en particulier, les prétendus spongiaires ne montrent pas la structure des vrais spongiaires vivants ou fossiles, ni les soi-disant coraux celle des polypiers des Anthozoaires, ni les Crinoïdes imaginaires celle des Crinoïdes reconnus ;
\item Les structures observées sont ou dues à la présence d'une matière incrustante opaque ou le résultat d'illusions d'optique, causées par une méthode incomplète de recherches microscopiques ;
\item L'observation microscopique de plaques minces, obtenues par le polissage, poussé seulement jusqu'à une certaine limite, est insuffisante pour se rendre compte de la structure des chondres. Cette recherche doit être contrôlée par des observations faites sur des plaques réduites à la dernière limite, ainsi que par l'examen des chondres dissociés au moyen des acides et de la potasse caustique ;
\item Les observations de contrôle démontrent avec évidence que tous les chondres sont composés de pièces transparentes, cristallines, groupées de différentes manières, mais le plus souvent en colonnettes ou en aigrettes ramifiées et rayonnantes depuis un centre. Les interstices, les cassures et les séparations de ces pièces groupées sont remplis par une matière incrustante opaque, résistant en grande partie à l'action des acides, simulant des cloisons \og ayant corps \fg et autres particularités, attribuées à une structure organique ;
\item Les aigrettes composant les chondres sont identiques, quant à leur forme et au groupement des pièces cristallines qui les composent, avec les aigrettes d'enstatite artificielle obtenues par M. Stanislas Meunier dans ses expériences ; comme aussi les boulettes de givre, formées dans ces mêmes expériences, sont analogues, quant à l'ensemble du groupement, aux chondres ramifiés et articulés ;
\item Certains chondres à fin striage laissent voir un groupement colonnaire rectiligne, identique avec la structure des enstatites terrestres (Schillerfels de Baste dans le Harz).
\item La plupart des chondres contiennent une quantité de groupes de cristaux plus gros, identiques, quant à leur groupement, leur forme et leur structure, avec les groupes de cristaux d'enstatites obtenus par M. Daubrée par la fusion du péridot avec du fer doux ;
\item En dehors des masses pulvérulentes, des substances métalliques et de la matière incrustante non cristallisée, les météorites ordinaires ne sont composés que d'éléments cristallins, réunis en chondres, comme le démontre la désagrégation par l'usure ou par les acides.
\end{enumerate}
\clearpage
\rhead{Explication des Figures}
\begin{figure}[b]
\includegraphics[width=\textwidth,height=\textheight,keepaspectratio]{Fig1.png}
\caption{Coupe d'une branche de corail vivant (\emph{Seriatopora caliendrum} Ehrbg.) : a, Canal longitudinal de la branche principale. – b, c, d, Cellules coupées à différentes hauteurs. – e, Canal bourgeonnant. Sur les pointes, on voit les deux dispositions des cristaux, en plumes et en mailles. Gross. 100 diamètres. Fig. 1a. – Groupement des cristaux en mailles aux bords. Gr. 500.}
\centering
\end{figure}
\clearpage
\begin{figure}[b]
\includegraphics[width=\textwidth,height=\textheight,keepaspectratio]{Fig2.png}
\caption{Crinoïde Hahnien de la météorite de Vouillé. Gr. 50. On voit le point de départ des colonnettes ramifiées, articulées, rayonnantes, souvent un peu courbées et la zone corticale, présentant un dessin à mailles très fines et serrées. Des grains et esquilles de fer météorique sont dispersés dans la masse.}
\centering
\end{figure}
\clearpage
\begin{figure}[b]
\includegraphics[width=\textwidth,height=\textheight,keepaspectratio]{Fig3.png}
\caption{\emph{Pentacrinus europaeus}. Gr. 50. Pour montrer la structure réticulée propre à toutes les pièces du squelette, composant la tige, le calice et les bras naissants.}
\centering
\end{figure}
\clearpage
\begin{figure}[b]
\includegraphics[width=\textwidth,height=\textheight,keepaspectratio]{Fig4.png}
\caption{Cristaux simulant des filaments d'algues dans une diorite de la rivière de Leith, près d'Edimbourg. Gr. 180. Ces cristaux sont des prismes hexaèdres ; l'ombre des côtes produit dans quelques-uns des traits longitudinaux simulant des canaux. Dans d'autres, on voit des véritables canaux médians avec libelle d'air ou des bulles vides rangées suivant l'axe.}
\centering
\end{figure}
\clearpage
\begin{figure}[b]
\centering
\includegraphics[keepaspectratio]{Fig5.png}
\caption{Cristal obtenu de la météorite de Knyahinya par le traitement au moyen des acides. Gr. 300. On voit des cassures remplies par la substance incrustante raréfiée et sur l'un des bouts des piécettes articulées apposées en disposition colonnaire.}
\end{figure}
\clearpage
\begin{figure}[b]
\centering
\includegraphics[keepaspectratio]{Fig6.png}
\caption{Esquille de Knyahinya, traitée à la potasse, ayant une disposition colonnaire et articulée. Gr. 300. Nicols croisés.}
\end{figure}
\clearpage
\begin{figure}[b]
\centering
\includegraphics[keepaspectratio]{Fig7.png}
\caption{Cristal disloqué de Knyahinya. Gr. 300. La matière incrustante pénètre partout et remplit les petits creux de la surface.}
\end{figure}
\clearpage
\begin{figure}[b]
\includegraphics[width=\textwidth,height=\textheight,keepaspectratio]{Fig8.png}
\caption{Tab. 2. – Groupe de gros cristaux dans une coupe mince de la météorite de Vouillé. Gr. 180. On y voit quelques gros amas de fer météorique. La matière incrustante opaque remplit tous les interstices.}
\centering
\end{figure}
\clearpage
\begin{figure}[b]
\includegraphics[width=\textwidth,height=\textheight,keepaspectratio]{Fig9.png}
\caption{Tab. 2. – Coupe très mince de l'enstatite artificielle produite par M. Daubrée par la fusion du péridot avec du fer. On y voit une grande lacune presque circulaire obtusement anguleuse, laissée par un cristal enlevé. Le fer remplit les interstices. Gr. 180.}
\centering
\end{figure}
\clearpage
\begin{figure}[b]
\includegraphics[width=\textwidth,height=\textheight,keepaspectratio]{Fig10.png}
\caption{Tab. 2. – Coupe plus épaisse de la même enstatite artificielle. Gr. 180.}
\centering
\end{figure}
\clearpage
\begin{figure}[b]
\includegraphics[width=\textwidth,height=\textheight,keepaspectratio]{Fig11.png}
\caption{Tab. 3. – Chondre transparent de la météorite de Vouillé, montrant une structure finement striée. Un morceau disloqué c laisse voir la structure colonnaire. – a, Remplissage tubiforme d'une cassure, isolé. Gr. 100. – b, Extrémité du tube formé par la matière incrustante, montrant la lumière du canal. Gr. 500.}
\centering
\end{figure}
\clearpage
\rhead{Explication des Figures. Fragments de chondres de Knyahinya, traités par les acides. Gr. 300.}
\begin{figure}[b]
\centering
\includegraphics[keepaspectratio]{Fig12.png}
\caption{Cristaux plus gros, sur lesquels sont posés d'autres cristaux plus petits.}
\end{figure}
\clearpage
\begin{figure}[b]
\centering
\includegraphics[keepaspectratio]{Fig13.png}
\caption{Portion d'un corail Hahnien ; disposition colonnaire articulée.}
\end{figure}
\clearpage
\begin{figure}[b]
\centering
\includegraphics[keepaspectratio]{Fig14.png}
\caption{Bras d'un Crinoïde Hahnien ; disposition articulée et ramifiée.}
\end{figure}
\clearpage
\begin{figure}[b]
\centering
\includegraphics[keepaspectratio]{Fig15.png}
\caption{Tab. 2. – Disposition colonnaire et parallèle de cristaux rongés et marqués par des incrustations de la matière opaque.}
\end{figure}
\clearpage
\rhead{Explication des Figures}
\begin{figure}[b]
\includegraphics[width=\textwidth,height=\textheight,keepaspectratio]{Fig16.png}
\caption{Coupe mince de l'enstatite produite par M. Daubrée au moyen de la fusion de la lherzolite avec du fer. Fibres rayonnantes dans des champs circonscrits par des bâtonnets cristallins. Gr. 50.}
\centering
\end{figure}
\clearpage
\begin{figure}[b]
\centering
\includegraphics[keepaspectratio]{Fig17.png}
\caption{Fig. 17. et 18. – Deux de ces bâtonnets. Gr. 500. Sur l'un de ces bâtonnets, on voit des figures ressemblant à des mamelons à pores ou à des cicatrices de feuille ; sur l'autre, des pièces ressemblant à des crampons.}
\end{figure}
\begin{figure}[b]
\centering
\includegraphics[keepaspectratio]{Fig18.png}
\caption{Fig. 17. et 18. – Deux de ces bâtonnets. Gr. 500. Sur l'un de ces bâtonnets, on voit des figures ressemblant à des mamelons à pores ou à des cicatrices de feuille ; sur l'autre, des pièces ressemblant à des crampons.}
\end{figure}
\clearpage
\begin{figure}[b]
\centering
\includegraphics[keepaspectratio]{Fig19.png}
\caption{Fig. 19, 20, 21. Tab. 3. – Groupes de l'enstatite artificielle en givre, produite par M. St. Meunier. Gr. 500. Fig. 19, Articulation latérale des colonnettes. Bras de Crinoïde Hahnien. Fig. 20, Corail Hahnien ; cicatrice d'un canal bourgeonnant. Fig. 21, Groupement stellaire.}
\end{figure}
\clearpage
\begin{figure}[b]
\centering
\includegraphics[keepaspectratio]{Fig20.png}
\caption{Fig. 19, 20, 21. Tab. 3. – Groupes de l'enstatite artificielle en givre, produite par M. St. Meunier. Gr. 500. Fig. 19, Articulation latérale des colonnettes. Bras de Crinoïde Hahnien. Fig. 20, Corail Hahnien ; cicatrice d'un canal bourgeonnant. Fig. 21, Groupement stellaire.}
\end{figure}
\clearpage
\begin{figure}[b]
\centering
\includegraphics[keepaspectratio]{Fig21.png}
\caption{Fig. 19, 20, 21. Tab. 3. – Groupes de l'enstatite artificielle en givre, produite par M. St. Meunier. Gr. 500. Fig. 19, Articulation latérale des colonnettes. Bras de Crinoïde Hahnien. Fig. 20, Corail Hahnien ; cicatrice d'un canal bourgeonnant. Fig. 21, Groupement stellaire.}
\end{figure}
\clearpage
\begin{figure}[b]
\centering
\includegraphics[keepaspectratio]{Fig22.png}
\caption{Fragment d'enstatite tiré d'une coupe mince du « Schillerfels » de Baste dans le Harz. Gr. 300. Disposition colonnaire et articulée rendue visible par le choc du polissage, comme dans le fragment du chondre transparent de Vouillé, Fig. 11.}
\end{figure}
\clearpage
\begin{figure}[b]
\includegraphics[width=\textwidth,height=\textheight,keepaspectratio]{Fig23.png}
\caption{Groupe de cristaux dans une coupe de la météorite de Knyahinya ressemblant au produit artificiel de la fusion de la lherzolite avec le fer doux. Gr. 50.}
\centering
\end{figure}
\clearpage
\end{document}
