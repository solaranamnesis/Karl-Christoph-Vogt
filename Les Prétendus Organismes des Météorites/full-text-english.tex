\documentclass[a4paper, 12pt, oneside]{article}
\usepackage[utf8]{inputenc}
\usepackage{fouriernc}
\usepackage{booktabs}
\setlength{\emergencystretch}{15pt}
\usepackage{fancyhdr}
\usepackage{graphicx}
\graphicspath{ {./} }
\usepackage{float}
\usepackage{microtype}
\begin{document}
\begin{titlepage} % Suppresses headers and footers on the title page
	\centering % Centre everything on the title page
	\scshape % Use small caps for all text on the title page

	%------------------------------------------------
	%	Title
	%------------------------------------------------
	
	\rule{\textwidth}{1.6pt}\vspace*{-\baselineskip}\vspace*{2pt} % Thick horizontal rule
	\rule{\textwidth}{0.4pt} % Thin horizontal rule
	
	\vspace{1.5\baselineskip} % Whitespace above the title
	
	{\LARGE The Alleged Organisms}

	\vspace{1.2\baselineskip}

	{\LARGE of the Meteorites}

	\vspace{1\baselineskip} % Whitespace above the title

	\rule{\textwidth}{0.4pt}\vspace*{-\baselineskip}\vspace{3.2pt} % Thin horizontal rule
	\rule{\textwidth}{1.6pt} % Thick horizontal rule
	
	\vspace{1\baselineskip} % Whitespace after the title block
	
	%------------------------------------------------
	%	Subtitle
	%------------------------------------------------
	
	{By Carl Vogt\\ President of the Geneva National Institute} % Subtitle or further description
	
	\vspace*{1\baselineskip} % Whitespace under the subtitle
	
	%------------------------------------------------
	%	Editor(s)
	%------------------------------------------------
    \vspace*{\fill}

	{\small\scshape }

    {Central Geneva Printing House, Rue du Rhone, 52\\ Geneva 1882} % Subtitle or further description
    
    Internet Archive Online Edition  % Publication year
	
	{Attribution NonCommercial ShareAlike 4.0 International} % Publisher
\end{titlepage}
\setlength{\parskip}{1mm plus1mm minus1mm}
\clearpage
\frenchspacing
\listoffigures
\clearpage
\section*{The Alleged Organisms of the Meteorites}
\pagestyle{fancy}
\fancyhf{}
\rhead{The Alleged Organisms of the Meteorites}
\cfoot{\thepage}
\paragraph{}
Toward the end of 1880 there appeared in Germany a work in \emph{quarto}, which could not fail to arouse one's attention. It was entitled: \emph{The Meteorite (Chondrite) and its Organisms}, presented and described by Dr. Otto Hahn. Thirty-two tables with a hundred and forty-two photographed pictures. Tübingen, 1880. Laupp, publisher.

I summarize, by literally translating the author's words, the main results he lays out.

``The chondrites, an olivine-feldspar (enstatite) rock, consist of an animal world, they are not part of a sedimentary rock layer nor a conglomerate, but a felt of animals, a fabric whose meshes were all once living beings and life of the lowest kind, the beginnings of creation.'' (p. 3)

``As one examines the tables of this work, it immediately becomes clear that these are not mineral forms, but organic ones; that we have before us the images of life, images of life of the lowest order, a creation which in greater part finds some of its closest relatives here on Earth — regarding the corals and crinoids, this is determined with absolute certainty; however, the sponges have only a little similarity with those forms of the terrestrial genera.'' (p. 7)

``Anyone who even superficially surveys the forms will soon find that they provide an actual historical development. All the transitions from the sponge to the coral, from the coral to the crinoid are present, so that it becomes doubtful if one should assign new species to these transitions.'' (p. 3)

``The investigations up till now, in the whole field, with the exception of [Karl Wilhelm von] Gümbel's work in the Munich Academy, are of little use, both regarding the accuracy of their observations and even more the interpretations based upon those observations, i.e. on unproven hypotheses and weak assumptions — not suitable for scientific findings as such.'' (p. 7)

Hahn therefore believes that he has provided ``incontestable proof that the chondrites are the remains of animals that lived in water, that the entire meteorite is formed only of the remains of sponges, corals, and crinoids, metamorphosed by petrification into enstatite. It is true that there are small rare places where there are real crystals, but these crystals are so disposed that they cannot have any influence on the value of my actual proofs.'' (p. 21)

``When I said that the chondrite is nothing but an animal-fabric, an animal-felt, a qualification must be sustained.''

``There are, however, very small, sharply outlined places in this animal-bone stone, which could probably (but not necessarily) be from the beginning rocks. These are slate-blue, uncommon inclusions with 3-5 mm. diameters \emph{lacking definite recurring forms}, which include distinct crystals in their grayish mass, these are on average either squares or rhombuses, at other times it includes hexagons. This mineral can be either augite or olivine. It does not knock on the fact, \emph{that in the olivine strata formations exist and that these are the cause of the construction of the planet bodies, their self-constructed development and complex composition}.''

``In all cases, however, the ratio in the chondritic rock is the opposite as that in the sedimentary layers of Earth. In the latter the organisms are stored and the rock strata enclose them; in the first there are only organisms and the rock strata are masses of such.'' (p. 35)

``These forms are not mineral forms,'' says Mr. Hahn with absolute certainty. But knowing very well that similar such assertions are rarely accepted by the scientific world, without palpable proofs, he seeks to give them by grouping them into two categories, stating positive proofs and negative proofs.

``In order to prove that a plant or animal organism is present, I consider it necessary to prove:
\begin{enumerate}
\item a \emph{determinate} form, (I do not know how to translate the term used several times by Mr. Hahn, ``geschlossene Form''; the literal translation, ``closed form'' has no meaning)
\item a form \emph{that repeats},
\item one which repeats itself in degrees of development,
\item structure, namely cells or vessels,
\item resemblance to known forms.''
\end{enumerate}
\paragraph{}
``If these requirements are valid, it remains only to decide whether plant or animal? Now ask yourself, do my forms fulfill these requirements?'' (p. 20)

Needless to say, the response is affirmative.

Of all these conditions laid out by Mr. Hahn, there are obviously only two that can decide the question from certain points of view; the others are equally applicable to minerals. Crystals have determinate forms, which always repeat themselves and always better than organic forms, in the various phases of development. Until now we were quite convinced that it was a privilege of the great number of organic types to change form during the different phases of their development; apart from spawn, germs and seeds, and larval forms, for example, which are often very different from those of definitive animals, and the cotyledons of plants, which often do not resemble definitive leaves in any way, crystal forms are extremely stable. Mr. Hahn maintains that we are in error. Granted --- only, in this case, the first three conditions he poses do not say anything about the distinction between organic and inorganic forms.

The structure that Mr. Hahn invokes as the fourth condition is without a doubt preponderant, provided however that the animal or plant parts subject to the petrification persist. Hahn poses as a condition of this structure the presence of cells or vessels. That's very well --- but I'd like to know, what cells and vessels could remain when a sponge undergoes fossilization? It is known that the tissues of these animals are composed of extremely delicate cells, which dissolve with great ease, and all that can be found in a petrified sponge consists of calcareous or siliceous mineral spicules, in which neither cells nor vessels can be seen! And if the presence of cells or vessels is an indispensable feature, what is to become of fossil corals, where one definitely sees only lacuna surrounded by crystals?

All that remains of the five conditions posed by Mr. Hahn is that last, the similarity with known forms. But here again the greatest uncertainties can take place. Are these the exterior forms? Are these the details of the structure of the forms? We mention, in another essay, a host of cases where prominent mineral conformations, produced artificially or by nature, mimic in a perfect manner organic forms and we have, on the other hand, in the corals, in the intracellular crystals of plants, in the otoliths of animals, a quantity of examples of mineral forms produced by organisms.

\emph{We must therefore address the forms and special comparative structures}. We must push the comparison to the most minor details in appearance when we want to prove that this object which we have before our eyes is a sponge, a coral, or a crinoid. We leave aside, for the moment, the so-called negative proofs by which the author wants to demonstrate to us that the objects displayed by him cannot be mineral forms --- they are of about the same value as his positive proofs. We address the special forms, which by their resemblance to known forms and by their identical structure have to provide incontestable proof that the chondrites are formed by organisms related to those of the Earth.

We sequentially give a review on these alleged organisms by enumerating, with the same terms of the work, the aspects that the author attributes to different organisms which he believes to have recognized.
\clearpage
``A. --- Sponges''

``1. \emph{Urania}.''

``Round, lobed bodies with an obvious place of growth.'' --- ``Folds caused by contraction.'' --- ``Circumvented spiral.'' --- ``The structure consists of an outer membrane enclosing lamellar layers.'' --- ``Blue color.'' --- ``Obvious stratification. One might attempt to place the form among the corals if the outer form did not exist.'' --- ``We believe to see the indication of a mouth opening.''

``After all this, I think \emph{Urania} is a sessile sponge that contracts in a spiral form, absorbing and expelling water like our living sponges.'' (pp. 23 and 24)

These are the structural details that must convert us to the opinion of Mr. Hahn. The \emph{Urania} fill, according to him, three twentieths of the total mass of the stony meteorites; they are displayed on six tables comprising thirty-one figures.

In a previous work by the same author, \emph{Primordial Cell}, \emph{Urania guilielmi}, dedicated to Emperor William [I], was represented as a plant with rounded leaves, wrapped up in its young age and equipped with capsules carrying spores. In passing through the present work, \emph{Urania} lost these capsules with their spores; it became a sponge. It is true that we are not allowed to learn of the point causing this change of place, so considerable, to occur; the author does not say a word about the reasons which obliged him to change his opinion. What aspects of this supposed organism were lost or gained to be transported from one kingdom to another? An inopportune question that the author does not answer.

``2. Sponges with spicules.'' (Table 7)

``I place Figure 1 among the \emph{Astrospongia}. The spicules are regularly crossed. Figure 6 is an irregular spicule framework with a weakly indicated cavity.'' (p. 24)

The supposed spicules resemble, mistakenly, linear crystals dispersed in a homogeneous mass, such as seen in the initial coming of lava. In a few places we see a slightly marked tendency towards a stellar arrangement, very common in crystals, unusual in the spicules of sponges, whose forms are known to be quite different.

The author could not have compared his \emph{Urania} and astro-sponges with living and fossil sponges; he could not have studied the structure of the latter, for it would be impossible with this acquired knowledge to convince connoisseurs, as the notions and figures given by him have little rapport with the microscopic structure and nature of sponges. Mr. Hahn must be entirely ignorant of the fine research of Mr. Zittel on fossil sponges. (\emph{Memoirs of the Munich Academy}, Vol. 12 and 13; \emph{Handbook of Paleontology}, Vol. 1), because with this knowledge he could not have presented to us, as obvious sponges, cross sections with rounded contours surrounded by a membrane [sic!] possessing a structure or fine striations or lamella, equally unknown in living and fossil sponges. We know, it is true, of a quantity of fossil sponges where the layout of the channels displays a radiating arrangement, already visible to the naked eye or the magnifying glass (\emph{Aulocopium}, the Ventriculitides); but in all these sponges the spicules, being either loose or forming a very regular reticulated skeletal mesh, are always recognizable in the magnifications used by Mr. Hahn. In the alleged sponges of the meteorites there does not exist any trace whatsoever of this characteristic skeleton. We also know from Mr. Zittel's research the conditions under which, by the pseudomorphosis of siliceous sponges in limestone and that of calcareous sponges in silica, the inner structure may be entirely or partly lost; but in these cases the indication of the channels equally disappears and there remain only amorphous masses without apparent structure, formerly called ``petrosponges'' but which have been entirely removed from this classification ever since Mr. Zittel made known their true primitive structure.

Conclusions: The alleged sponges of the meteorites have neither the form nor the structure of known sponges.

``B. --- Corals''

``Here we have such well-preserved terrestrial forms that not a doubt is left remaining.''

``Table 8 shows a sample image, Table 9 its channel structure: obvious bud channels that are tubular connections (for there are such). In addition, there is the curvature of the channels, which absolutely cannot be mistaken for a sheet breakage, plus there is the very clear tube openings and finally an equally clear growth site. The bud channels are 0.003 mm apart. Of course, everything you can ask for from a \emph{Favosites} structure.''

``In Table 11 any researcher will easily recognize the image of living coral forms, the more so as the cup shape (cavity) is indicated in Figure 1 above. The same object also shows the cross partitions of the tubes, which clearly emerge.'' (Unfortunately, I fail to see in this figure any indication of a cavity, tubes, or transverse partitions.)

In other figures: ``Obvious lamellar structure.''

In others: ``Tubular corals obvious. In the original, one can clearly distinguish: glassy like intermediate masses, black tube walls, yellow tubular filling material, occasionally the latter is also black. This form occurs a hundredfold in all the chondrites.'' (pp. 25 and 26)

Corals constitute, according to the author, one twentieth of the total mass.

By attentively studying the thirty figures of the so-called corals, distributed on nine tables, we can be convinced from the outset that all the figures representing entire specimens show absolutely the same general form as the \emph{Urania} --- a rounded form with well-developed contours, similar to that of an entire round or oval leaf. The only difference that exists between the alleged sponges and the alleged corals is in the appearance of divergent ridges which eccentrically set themselves out from a narrow point of departure and which seem thicker and better marked in the corals. It is as one sees in the general form of the chondrules --- most of the figures give us absolutely nothing more than what we have known for a long time from the authors occupied by the meteorites. We come across, it is true, a few rare figures showing radiant streaks from several points of departure. Mr. Gümbel has already mentioned this exceptional disposition that I have also noticed in many of my cuts; we see another, designated by the name ``chain coral,'' where on a clear rounded space there are present some obscure spots with washed-out and irregularly arranged contours. This figure resembles, as much and perhaps more, the skin of a speckled cat over that of a coral. But the author wants it to be a coral; may your will be done, my lord!

The structure stands out above all in the two figures photographed under high magnification, Table 9 and Table 15. On the first, one sees columns with straight fixed contours, occasionally a little curved; a few of these columns show a series of dark dots aligned in the center. These dots can be seen on a few columns of the fifteenth table, but this magnified figure at once gives the explanation of the phenomenon, which, according to Mr. Hahn, provides proof for the existence of an axial channel in the center of the columns. In fact, we see a small column chipped at nearly regular intervals on one of the sides and cracked transversely into several pieces, thus resembling a gear shaft. Fractures in the breaks are filled with a black encrusting material. Imagine the figure of a battered and worn bevel gear shaft, on its surface erosion has carried to the bottom of hollows a substance and we will have the image of a small column marked with points aligned along the axis, such as the figure of Mr. Hahn.

If it is already now astonishing, that among these numerous figures, compared sometimes to the \emph{Favosites} of the Silurian, at other times to crateriform, star or even chain corals, \emph{there is not one to be found that displays a general form different from the alleged Urania}, our astonishment increases even more if we compare the structures (not described, because Mr. Hahn does not give descriptions, but depicted) to those which we know of living corals or well-characterized fossils. Very reckless indeed, one who would like to find in the figures of Mr. Hahn something analogous to the figure that we give of a piece of a section of a branch of \emph{Syringopora caliendrum} (Ehrenberg), which has been obligingly borrowed from our colleague Mr. Th. Studer, professor in Bern, and which gives the ideal section of star corals, stony corals [Scleractinia], maze corals, \emph{Fungia}, \emph{Tubipora} and \emph{Favosites} in our possession because it summarizes, essentially, the modifications of structure that can be found among other corals. This section (Fig. 1) indeed shows a branch of coral cut longitudinally. The section traverses broad areas encompassed by a thicker skeleton and fine tips, faded down to the most complete transparency.

``The microscopic structure of stony coral [Scleractinia] skeletons,'' says Mr. Zittel (Palaeontology, p. 206), ``is very uniformly fibro-crystalline. The small fibers that outwardly radiate from the centers of crystallization form star-like patterns, similar to feathers.''

The skeleton of Anthozoan polyparies displays, as a matter of fact, a microscopic structure that, in the majority of cases, is plainly crystalline. A tube or a branch of coral is not simply a piece of solid limestone, pierced along its axis by a roundish central channel or divided by partitions, like Mr. Hahn presents; the branch is always composed of a multitude of tiny crystalline pieces, assembled in a specific order. In transverse cuttings of the channels or cells of the \emph{Favosites} and \emph{Tubipora}, we see the tops of these parts protruding inward; in longitudinal cuts, they seem arranged like the barbs of a feather. The bud of a channel (our figure displays one), even if it was one-tenth of a millimeter thick, will still show this composite structure for the simple reason that the skeleton is primarily comprised of crystalline spicules isolated from one another, which are brought together only later. These scattered spicules can be seen with ease in the cortical layer of the Gorgonacea and within the fleshy mass of Octocorallia. In the polypary's fan parts, in the feeding lamellae, in the septa frequently very fine, these crystalline pieces collect into stars, occasionally simulating through their forms osseous corpuscles or even exhibit a reticulated aspect, yet in which the small parts are just recognizable under a strong magnification. We provide a figure (Figure 1a) of this reticulated structure under a magnification of 500 diameters. This structure does not disappear at all, unless a petrifying crystallization has filled it entirely, even skeletal spaces; we may also observe about the thinnest sections, that they appear much better than the sections only a little bit thicker; it is seen, regarding the latter, in the ever so thin partitions of the \emph{Favosites}.

Yet, this structure so characteristic with its crystalline elements of multi-faceted form, but constant in every specie, is completely lacking in the alleged corals of Mr. Hahn, shots of chondrules. We posses before our eyes a thin section with chondrules, which represent this author's corals; the object is composed of rods or small solid columns, radiating from an eccentric center (attachment point for Mr. Hahn), occasionally dichotomized at very acute angles, separated from one another by an opaque encrusting mass, which has infiltrated the transverse fractures or superficial chips, thereby simulating a longitudinal series of pits and grooves.

There is therefore not a single similarity between the alleged corals of Mr. Hahn and genuine corals, such as we know them from the various formations in the most ancient strata of the Earth. There is not even a similitude with the external forms, because the tubiform cells of \emph{Favosites} are distinctly polygonal and pierced by holes on their wall, and the entire polyp is either loosely branched or very organized in a thick mass.

We arrive at the final class, representing, according to Mr. Hahn, most of the chondrules of the meteorites and that themselves make up, according to the author, sixteen-twentieths of the total mass. It is the class or even, if you will, the phylum of Echinoderms, represented by the crinoids. Studied with preference by our author, this type did not provide fewer than sixty-six figures. Here, we will undoubtedly come across a more ample yield of facts and observations. The structure of the crinoids is complicated; their forms are quite varied; study offers plenty of difficulties, on which the sagacity of the observer can be applied. Given the multitude of specimens found within the meteorite of Knyahinya alone, the bottom of the planetary sea, from which the aerolites originate, must have resembled a submarine crinoid forest, an occurrence known from the dredging of modern expeditions.

``C. --- Crinoids''

``They are found from the most simple form of an articulated arm to complete crinoids with stem (we have searched in vain for a stem in the figures), with calyx, main and auxiliary arms. The conservation is ordinarily good. The difficulty comes with the thousands of directions of the cut that always result in different images of the same object. The oviform remains, which were considered to be glass, are calyxes of crinoids.'' --- ``Arms broken by pressure from above.'' --- ``Crinoids with as many arms as one likes'' (Mit einer beliebigen Anzahl von Armen). --- ``Crinoid with five arms.'' --- ``Reticulated structure upon a few forms, which agrees with the structure of schreibersite in the meteoritic irons.'' --- ``Different uncertain forms; we are not sure if they are sponges, \emph{Urania} or corals.'' --- ``Reminds one of the genus \emph{Comatula}.''

I believe that I have omitted nothing in my report of the observations on the forms and structures. The rest must be guessed from the figures.

We admit that it is very meager. A few assertions without any proof.

As I already hinted in my talk about the facts of the sponges and corals, the author does not present any comparison, even superficial, with the structure of other living or fossil organisms belonging to the same class. Mr. Hahn contents himself with the most crude resemblance. As a matter of fact, the objects in the figures resemble crinoids like a leaf of the \emph{Sabal} or \emph{Chamaerops} resembles a fan. That is all.

We could speak at length if we wanted to get into an itemized critique of the numerous figures photographed by the author. So, for all the figures of Table 29, this is how they will be taken by all observers who have been occupied by research on thin sections of rocks: as assemblages of more or less acicular crystals, assembled in the highly common form of asterisks grouped around different centers, such as we are used to seeing, for example, in the actinoliths. The majority of the figures in the following plate will not contradict this diagnosis. The other figures, such as those of Tables 17 and 28, do not display any resemblance, neither remote nor rough, with a part or section of a crinoid; as for the other figures, that is to say (Table 19), cuts of large poorly defined crystals with worn out corners and traversed by channeled breaks in all directions, they are boldly granted to us as the panels of the calyx of a crinoid, whose arms resolve themselves immediately, without transition, into a mass of secondary rays.

We may apply to all these alleged crinoids the same remarks we have already made about the corals. \emph{All of them, as they are a whole, possess precisely the same form in rounded sheets, like the corals, like the Urania}. We could copy exactly the contours of the \emph{Urania} sponge and apply them to a coral, to a crinoid, without having the need for the slightest alteration. We present a figure of a Hahnian crinoid (Fig. 2), drawn from a distinct chamber in a thin section of the Vouillé meteorite, which Mr. Daubrée has permitted us to use with his habitual helpfulness. This figure is even more complete than any of the figures photographed in such large numbers by Mr. Hahn --- were we observe exactly the same rounded leaf form. However, admittedly, we are not in any way certain if our determination is right --- is it an \emph{Urania}, a coral, a crinoid? We willingly leave the choice to the reader --- what we are certain of, in any event, is that this is a section of a complete chondrule, within which are embedded fragments of meteoritic iron in places.

Surely, none of the figures produced by Mr. Hahn correspond with the exterior likeness of crinoids, as we know them. Does the general order of the body correspond better? One is permitted to be in doubt. Except in a single case, none of these meteoritic crinoids obey the general law, which establishes the number of five branches for animals of this class. Just a few rare cystoids present exceptions to this rule in that they have a number of reduced arms always not very developed, simple, without branching, so barely apparent that their existence was denied for a long time. With the crinoids of Knyahinya, on the contrary, what a plush growth of arms, branched to excess, in number as considerable as one wishes! The few genuine crinoid fossils with six arms (\emph{Hexacrinus}, \emph{Atocrinus}) are so rare, so similar to adjacent genera, that the majority of authors deem them as monstrosities. But they may not be compared in any way with those Briareus who fell to Earth and who were likely premature, for they came into overt rebellion against the law established for the terrestrial creations.

The general form leaves us with shortcomings, the order of the parts of the body eludes us --- we are thus required to secure the inner, microscopic structure of these beings, devoid of stems and calyxes, and supplied with an infinite number of arms overly branched, which, above all, are not arms and would have been very awkward, according to all appearances, for accommodating the organs necessary for life, that is, if they had been alive.

The microscopic structure of the calcareous parts of echinoderm skeletons is easy to identify. It is a consistent fact that all of these parts, whatever they are, plates, pieces of stems, arms, cirri, or pinnules, always possess a reticulated structure, with tight lattices more or less perforated, structure which manifests itself as early as the formation of the skeleton in the juveniles and maintains itself up into adult age. All these parts of the skeleton are built upon the same fundamental type, for they are formed through the meeting of sharp-edged constituent elements, primitively isolated from each other, but which are bound through their prominences. The lattice may be looser or tighter, but it is never lacking, even in the more solid parts of the skeleton.

As an example of this structure, I provide a figure of the \emph{Pentacrinus europaeus} (Fig. 3), the well-known larva of the comatulid, drawn according to nature and under low magnification. One observes this reticulated lattice structure on the stem, comprised of jointed cylinders, on the principal and axillary plates of the calyx, and even on barely developed arms. I need only to mention the descriptions and figures given by Mr. Carpenter (\emph{Embryogeny of the Antedon (Comatula)}) and those of the ever erstwhile Mr. Valentin (\emph{Monographies of the Echinoderms Living and Fossil} by Agassiz. Neuchâtel 1838-45. Echinus). Mr. Zittel outlines this structure very nicely in his \emph{Paleontology} (Vol. 1, pp. 311-315). This author mentions, while speaking about fossil crinoids: ``They almost always show an essentially crystalline conformation, due to the infiltration of calcareous spar, but rarely does it destroy the microscopic reticulated structure in a complete way. In contrast, this is lost when the limestone is replaced by silica.''

Yet, nothing, absolutely nothing of this structure shows up in the figures of Mr. Hahn. What he likes to refer to under the title of ``reticulated structure'' (Tab. 30, Fig. 6; Tab. 21, Fig. 5) does not in any way look like the lattice structure of echinoderm parts, but instead like super small crystals, cut obliquely and arranged in tiers. Mr. Hahn thinks he has found a ``remarkable'' resemblance with the schreibersite of meteoritic irons that might, with help from the imagination, morph into an organism. However, neither the arms of any of these alleged crinoids, nor, above all, the colossal plates making up the so-called calyx of one of these crinoids, figured in Table 19 and which are nothing else other than a crystal traversed by breaks filled in with an opaque substance, display any trace of the characteristic structure of crinoid skeletal parts.

I frankly confess that this absolute absence of comparative investigation regarding the identified animals, living or fossil, and this complete absence of the known properties of microscopic structure, such as can be found in types of highly organized skeletal parts like the echinoderms, inspired in me the foremost doubt about the validity of the conclusions that Mr. Hahn drew from his laborious observations.

It appears that one of Mr. Hahn's defenders, his friend Mr. Weinland, a zoologist, has completely abandoned the ``so-called crinoids'' of his friend ``since he is not able to follow the zoological determinations everywhere.'' (\emph{Das Ausland}, No. 26, 1881)

I was talking of my doubts. They were compounded when I discovered, permit me to mention it, the flippancy with which Mr. Hahn moved his organisms, not only from one class, but even from one organic kingdom to another. An object, which appeared to him as a coral at the moment when he was arranging his plates, became, during the writing of the text, a crinoid or sponge, as if there were not an abyss between those different types, as if their structure were not, as we have demonstrated, fundamentally different. The \emph{Urania}, a plant close to the \emph{Florideae}, which possess reproductive organs drawn and described in a previous publication (\emph{Primordial Cell}), with all of a single stroke have lost their organs and suddenly become sponges. If, in his response to Mr. Rzehak's critiques (\emph{Das Ausland}, No. 20), Mr. Weinland excuses his friend by saying ``that at the beginning of our century most proficient pundits still took sponges for plants,'' then it seems to us that this excuse is worst than the error, because a contemporary author should not revert to the mistakes committed eighty years ago! Another author would have sensed the necessity, vis-à-vis a scientific audience, to lay out the reasons that led him to modify his assessment, whether these reasons consisted of newly discovered details of the structure, of comparative studies performed on algae and sponges, etc. Here, nothing of the like, \emph{sic volo, sic jubeo, stat pro ratione voluntas}!

I am wrong. Mr. Hahn formulates these transpositions, in one of the most unusual chapters that has been written in our time, such that we do not know what to admire most: the complete ignorance of the author with the laws of evolution or the audacity with which he states his views --- in terms worthy of the oracle of Delphi. In effect, our author demonstrates ``the unitary type of all the meteoritic organisms.'' Sponges, corals, crinoids are of a unified type! The forms develop one from another. I quote verbatim: ``It is certain that \emph{Urania} is the simplest form. But, this form is the starting point for the others.''

``The semicircular flap subdivides into layers, the layers into tubes, the tubes themselves are cross partitioned. The arms maintain their form, reuniting through a channel. A calyx forms between the arms and the stalk's attachment point and the simplest crinoid is there!'' Really, it is seriously as simple as that!

There is, however, an element of truth within that singular statement. All the organisms of Mr. Hahn proceed in effect from a similar type, however it is far from being organic. I will return to this subject, demonstrating that the term ``organic structure,'' which Mr. Hahn and his friends have truly abused through usage, is a term entirely meaningless when employed in general and applied to all the forms without exception and that it can only be employed by applying it to a determined and known object. One can say: such a structure is identical to this one from the sponges, from the corals, from the crinoids, consequently it is organic: one may not say: such an object has an organic or inorganic structure, because from one aspect the bodies created by the organisms, like the polypiers of the corals, are not composed of anything but crystals and from another aspect absolutely inorganic bodies may lead to forms impossible to distinguish from organic formations.

And, I as have come to show, the alleged organisms of Mr. Hahn are not in any way the structure of the animals to which he connects them; so we may say that the positive proof is not provided.

With a lack of positive proofs, Mr. Hahn sought to accumulate a certain number of so-called negative proofs, which may be summarized in the following manner: the forms that I have described and displayed cannot originate from inorganic bodies, thus they are organic.

We are not going to follow in pursuit of Mr. Hahn in these generalizations which, as we have just said, are in themselves meaningless; we will investigate the details, by studying the facts provided by observation, in order to arrive afterwards at general conclusions.

Mr. Hahn examined nineteen meteorites. It is that of Knyahinya (June 9, 1866) that supplied the greater part of his material. His collection of 360 thin sections must be, if we are to believe Mr. Weinland, the most magnificent collection in the world. We will gladly trust him. Save a few exceptions, which give no new type from the rest, all the figures of the publication in question represent alleged organisms of Knyahinya. A sole fragment of that fall has provided this multitude of forms, which Mr. Hahn estimates at several hundred. It is with much delight, no doubt, that in a single stone so many forms can be found combined together. We otherwise terrestrial paleontologists are not so fortunate.

The analysis method, followed by Mr. Hahn and his friends, is still the same very well-known for a long time; thin sections are made and observed with a microscope.

``I deliberately made,'' says Mr. Hahn, ``the cuts in three thicknesses; not very translucent, in order to have the included bodies as complete as possible; very thin, in order to see the structure clearly; the majority of it in such a way that both views were obtained.''

``I add here a remark, which will be confirmed by everyone who has dealt with thin sections of petrifaction.''

``It is only in rare cases that the structure remains visible on sections perfectly transparent and consequently very thin. The observer with a microscope is in the supreme degree delighted by the beautiful forms and lines which one sees in the semi-transparent section. In joy, one will wish to do even better and expects, continuing to grind their section, to see a perfect image. But when one puts the section under the microscope for the second time, nothing is seen but an almost structureless area, with forms barely showing, uncertain in their contours, which no longer allow one to recognize under the microscope that which was seen a moment before under the magnifying glass. However, this phenomenon is in connection with the metamorphosis of rocks and the forms that are included in them. The matter is moreover well-known and does not need more special details.'' (pp. 16 and 17)

I confess that my experience comes to the contrary conclusion. On the semi-transparent sections I only see confusing things and it is on very thin and very transparent sections that I see the details of the structure. I will revisit this subject in the remaining part.

In my investigations, enterprises with the aim of convincing myself of the existence of organisms in the meteorites, I necessarily had to apply myself to the chondrites and especially the chondrules themselves, which form the greatest portion of them. For Mr. Hahn the chondrites are, as we have said, a ``felt of organisms'' and crystals constitute rare exceptions. Mr. Weinland does not go so far. ``The various chondrites,'' he said, ``are very unequal in their organic conformations; some of which are composed of two thirds or more of them.'' And the third third of the mass? I suspect that the two friends will agree on this third, organic for the one, obviously inorganic for the other. It is a detail of appreciation, no doubt; but since it applies to the very objects prepared by Mr. Hahn and that Mr. Weinland has at his disposition, it is important. What happens to Mr. Hahn's negative proofs in the face of this third, according to which the forms of this third are not allowed to be inorganic?

Consequently, it was necessary to address the chondrules. While going through the publications, I saw with astonishment, that despite the opinion of Mr. Hahn, mentioned above, the structure of these bodies had already been fully identified by Gustave Rose, who provided them their name (\emph{On the Constitution of the Meteorites}, 1862), by Mr. Daubrée (\emph{Comptes Rendus}, 1866), by Mr. Tschermak (via his numerous communications with the Academy of Vienna), and by so many others; that Mr. Gümbel had made a comprehensive summary of this knowledge base (\emph{Academy of Munich Bulletin}, 1878), incidentally cited with praise by Mr. Hahn, and that Messrs. Makowski and Tschermak had finally completed these details by way of the meteorite of Tieschitz (\emph{Mémoires of the Academy of Vienna}, 1878). The figures by Mr. Gümbel, although very accurate, are in effect insufficient, being drawn under a magnification far too weak, while those provided by Messrs. Makowski and Tschermak show the exterior forms and the radiating structure of the chondrules, as well as the details of the inclusions and encrustations. I give here the description made by Mr. Gümbel in order to avoid restating the results of matters which are well-known.

``All the chondrites are without doubt rock debris, composed of small or large mineral splinters, from the well-known chondrules, almost always perfectly preserved, but often also as broken pieces and finally the metallic grains, meteoric iron, chrome or sulfur. All these fragments stay together, but are not bound by any intermediate substance --- one does not find amorphous, glassy, or lava substances.'' (Mr. Tschermak has, however, found these glassy substances in the Orvinio meteorite (\emph{Mémoires of the Academy of Vienna}, Vol. 20, 1870), and the question can be raised, if the encrusted substance of the columns, of which we will talk about, is not found in a state of fusion or half-fusion, which appears all the more likely in that it often has a blistered aspect and that it forms inclusions between the crystals. This substance gets deeply into the thinnest interstices, so that it can be thought that it comes solely from the fusion crust.) ``It is only in the fusion crust and in the black encrustations similar to the fusion crust and which penetrate into the gaps where we encounter a glassy amorphous substance, but which was generated later during the fall of the meteorite through the atmosphere. The larger granules that are difficult to melt are still usually embedded in this fusion crust without being melted. The mineral splinters display no signs of wear or rolling; they are sharp and pointed angles. The surface of the chondrules is never smooth, as it should be, if these globules were the result of rolling wear; on the contrary, it is uneven, hilly, rough as the surface of a mulberry or cut into crystalloid facets. Many of these chondrules are elongated, with some tapering in a specific direction, as happens with hail. One often encounters pieces which apparently must be regarded as parts of chondrules that have been shattered or torn. Exceptionally, chondrules are seen joined together like twins; more often one sees some on which or in which there are pieces of meteoric iron. Judging from many thin sections, the chondrules are diversely composed. Most often one finds a fibrous structure radiating eccentrically, so that from a point situated in the thinner part and far from the center radiate beams towards the periphery. The cuts directed along the most diverse planes consistently allow one to identify in the radiant substance an arrangement in the columns, needles, leaves or lamellas; it can be concluded that the chondrules are in effect formed by fibrous columns. In correspondence with this point of view, one sees in certain cuts, directed at right angles to the longitudinal fibers, areas irregularly angular and excessively small, as if the whole were composed of small polyhedral granules. Sometimes the chondrules also present an appearance as if they were composed of several systems radiating in different directions. It seems that the center of radiation was changed during its formation, which in certain cuts produces a structure of confused appearance. The fibrous structure becomes obscure towards the place of the periphery where the junction point of the radiating beam is found; here it becomes replaced by a granular agglomeration structure. In none of the many cut chondrules, though they were whole, could I observe that the beams extended all the way to the edge as if their point of meeting was situated outside the globule. The elegantly articulated transverse columns do not, in most cases, extend in the same way throughout the length of the beam; they become more pointed, branch out and terminate to make room for others, so that the cross sections present various designs with reticulated meshes. The columns are composed, as has already been said, of a lighter core and a darker envelope; the first is more or less attackable by acids, while the envelope is more resistant.'' (Based on my observations, the columns resist the action of boiling aqua regia while a part of the substance serving as an envelope is dissolved by hydrochloric acid alone.) ``The enveloping encrustations that as a rule only extend over a small part of the globules and appear to be composed of meteoric iron are very remarkable. The same unilateral encrustations, visible as curved streaks in an arc are also found in the interior of the chondrules and provide strong evidence against the supposition of a genesis of the chondrules through wear of some material. The entire arrangement of the radiating structure of the chondrules speaks moreover in a decisive manner against this supposition. But not all chondrules are eccentrically radiating --- many, especially the smaller ones, show a finely granulated structure, as if they were composed of a powdery mass kneaded into a ball. But even in this case the unilateral conformation of the globules is indicated by a more considerable eccentric compression of the powdery particles.'' (Gümbel l. c. p. [\emph{On the Stone Meteorites Found in Bavaria}] p. 58)

I wanted this description in its entirety because it corresponds reasonably, except for the points indicated, to my own observations and because it only imparts facts observed without any preconceived opinion and without any other more or less hypothetical explanation. Mr. Gümbel, a consummate mineralogist and geologist, started out with the study of a few meteorites fallen in Bavaria in order to construct generalities which find easy application everywhere.

I should quote here a strange fact. Mr. Gümbel also studied the carbonaceous meteorites of Bokkeveld and Kaba. ``I was hoping,'' he says (p. 71), ``that by means of thin sections I could perhaps discover within the carbonaceous mass a trace of organic structure. This mass displays, in the rare areas where it becomes rendered transparent, the membranous or finely granular structure that one encounters elsewhere on similar substances...'' ``I was not able to discover any indication of organic structure...'' He repeats, while talking of the Kaba meteorite: ``Also, this carbonaceous meteorite, treated with the method indicated (treatment with potassium chlorate and then with nitric acid), displays no trace of organic structure. Perhaps it will be accomplished eventually by employing the same procedure on larger masses or on other carbonaceous meteorites, the proof of the existence of ogranic beings on celestial bodies outside the Earth.'' (L. c. p. [\emph{On the Stone Meteorites Found in Bavaria}] p. 72)

In his ardor to find partisans, Mr. Hahn cited this phrase in the following manner: ``Mr. Gümbel ends with a description of the Kaba meteorite: ``Perhaps, however, it is still be possible to prove the existence of organic beings on celestial bodies outside of the Earth.'' I hope,'' adds Mr. Hahn, ``that I have succeeded!''

Isn't it strange that Mr. Hahn mentions nothing about the restriction, profoundly wise besides, that Mr. Gümbel places by basing his hopes uniquely on the carbonaceous meteorites?

Now I come to my observations.

In addition to a collection of several hundred fine sections of various rocks formed over a long time, the material at my disposition was lent to me in the most obliging manner by Messrs. de Hochstetter and Brezina (a beautiful entire specimen of Knyahinya), by Mr. Daubrée (artificial peridot and enstatite formed by melting; meteorites from Vouillé and Knyahinya), by Mr. de Marignac (a dozen chondrites of diverse origins), and by Mr. Stanislas Meunier (artificial enstatite glazed). --- Not having the intention to provide descriptions of these different meteorites, I will limit myself to that of Knyahinya and secondarily to that of Vouillé, which will furnish sufficient material for the purpose that I propose.

The first question that I have to raise is this: Is the method of research, followed exclusively by Mr. Hahn and his friends, exempt from possible errors?

Negative answer. In effect, the observable structures on living and fossil organisms are preserved even in the thinnest cuts and become quite noticeable as the measure of the cut gets very sheer; --- in contrast, the structures observed by Mr. Hahn are only visible, regarding the majority of cases, as he says himself, on the semi-transparent cuts and disappear when further work is performed. It was therefore necessary to find out what is supporting this fundamental difference; it was necessary to search, furthermore, if it was not possible to control the results produced by microscopic observation of the thin sheets, by employing alternative methods of exploration.

Be sure to believe that I have not neglected the straightforward inspection of thin sections and that the premier instruments of Leitz, Seibert and Krafft, Verick, and Zeiss have served me in their entire capacity. I would not have mentioned this detail, absolutely insignificant, for everyone nowadays has a good microscope, if it had not been endorsed in a quite distinctive manner within a popular article the excellence of the instrument with which Mr. Hahn makes his observations.

It was not necessary to go far into the examination of the cuts made along the plane of radiation in order to realize that the chondrules were composed, as Gümbel mentions, of small crystalloid columns, often simple as well as ramified, the branches departing, in the latter case, under very acute angles and then gradually diminishing in thickness from the departure point towards the periphery. In the majority of cases, these small columns are perfectly straight, in the others they are slightly curved, Mr. Hahn returns, on a number of occasions in his book, to his response to Mr. Rzehak (\emph{Das Ausland}, No. 26, 1881. p. 506) regarding the axiom that curved lines may not be found in the mineral kingdom, I provide, in another essay, the figures of a few groups and groups of curved crystals, similar to fronds of certain algae and which may be detected within lava and other crystalline rocks.

These small radiating columns, ramified or not, more or less dense, always display opaque encrustations, visible in the finest cuts and persisting to a large extent despite the action of acids. This encrusting and strongly adherent material fills in all the interstices of the small columns and penetrates the very frequent and often orderly transverse breaks of the small columns in a manner that mimics partition walls. These partition walls are often spaced in a manner so regular that one believes to see, upon considering a single small column, the filaments of algae. One also observes that the opaque encrusting substance is not everywhere of equal thickness; where it appears less opaque one sees roughness, small cavities, even deeper hollows that penetrate into the perfectly clear substance of the small columns, and which are filled by the opaque substance. The transparent substance of the small columns is nearly always rough, almost gnawed away, scarred by thousands of diverse smashes and yet always these cavities and guilloches of encrusted material.

Messrs. Weinland and Hahn are quite insistent, both on the occasional orderly arrangement of these apparent partition walls, and on their nature as partition walls. They are not breaks, they are partition walls; a break forms a simple line, it is ``an optical phenomenon''; here, they are ``bodily partition walls.'' I confess that I do not understand the difference between a break, whose two faces are slightly separated and whose gap is filled by an opaque material, and a bodily partition wall. In order to demonstrate that one comes across breaks more or less regularly distanced in crystals which simulate the filaments of algae, I give the figure of similar crystals detected in a thin section of diorite originating from the Leith River, near Edinburgh (Fig. 4). In the majority of cases the edges of these breaks correspond so exactly that one sees only a single line; in the others, more uncommon, one observes two parallel lines; the space is then filled by a clear and limpid vitreous substance. When the infilling substance is slightly opaque, one sees a bodily partition wall with a measurable thickness. I will supply the evidence further on, made through the observation of disaggregated cuts and an analysis of the pieces resulting from the action of acids, that such an effect is the real explanation of the partition walls ``being bodily.''

A second particularity upon which the designers of the chondrites insist is to rely on the fact that the small columns are truly round tubes, formed by an opaque wall and surrounding a clear substance, a filling of olivine or enstatite. According to them, the opaque encrusting substance would thus be the original skeleton of the animal, whereas the clear substance of the small columns would form the mold for the cavities, previously filled by the soft and shredded substance of the animal.

We pose that in fact any transparent body, whether it is a dodecahedron or an elongated prism with rectilinear facets, \emph{will appear rounded under the microscope due to the transmitted light, where it is surrounded by a more opaque substance}. It is an elementary phenomenon and which is completely accounted for by the disposition of the enveloping substance, which permits a greater amount of light to pass through the middle than at the edges, where it shows more considerable thickness. Shadows gradually decreasing towards a center or line, and gradually increasing towards the edge, gives us the impression of a rounded bulge with curved surfaces. This occurs all the more readily when the facets of the edges come together under blunt angles. Yet, just as massive enstatites display angles so dull that they seem round, likewise the elongated prisms of the enstatites look rounded and completely circular when they are surrounded by a more opaque material like a sheath.

To these difficulties, inherent in the nature of these objects, is added another. Inside the majority of the chondrules, the little columns are so confined and thin that it becomes physically impossible to make a cut that has a depth of only a single small column. All the cuts, even the thinnest, consequently contain quite a few superimposed layers of small columns. One can easily imagine that these superimposed bodies, transparent, although encrusted by an opaque material, and whose edges do not correspond in their layering, will necessarily produce fallacious and most of the time indecipherable shadow effects. An opaque interstice between two subjacent small columns, located within the median axis of the small column identified in the focus of the microscope lens, will impart to this small column an appearance like it was pierced by a longitudinal channel; partitions situated a little obliquely with respect to the axis of the small column, in between which are located the shadows produced by the subjacent partitions, will give to the small column the demeanor of being arrayed in a string. Even with the greatest volition in the world and despite the employment of superior instruments, all these difficulties cannot be vanquished; I would even state that the more one is trained in microscopic observation, the more one is persuaded that certitudes may not be acquired.

I have assayed polarized light, whose application should never be omitted when dealing with the analysis of minerals or rocks; the results were not conclusive enough to eliminate all the doubts. I will disclose these results later in their entirety.

Mr. Hahn sees the entire mass of the chondrites composed of organisms; Mr. Weinland sees only two-thirds of it; Mr. Rzehak (\emph{Das Ausland}, No. 26, 1881) does not see any at all, and examining everything, I had to align myself with the view of the latter observer.

It was therefore necessary to search for alternative methods and other comparisons.

Mr. Gümbel had already indicated the route. He always was concerned with verifying his observations on thin sections with microchemical operations. Referring to the Mauerkirchen meteorite (Nov. 20, 1768), he says (p. 19): ``After having treated the finely crushed (not pulverized) material with aqua regia and caustic potash, I saw that the metallic parts and the yellowish splinters (olivine) had disappeared and that the residue consisted of white or brownish morsels which were easily distinguished under the microscope. The brownish fragments are considerably cracked, at times filled with traces of opaque parallel striae; they are transparent and vividly colored with multicolored spots in polarized light. These are without doubt fragments from the augite mineral group. The white splinters, in contrast, are oftentimes entirely translucent, partially worn by the acids and show, in polarized light, matte colors disposed in patches which here or there indicate banded arrangements.'' And in talking about the Krahenberg meteorite (May 5, 1869) (p. 57): ``One views in a thin section treated with hydrochloric acid and still maintaining itself as an ensemble of numerous gaps, more or less wide, indicating the place of the dissolved material by the acid. By treating this section afterwards with a solution of caustic potash, it disaggregates into smaller pieces, granules and pulverized parts, among which the splinters stemming from the largest inclusions are distinguished by their greater consistency. It is most remarkable that in the pieces possessing a reticulated structure with striae, when they still hold together, the transparent striae are completely destroyed and just the opaque intermediary lamellae are conserved and present themselves like a skeleton. One may place this fact beyond doubt through the examination with polarized light.''

I followed this method. I treated cuts, I treated crushed chondrules, not pulverized, and as it was the Knyahinya meteorite which alone provided all the forms described by Mr. Hahn, I chose this meteorite for my experiments.

After having crushed the fragments into small pieces of approximately a millimeter in diameter by diameter, I consumed with boiling hydrochloric acid this shot, within which a lot of chondrules were still able to be seen almost intact with their spiky surfaces of tiny crystalline points. There is a moderately tumultuous outburst of sulphurated hydrogen, proof of the presence of pyrites; the dissolved iron colors the acid greenish yellow. I obtained a lightweight cloudy, almost gelatinous, precipitate that deposed very slowly, while also small brilliant and colorless particulates rapidly settled to the bottom and formed a white powder which collected the remaining grains entirely at bottom of the test tube.

Examined under a microscope, the light cloudy precipitate presents itself as an amorphous substance with extremely fine powdery granules. A few rather rare trichites, very dark and very fine, are encountered arranged into tufts in the middle of this mass. --- I attribute them to scraps of the fusion crust, parts of which were still attached to the analyzed fragment. The white, heavy, and powdery precipitate, in contrast, is totally composed of tiny crystalloid pieces, the description of which I will give later.

In addition to the pyrites and dissolved metals, hydrochloric acid then disjoined some end particles from the small columns by dissolving and decomposing an encrusting silicate probably rich in iron.

I attack with boiling aqua regia. A tumultuous release of nitrous acid; the acid is again colored yellow from iron. The aqua regia thus dissolved another ferric silicate more resistant to the attack. More cloudy precipitate; yet the powdery precipitate increased. The remnant grains are a dirty gray, spiky with asperities.

I examine this powdery precipitate under the microscope after having prepared it with balm.

I immediately see that on the majority of the scraps the opaque encrusting material has not completely disappeared. There must therefore be a substance, probably a silicate, containing iron or a different metal, which is insoluble in the strongest acids. However, the encrusted material has widely diminished and I find a quantity of small pieces that are entirely cleansed and transparent like the aqua, while the others display a greater opacity.

The isolated and transparent little pieces are prismatic, elongated, with terminal planes severed vertically in some instances; although more often than not they displayed at their extremities facets upon which were undoubtedly even smaller articulated pieces (Figs. 5 and 12-15). The sides of the prisms are rough; one can ordinarily see small impressions or quite deep cavities, within which still persists a little of the opaque material; in other cases, these planes are perfectly rectilinear, however, the angles under which they meet appear rounded. Facets similar to those of the ends are also displayed here and there on the sides of the prisms; they represent, without doubt, the articulation of the small lateral crystals located at bifurcations. Numerous transverse and longitudinal fissures are particularly remarkable upon the largest pieces (Fig. 5); very frequently these transverse fissures display an opening at the edge, while those in the interior of the piece appear like they ``have bodily partition walls''; one distinctively sees that these fissures are once more replete with the encrusting substance which binds together the fragments separated by the fissure. There is not a single clear and transparent morsel that does not display evidence of crystalline structure. The clear constitutive mass does not always appear entirely homogeneous; one sees cloudy designs, sometimes dots without definite form. All these small clear pieces, sometimes faintly colored yellow, considerably refract light; their contours are noticeably defined. Via crossed polarized light they exhibit the most beautiful colors organized into tiny irregular patches.

I reserve the description of the more composite morsels with a reticulated and fibrous structure, similar to those of the chondrules, for later.

I divide the rest of the material, treated successively by the two acids indicated, into two portions and I treat one of these portions with caustic potash, while I attack the other with concentrated sulfuric acid.

Concentrated sulfuric acid has no more action; caustic potash, in contrast, decomposes a portion even more. It forms the same almost gelatinous substance, which deposits very slowly, and the same powdery precipitate, as in the action of the acids employed in the first step. Lastly, there remains a grayish deposit of an indecomposable substance, which perhaps would have been reduced as well, if I had continued the cooking process even longer. The powdery precipitate is entirely composed of very fine crystalloid splinters, strongly refracting the light and glowing, under the crossed polars, with a faintly bluish white light. The gray deposit displays remnants of chondrules still held together. With the encrusting material being significantly diluted, these pieces gleam, under the crossed polars, with the most beautiful colors of the rainbow. I have drawn one in this state (Fig. 6). It is additional proof that the appearance of the colors of double refraction with the polariscope is impeded merely by the presence of the encrusting opaque material.

The small splinters and slender fragments, which can be reduced to a fine section by consuming them to the final limit, exhibit absolutely identical forms, as those produced by the action of acids, with the difference being, however, that the opaque parts of pyritic and magnetic iron are still encountered and that the encrusting material is conserved in its entirety. The greater part of these splinters are composed of evident, transparent crystals, frequently colored yellow, strongly refracting light and adorning themselves with beautiful colors through polarized light via crossed polars. These crystals are always fissured in all directions and often disaggregated, in such a manner that shows the fissures still filled with encrusting material. These can also be penetrated by small round holes more or less deep, which produce, according to the alignment or the distance of the focus, the impression of bubbles, holes or rings; one often sees attached to their extremities small prismatic or pointed crystals. I give a drawing of one of these crystals (Fig. 6). In addition to these crystals, there are also fragments of the fibrous masses with small columns, such as in the pieces disintegrated by the acids and to which I will return.

A principal point to take note of here is that, contrary to Mr. Hahn's assertion, the greater part of the Knyahinya meteorite is manifestly composed of crystals, refracting light and breaking polarized light. ``If (the chondrites) are crystals,'' says Hahn (p. 23), ``and if the lamellar fissuring was the cause of the structure, the mineral would necessarily have to refract light. Yet, in most of these inclusions no refraction is seen, nor even aggregate polarization! They can therefore neither be simple minerals nor crystals, even less can one explain the structure by lamellar fissures. This fact alone, the optical quality, should have led to the correct interpretation.''

I have already stated that Mr. Hahn considers the presence of crystals in meteorites as a very exceptional fact; in Knyahinya they must be completely deficient according to him, because he attributes the totality of the twentieths to organisms. Now, I maintain that this same Knyahinya meteorite is decomposed by the action of acids, potash and mechanical wear into evident crystals, refracting and decomposing light and that these crystals and crystal fragments form the greater mass of the splinters obtained by the two methods described. These crystals, when they are a little larger, united and glued together into groups by the encrusting material, are moreover easily noticed in the fine cuts, and I provide a figure of a similar group taken from the Vouillé meteorite (Fig. 8), where they are generally larger than those of Knyahinya. I have, however, encountered similar groups in several cuts of Mr. Hahn's preferred meteorite. In the sample from the Vienna Museum that I have detailed, I noticed, embedded in the middle of the mass, an oval chondrule, as big as a small pea, one centimeter long and seven millimeters wide, which was entirely composed of crystals traversed by slits slightly marked, but numerous, in which one could barely see the encrusting material. The chondrule was an almost white color, faintly greyish; its surface was rough and on part of this surface, which had been disengaged during the polishing of the surrounding gangue, one noticed small black dents, similar to chunks of slag. In polarized light, these crystals took on colors passing from a greenish, cadaverous, but very luminous tone, with brownish-yellow and reddish-brown tints.

These groups of cracked crystals, traversed by ``bodily partition walls,'' are incidentally present in meteorites with absolutely the same appearance as the artificial enstatites obtained by Mr. Daubrée through the fusion of peridot with 15\% soft iron and to which I am indebted for the helpfulness of my scholarly friend. In these artificial enstatites (Figs. 9 and 10) the excess iron played the same role as the encrusting material of the meteorites; filling in the interstices and fissures. Around large, almost globular crystals, which have often popped out from the wear leaving behind an obtuse angular void, are found clusters of agglomerated crystals. Yet, it is on this substance, hard enough to scratch glass, that I have observed a fact which will give, I think, the justification for the so diametrically opposed assertions of Mr. Hahn and myself. A very fine cut of this substance (Fig. 9), transparent and worn down to the final limit, displays under crossed polars the most beautiful yellow, blue and red colors, arranged in patches. One could not find a better substance to demonstrate the action of polarized light. From the same chunk I set about making the cuts a little thicker, translucent, or semi-transparent (Fig. 10); under crossed polars they show that there are, alongside a few strongly colored crystals, here and there some pale colored patches scarcely perceptible. It is exactly the same situation as in meteorites; in the fine slices of Knyahinya as well as Vouillé, which show images as presented by Mr. Hahn, and which are therefore worn just to the limit, I see but a few very small pale colored patches; on the cuts entirely worn down and on the detached fragments I see them widespread throughout and shining with all their brilliance. It is therefore evident that the superposition of the crystals equipped with their opaque encrustations impedes the perception of the colored rays generated by the polarized light.

Another example will confirm what I just said. A thin section of the Vouillé meteorite displays on one of its edges a chondrule measuring about two millimeters along its largest diameter and which I have represented in Figure 11. This cut would doubtless be the delight of an observer who believes in organisms. A central kernel, on which one sees nothing but a fine pointillage and a part rendered less clear by a thousand finely crossing lines, is surrounded by a more opaque border, from which depart radiating fine lines often presenting ramifications and which continue until at the edge, surrounded by a semi-circular belt of a completely black substance. The entirely transparent mass of this chondrule is furthermore traversed by a few radiating crevices equally filled with the black substance. On one place, the encrusting mass has completely detached itself and manifestly reveals the form of a cylindrical channel. I have designated this channel by the letter \emph{a} in Figure 11; by observing it under a very high magnification, the central edge (\emph{b} of the same figure) shows up well beneath the form of the orifice of a beveled channel. The fine radiating lines are so thin, that the strongest immersion lenses merely make them look like a line. It is thus a model \emph{Urania}, according to the figures supplied by Mr. Hahn. Yet, all this fibrous part, in which one sees no trace of transverse partitions, shows under crossed polars a radiating series of almost square patches, infinitely small, of alternating red and blue colors. Here, in this object, the encrusting material is so thin that it does not exert any influence on the absorption of polarized rays. A detached bit \emph{c} gives, as we will see later, the explanation of the colored drawing furnished by the polariscope.

I return to the Knyahinya meteorite treated with acids or worn until reduced to splinters. I said that in addition to the immediately recognizable crystals, which make up the major part of the fragments, one finds others which are less transparent and present this structure with ramified tubes, with transverse ``bodily'' partitions, that Mr. Hahn considers as decisive on behalf of the organic nature of chondrules. I give (Figs. 12-15) some drawings of several fragments; one (Fig. 12) represents a few pieces that are still quite large, on which are laid out a few small, nearly cylindrical or prismatic pieces with blunt angles; in two others (Figs. 13 and 14), everyone will easily recognize the structure of crinoids with ramified arms, such as represented by Mr. Hahn. Yet, wherever these minor fans still hold together, one sees the articulated pieces, separated by ``bodily'' partitions as if rounded by the slight lateral shadows; but where the available extremities of the small columns are present, they have acute edges and angles and are noticeably terminated. Examined with a polariscope, these fragments with organic structure show no reaction whatsoever as far as they form a body; however, the available extremities present the colors of double refractive substances.

The crystal composition is more manifest in other fragments with a lamellar structure, as I have featured in Figure 15. The interstices are replete with the encrusting material which enters the longitudinal and transverse fissures, the cavities and the pores of the clear pieces which seem to possess a pronounced lamellar structure, as if thin and long little planks were spliced together, often presenting their narrow side. These fragments as a whole have the same grayish color as the preceding ones; they exhibit no changes under the crossed polars; but their beveled or tiered extremities, which protrude from the encrusting material, shine with the most vivid colors.

Lastly, through the action of the acids there remain undecomposed globular chondrules, bristling with asperities, the size pin heads, which I have prepared with balm in a cell with one millimeter thick lining. The body of these chondrules is, needless to say, absolutely opaque under the microscope, while in direct light they present a light gray color. But the asperities, with which they are bristling, are in general transparent, carved into sharp angles and which through crossed polars appear as colored patches.

I am required to report these details, tedious perhaps, because they illuminate, it seems to me, the question in a positive manner. Thanks to the analysis through acids and attrition, I can now say, without fear of serious contradiction, that the Knyahinya pieces that I have examined and which are authentic samples, on which Mr. Hahn has identified ``hundreds of organic structures,'' \emph{only contain, besides the metallic splinters and the relatively pulverized parts, crystals, nothing but crystals}, variously developed in size, arranged, agglomerated, agglutinated in different ways. I then assert with certainty, that all the so-called organic structures are produced by crystals belonging to at least one specie, perhaps even several mineral species with single and double refraction.

One could raise the objection that the organisms were destroyed by the acids and that the crystals alone resisted. It is easy to rule out this objection for the following reasons: 1. The fragments with alleged organic structure and almost all the chondrules have resisted acids, only revealing their crystalline structure through the rarefaction of the encrusting substance; 2. The mechanical action of polishing down to the lowest limit has produced the same effects.

Arriving at this point in my research, I necessarily has to ask myself if analogous or identical forms to those of the chondrules could be demonstrated, either through artificial productions or within natural rocks. As for the former, I could only apply to Messrs. Daubrée and Stanislas Meunier, these two scholars being the only ones who have been occupied with experiments pertaining to the genesis of meteorites. I must thank these gentlemen who have placed at my disposal, with the greatest amiability, a considerable amount of material.

I have already given the description of the artificial enstatites produced by Mr. Daubrée through the fusion of peridots with soft iron. One can compare the drawings of a very fine cut of this product (Fig. 9) and that of another less thin (Fig. 10) with the reproduction (Fig. 8) of part of the Vouillé meteorite; it is impossible to find more comparable samples of the same mineral. Mr. Daubrée was therefore perfectly well-founded in saying that through his fusion process, already described in 1866, he had produced forms and aggregations similar to those found in the meteorites. Everything, form, interstices replete with an encrusting material, optical qualities, everything corresponds exactly. There is only a difference in the color; the crystals of the Vouillé meteorite are slightly tinted yellowish, while those of the artificial product are colorless. The yellow color is almost always produced by the infiltration of iron; by considering these patches, one arrives almost infallibly at a black splinter of meteoric iron which it surrounds like a halo. Similar groups of crystals are bestowed to us by Mr. Hahn (Tab. 21, Fig. 5; Tab. 22, Figs. 1 and 2) as parts of crinoids.

The products of the fusion of lherzolite with soft iron, obtained by Mr. Daubrée, provide guidance concerning a fact invoked with great force by Messrs. Hahn and Karsten (\emph{Nature}, 1881, No. 16). I have already remarked on the peculiarity of the microscopic forms of these products, of which I have given drawings (Figs. 16 to 18). Long clear rods, only ornamented in the most diverse fashion, circumscribed angular areas, occupied by a transparent substance, in which radiate brown fibers, extremely loose, which, under an immersion lens, pose as crossed lines or like rosaries. These fibers sometimes radiate from a center, sometimes they form feather figures; in most cases, they are straight, although we also remark that some show a slight curvature. Under the crossed polars, these areas with their fibers indicate no change, while the rods glow with the most vivid colors.

I give two figures of these rods, drawn under a magnification of 500 diameters (Figs. 17 and 18). I could have given fifty figures and more, because, examined in detail, each of these rods shows a different structure and frequently even the appearance of this structure changes quite a few times along the length of the rod. Here, there are fine crosshatchings; there, asperities which imparts on the stick an appearance of being bristling with hairs; in another spot you see pieces in the form of anchors or spikes placed on these rods or little raised protuberances in the form of stomata or cell pores. Mr. Hahn and his adherents always mention the ``lack of structure'' in minerals; I don't know of any organic parts, which present a more complicated structure than these artificially produced rods. Pores, openings on the small columns of chondrules, are equally invoked as obvious proof that lateral channels divide these locations from the main channels, which Mr. Hahn attributes to the corals, whereas Mr. Karsten sees them instead as filaments of algae of a Hystérophyme (\emph{Leptomitus} or \emph{Leptothrix}) (\emph{Nature}, 1881, No. 16, p. 184). ``It is, in any case,'' says Mr. Karsten, ``an organized body, because true crystals, which form in solutions that evaporate or condense are homogeneous and without structure.'' One need only examine my two drawings to see that crystals formed out of a cooling molten mass can present a most complicated structure, which is also manifested through the polariscope. The rod with pores, which in some places resemble leaf scars such as they protrude from the trunks of ferns and \emph{Sigillaria}, exhibits under the crossed polars a series of marked protuberances, in the middle of which is shaped a clear space like a hole. All these rods present, under the crossed polars, the most vivid colors.

If the crystalline forms, similar to those produced by Mr. Daubrée by means of molten lherzolite, are relatively rare in meteorites, it should not however be concluded that they are completely absent. I count, as a matter of fact, among the crossovers of the ramified chondritic structure with that of the lherzolite the following forms, all observed in the Knyahinya meteorite:
\begin{enumerate}
\item Chondrules with a combined structure, where in the middle of an almost pulverized mass very elongated articulated small columns are made out, are generally arranged like the spokes of a wheel. I observed one of the chondrules that presented on one of these halves six rays very regularly spaced, and on the other half there was a whole group of columnar crystals, partly branched, very tight and while all these rays departed from an eccentric center, although not too close to the edge, one saw near this center a crystalline rod of considerable length, which traversed the whole chondrule from one end to the other. On the side of the large chondrule there was a small one, formed of small columns extremely fine like lines, but interwoven with more considerable radiating small columns.
\item Forms, rather similar to feathers. From a central axis, on which is seen articulations, depart from one side completely transparent rays, like the axis itself, disposed at irregular intervals, yet all parallel and forming an angle of approximately 40 degrees with the axis. The intervals between these secondary axes are filled with crystalline fibers, arranged at right angles, like the barbs of a ramified feather. On the other side, these barbs depart from the axis itself and one sees some clearer spaces with no fixed direction. The barbs present themselves in the same manner as the fibrous forms of the artificial enstatite.
\item Finally, groups so exactly resembling the enstatites produced by the fusion of lherzolite, that they could be mistaken for each other (Fig. 23). Elongated prisms, fissured ad infinitum, arranged along several rows and joining together at obtuse angles, which circumscribe an almost round space and could well correspond to the facets of a cut dodecahedron, encompass an area traversed by large long crystals about whose nature one cannot have any doubt. In the spaces left behind between these crystals have developed fine fibers arranged in rays, crossing under several angles forming clusters. One only has to compare Figures 16 with 23 in order to be struck by the resemblance of the grouping of these fibers between the large crystals. The reaction under the crossed polars is exactly the same. It is therefore a complete identification between the artificial product and the natural product of this same Knyahinya meteorite, including the crystals which were to be strictly excluded. I must honestly say that Mr. Hahn photographed (Tab. 29, Fig. 2) an analogous grouping from Knyahinya, where a star with six rays, two of which are only indicated, while the four others are formed into groups of parallel crystals, is also surrounded by series of elongated crystals --- however, the interstices between the rays are, in the figure of Mr. Hahn, also filled in by larger crystals, whereas in the specimen one sees the fine crystalline fibers of lherzolite. For Mr. Hahn, it is a crinoid viewed from above; I do not think that the idea of a comparison with a crinoid, viewed from whatever side it may be, can come into sight of my drawing.
\end{enumerate}
\paragraph{}
Whatever the case may be, these facts clearly prove that even the strangest forms of enstatite engendered via the fusion of lherzolite are intimately connected with the constitution of certain meteorite chondrules; that there are gradual crossovers, between these different forms, under which the crystals have developed and grouped and that between the irregular assemblages of large crystals the columnar configuration and finally those dendritic or fibrillated, we cannot make a decision to adjudicate the differences.

However, the most complete resemblance with the articulated and ramified chondrules is offered by the artificial enstatite glaze, produced by Mr. Stanislas Meunier in the experiments which he set out in the records of proceedings (meeting of February 23, 1880) and on which he again called attention to in a recent communication with the Academy of Sciences (meeting of November 7, 1881).

Mr. Meunier insisted on the resemblance of this glaze to chondrules; Mr. Rzehak restated this resemblance; Mr. Hahn and his friends turned a deaf ear. Mr. Meunier was perhaps at fault for not supporting his assertions with figures; thanks to his helpfulness, I am able to make up for it. I give drawings made under a magnification of 500 diameters (Figs. 19-21) and I think that no one will be able to contest, I am not saying the resemblance, but the identity with the figures of fragments of chondrules treated with acids. They are the same small columns, the same arrangement, the same radiation departing from larger pieces to form ever more loose branches, the same apparent transverse partitions in both. In one of these figures one notices round scars, originating from broken branches, which part in a slightly different direction (Fig. 20, \emph{a}); on the others one sees a remarkable ramification, unilateral in some places (Fig. 19); lastly, a third figure (Fig. 21), shows the radiation from a central point, attachment point of the crinoid stalk for Mr. Hahn (Tab. 29, Fig. 4). Most of the branches are straight, but a few of them are manifestly curved, which, according to Mr. Hahn, is an absolute characteristic of organic conformation. Mr. Meunier may boast of having produced organisms through the assistance of mineral substances in a tube, heated to dark red! The transverse partitions, rigorously drawn with the \emph{camera lucida}, are as equidistant as they can be in a filament of algae or in an arm of a crinoid. All the pieces constituting these radiating aigrettes are solid, transparent, without any trace of interior structure, like the little pieces that come out of the aigrettes produced by the dissociation of the chondrules.

The glazes at my disposal were preparations, covered with a thin glass slide. But their distribution over varying degrees already shows that the small columns have to radiate in all directions and form clumps of flakes. Mr. Meunier informs me that, in effect, the glazes emerge in this form from the tube in which they were constituted; but these flakes are so delicate that the pressure of the coverslip is sufficient to flatten them completely. I recently received a small tube filled with glaze, just as it came out of the experiment, and I was able to convince myself that it contains small globular flakes, composed of aigrettes radiating in all directions.

I think that the demonstration is as complete as possible. The chondrules of Knyahinya, considered as animals by Mr. Hahn, only freed from as much as possible of the encrusting material, ended up being, as Mr. Meunier said, composed of exactly the same elements as the glaze of artificial enstatite.

It is therefore achieved in the debate that the most significant groups of crystals, joined by an encrusting material, which fills in the interstices and breaks in such a manner that produces bodily partition walls and which are encountered in profusion within the Knyahinya meteorite as in the other chondrites, were artificially reproduced by Mr. Daubrée, while the radiating, ramified, and articulated forms of the chondrites were procured by Mr. Meunier.

The second question that presented itself was this: Does one find forms within terrestrial rocks similar to those encountered in the meteorites?

If one thinks hard about the consequences of Mr. Meunier's experiments, one must say to oneself that the particular conditions under which the glaze of artificial enstatite was formed could scarcely be found except in the action of volcanoes. We have also compared the chondrules to globules which are found quite frequently in volcanic tuffs. However, the difference is great; the volcanic tuffs are generated by ash or lapilli cemented by water, and this ash itself results from the pulverization of lavas, that is to say of semi-molten rocks, composed of preexisting crystals and vitreous masses in varying proportions. Tuffs are therefore not formed directly in an atmosphere of superheated water vapor, but are the result of a reworking of substances that are melted and then pulverized. The formation conditions are therefore not the same.

Consequently, if there exist in the chondrules forms comparable to Mr. Meunier's glaze, and, if these forms have to be attributed to analogous causes, we cannot however doubt that there exist in the meteorites additional parts that appear to be own their origin to causes similar to those implemented by Mr. Daubrée, namely, the fusion or half-fusion in an effective reducing medium. The large Knyahinya chondrule, of which I spoke above, looks to me like an unambiguous product of crystallizations from a molten medium. The crystals that it is composed of are much too close together for one to admit another formation and several pulverized masses forming a lower part, which are embedded in the middle of the chondrule, also appear to me to advocate in favor of this opinion. The oftentimes bullous, puffy structure of the encrusting material, the thousand imprints, scratches and erosions of the crystals coated by this material, which has penetrated into the most available fissures and breaks, also speaks in favor of crystallization from an igneous fluid. The surface of a quantity of crystals entirely resembles through these various accidents that of crystals existing in lavas, and it is probable that these accidents have an analogous origin. I am thus not far from admitting that the immediate transition from the gaseous state to the crystalline state on the one side and the more or less accomplished fusion on the other, both played their role in the formation of chondrites and that, depending on the case, the one or the other of these causes may have engendered more dominant effects.

For me, what confirms this opinion is the study of that transparent and almost whole chondrule from the Vouillé meteorite, which I cited above as a model type \emph{Urania} of Mr. Hahn and of which I provided a drawing (Fig. 11). I already said that this chondrule with extremely fine flexible lines displays, under the crossed polars, a columnar or serial arrangement of small alternating blue and red patches. Yet, on one of the ends of the preparation, a few bits of this chondrule have been detached by the polishing. These morsels (Fig. 11, \emph{c}) have been shattered by mechanical action, their joints have become more apparent and they appear to be composed of a succession of fine articulated small columns, traversed by numerous partitions and running in parallel without ramifications. The structure indicated by the polarized light has consequently been made manifest through mechanical shock and weakening. In this piece too, the crossed polars produced a marvelous effect. I came across, on a section of the Knyahinya meteorite, a fragment with absolutely the same appearance.

These chunks of \emph{Urania} manifestly resemble, if I am not mistaken, a fragment of enstatite also detached by the action of polishing from a large mass that I encountered in a thin section from the famous ``Schillerfels'' of Baste in the Harz. The part from which this chunk has detached indicates no trace of a columnar structure; one sees thin bands of a greyish brown, with uncertain edges and a little flexing. The entire mass shows a striation just as fine as the chondrule of Vouillé. Neither the polariscope, nor the higher magnifications give anymore instruction about the structure of this mass. But the fragment detached by the shock of polishing (Fig. 22) exhibits the most evident columnar structure, entirely comparable to that of the fragment of Vouillé's chondrule and, let us say this right now, also to that of a chondrule fragment from Knyahinya, drawn (Fig. 15). They are the same straight, parallel small columns, divided by numerous fine transverse partitions, and one can only be surprised that this structure, so apparent on the fragment, is quite concealed on the mass, from which the chunk has been detached. Yet, what conclusion can be drawn from this? If the Knyahinya meteorite is composed, as Mr. Hahn desires it, of manifest organisms, the Vouillé chondrule must be an organism too and the Schillerfels of Baste enstatite cannot be anything other than an organism; but if the latter is an enstatite, in whose formation organic life took no part, the chondrules of Vouillé and Knyahinya should also be excluded from the organic kingdom.

A quite animated discussion of this mineralogical resemblance of the chondrules with comparable conformations in terrestrial rocks has arisen between Mr. Rzehak, on the one side, and Messrs. Hahn and Weinland, on the other, in the journal \emph{Das Ausland} of 1881, Mr. Rzehak had criticized (No. 20) Mr. Hahn's work by leaning on the fact that chondrules had been observed having multiple centers of radiation and that the ``Favositoid'' structure was only an ulterior pre-formation of the small column structure of the other chondrules. --- ``I could observe,'' he said, ``this structure on a feldspar (?) whose rectilinear contours were quite recognizable; the lamellae or small columns are not radially arranged, but are particularly interesting in their globular vitreous inclusions, arranged along the longitudinal axis, in my opinion; the small inclusions were undoubtedly taken for perforations analogous to those which are encountered on the walls of the tubes of the supposed \emph{Favosites}. Every so often these isolated droplets mislead in a manner which simulates a channel in the axis of the small column. The apparent perforations of the walls are also found in places where the partitions dividing the coral tube are missing. Incidentally, the often missing partitions where they are developed are recognized quite simply as transverse breaks by their irregularity.''

Messrs. Weinland and Hahn retaliate in No. 26 of the same journal. The first affirms the animal nature of some of these organisms, which he will soon describe; the second to a large extent repeats what he said in his work by attesting that the structures observed by him are round tubes, consisting of ``substance forming the walls and a filling mass.'' We have demonstrated, I think, that transparent crystals, enveloped by an opaque encrusting substance, presents under the microscope absolutely this appearance of round tubes, formed by an opaque wall and a clear filling mass. Mr. Hahn strongly emphasizes the perforations and central channels of the tubes. What confuses us in turn is the manner in which Mr. Hahn destroys his own assertions. The so-called perforations, analogous or identical with the budding channels of the \emph{Favosites}, which he presents to us (Tab. 9 and Tab. 15) in his photographs, are black stains, aligned with the colorless part, upon the filling material of the alleged tubes. --- Yet, how a hole bored through the opaque sheath of the tube and penetrating into the interior of this tube replete with a transparent substance, how can such a hole appear like a dark opening? And, if the transparent mass is solely filling the tube, how can this mass present in its axis a central channel of darkened appearance? So there ought to be two tubes nested inside each other --- something absurd in itself, which does not need to be refuted.

We find in this reply from Mr. Hahn a very characteristic admission. ``The enstatite of the Bishopville meteorite, which is pure enstatite mineral, is quite consistent with the enstatite from Texas, figured in Table 1, Figure 2 (thus a meteoritic enstatite alongside a terrestrial enstatite), that the two images cannot be distinguished. If the meteoritic enstatite has the same structure as the terrestrial enstatite where it occurs only as a mineral, it follows, when the meteoritic mineral exhibits entirely different structures, that these structures must have another cause, which is not inherent in the mineral.''

``All is life! A felt of organisms, nothing else,'' exclaimed Mr. Hahn in his work, and here, in his reply, we literally drop from the sky an enstatite mineral within the Bishopville meteorite!

We have demonstrated the transitions that lead to the ``hundreds of structures'' of Mr. Hahn's so-called organic enstatite. From the forms of enstatite and bronzite, as they are ordinarily found in rocks, gradual modifications lead to the simple columnar structure, ramified, radiating and divided into partitions. ``Enstatite and bronzite,'' said Rosenbusch (\emph{Microscopic Physiography of Important Minerals in Petrography}, Stuttgardt, 1873. p. 253), ``are hardly ever seen in the state of crystals, but in the form of crystalline grains with irregular contours, which allow one to recognize a very tight striation... The surface of the cuts strongly inclined on the principal cleavage plane does not show itself in the same finely striated manner, but harsh in the form of steps. Transverse separation planes and breaks are not rare.''

It is in this situation that the groups of crystalline grains, formed artificially by Mr. Daubrée by means of the fusion of peridot with soft iron, and the groups of larger crystals in the meteorites of Knyahinya and Vouillé, show up; the accident at the Schillerfels of Baste thin plates showed us that the fine striation, of which Rosenbusch speaks, is due to a columnar structure, exactly similar to those chondrules of Knyahinya and Vouillé, of which also a part has been dissociated by the shock of polishing. The enstatite glaze, produced by Mr. Stanislas Meunier showed us that the ramified and articulated forms of the chondrules do not have anything organic, since these same forms can be produced by the formation of enstatites in a red-hot atmosphere; these glazes have shown us, moreover, that these radiated, branched, and articulated forms are only one more step in/from the tendency of these minerals, to subdivide ad infinitum, and this tendency is confirmed by the artificial enstatites produced by Mr. Daubrée by means of the fusion of lherzolite with soft iron. One may add, indeed, a few hundred more structures to those described or rather photographed by Mr. Hahn, by drawing and describing one by one the rods and the fine radiations visible in this singular artificial production.

In order to account for the quite diverse appearances under which the chondrules show up in thin sections, we have only to consider the grouping of the aigrettes composing these globules, around an eccentric point, from which they radiate towards the periphery of the ovoid. The section is just the surface, where the rearmost small pieces of the ramified small columns press against each other --- we will obtain the aspect of a finely reticulated body. Properly directed cuts, like those of the Vouillé meteorite, which I have figured, show for this reason a transparent, finely reticulated cortical zone. High magnifications allow one to see, in this peripheral zone, as Mr. Gümbel has already said, the contours of these infinitely small crystals, which have still retained their obtuse angles and respond strongly under the polariscope. --- If, in contrast, the cut passes through the starting point of the columns, conforming to the plan of the radiating aigrettes, one will see a so-called coral or crinoid with ramified arms. --- Does the cut pass through an almost tangential plane at the departure point of the aigrettes? The image of a coral with budding branches and radiating in all directions will unfailingly present itself. --- Lastly, if the cut passes through the departure point itself, one will see a group of large crystals or crystalloid pieces, in an irregular arrangement, separated by interstices, which are replete with a more or less opaque encrusting material. More or less oblique cuts will present, pursuant to the different direction of the plane of the cut, every imaginable intermediate figure.

Permit me a trivial comparison, but nevertheless quite fair. Grab a broom formed of ramified birch branches, such as is used in many countries, and treat it in a manner analogous to that in which chondrules are treated by making thin tranches. By slicing this broom along different longitudinal, transverse, and oblique planes, near the extremity of the branches at the periphery or near the press-fitting, one will be able to obtain images, crude it is true, but imitating too well the \emph{Urania}, corals, and crinoids, of which they want to gratify us with at the present time.

This approach to viewing is further confirmed by the aspect of the artificial enstatite glaze, as it comes out of the tube in which it was formed. Mr. Stanislas Meunier was kind enough to impart to me some of these globular flakes, preserved in a small test tube. They are small, very light, very brittle spheres, bristling with little spikes and with size of approximately one to two millimeters. They present under the magnifying glass a radiant structure. Examined under the microscope, after having mounted them in a cell with walls thick enough so that the coverslip does not touch or crush them, one sees the ramified aigrettes parting in all directions as in the chondrules and raising or lowering the focus, optical sections can be obtained which, except for the much larger interstices between the small columns, rather resemble real sections of chondrules.

I need not belabor any longer on these observations. They prove, I think, in a peremptory manner, that all the quite strange conclusions, which Mr. Hahn arrived at, rest on erroneous assessments, engendered by incomplete research, made without controls, without serious comparison with real organisms, alive or fossil and without criticism relying on the employment of different methods of exploration. All this alleged animal creation contained in the chondrules of meteorites must therefore be relegated to the domain of involuntary errors, of which the history of science pullulates.

In a second dissertation we will prove, my colleague Mr. Denis Monnier and I, through experiments without replica, that one can freely produce the essential organic forms, such as tubes, tubes with partitions, cells with porous channels, etc., by employing, for this fabrication of determined forms, nothing but absolutely inorganic substances, such as metallic salts, silicates, etc... We will prove that the form of these products is constant in this sense, that certain reagents produce cylindrical tubes, hollow inside, replete with granular deposits in the center of the tube, with membranous and transverse partitions, while other reagents produce cells with walls, with rounded porous channels, straight or flexible, radiating from the center and opening, on the periphery of the cell, with gaping orifices. We will demonstrate by these experiments that there does not exist a general character of form, which can be invoked as distinctive between organic and inorganic products, and we will expound in detail, with support by photographed figures, the results to which we have arrived at and which we gave notice to the Science Section of the Geneva National Institute in its meeting on December 13, 1881.

I believe, in summary, that the present dissertation justifies the following propositions:
\begin{enumerate}
\item The alleged organisms of the meteorites (chondrites) do not exist, and what has been described and figured as such is engendered through absolutely inorganic crystalline conformations;
\item None of these alleged organisms have the microscopic structure proper to the true organisms with which they have been associated; in particular, the alleged sponges do not show the structure of true living or fossil sponges, nor the so-called corals that of polypiers of Anthozoa, nor the imaginary crinoids that of recognized crinoids;
\item The structures observed are either due to the presence of an opaque encrusting material or the result of optical illusions, caused by an incomplete method of microscopic research;
\item The microscopic observation of thin slides, obtained by polishing, pushed only to a certain limit, is insufficient to completely render the structure of chondrules. This research must be controlled by observations made on slides reduced to the final limit, as well as by the examination of chondrules dissociated by means of acids and caustic potash;
\item Controlled observations clearly demonstrate that all the chondrules are composed of transparent, crystalline pieces, grouped in different ways, but most often in small columns or in ramified aigrettes and radiating from a center. The interstices, breaks and separations of these grouped pieces are replete with an opaque encrusting material, largely resistant to the action of acids, simulating ``bodily'' partitions and other peculiarities attributed to an organic structure;
\item The aigrettes composing the chondrules are identical, as regards their form and the grouping of the crystalline pieces which compose them, with the artificial enstatite aigrettes obtained by Mr. Stanislas Meunier in his experiments; as also the pellets of glaze, formed in these same experiments, are analogous, regarding the whole grouping, to the ramified and articulated chondrules;
\item Certain chondrules with fine striations point to a rectilinear columnar grouping, identical with the structure of terrestrial enstatites (Schillerfels of Baste in the Harz);
\item The majority of chondrules contain a quantity of groups of larger crystals, identical, regarding their grouping, in their form and structure with the groups of enstatite crystals obtained by Mr. Daubrée by the fusion of peridot with soft iron;
\item Apart from the pulverized masses, metallic substances, and non-crystallized encrusting material, ordinary meteorites are composed only of crystalline elements, assembled in chondrules, as the disintegration through wear or acids demonstrates.
\end{enumerate}
\clearpage
\rhead{Explanation of the Figures}
\begin{figure}[b]
\includegraphics[width=\textwidth,height=\textheight,keepaspectratio]{Fig1.png}
\caption{Cross section of a real coral branch (\emph{Seriatopora caliendrum} Ehrenberg): \emph{a}, longitudinal channel of the main branch. --- \emph{b}, \emph{c}, \emph{d}, cells cut at different heights. --- \emph{e}, burgeoning channel. On the tips, we see two arrangements of crystals, in plumes and in meshes. Magnification 100 diameters. Figure 1a. --- Grouping of the crystals in meshes with edges. Magnification 500.}
\centering
\end{figure}
\clearpage
\begin{figure}[b]
\includegraphics[width=\textwidth,height=\textheight,keepaspectratio]{Fig2.png}
\caption{Hahnian crinoid from the Vouillé meteorite. Magnification 50. One sees the point of departure of the branched, articulated, radiating columns, often slightly curved and the cortical zone, displaying a very fine and close mesh design. Grains and splinters of meteoritic iron are dispersed throughout the mass.}
\centering
\end{figure}
\clearpage
\begin{figure}[b]
\includegraphics[width=\textwidth,height=\textheight,keepaspectratio]{Fig3.png}
\caption{\emph{Pentacrinus europaeus}. Magnification 50. In order to point out the reticular structure specific to all the pieces of the skeleton, composing the stem, the calyx, and the budding arms.}
\centering
\end{figure}
\clearpage
\begin{figure}[b]
\includegraphics[width=\textwidth,height=\textheight,keepaspectratio]{Fig4.png}
\caption{Crystals imitating algae filaments in a diorite of the Leith River near Edinburgh. Magnification 180. These crystals are hexahedral prisms; the shadow of the ribbing produces in some of them longitudinal features simulating channels. In others, we see genuine medial channels with pockets of air or empty bubbles arranged along the axis.}
\centering
\end{figure}
\clearpage
\begin{figure}[b]
\centering
\includegraphics[keepaspectratio]{Fig5.png}
\caption{A crystal obtained from the Knyahinya meteorite by treatment with acids. Magnification 300. We see fractures filled by a rarified encrusting substance and on one of the ends articulated pieces affixed in a columnar arrangement.}
\end{figure}
\clearpage
\begin{figure}[b]
\centering
\includegraphics[keepaspectratio]{Fig6.png}
\caption{Splinter from Knyahinya, treated with potash, having a columnar and articulated disposition. Magnification 300. Crossed polars.}
\end{figure}
\clearpage
\begin{figure}[b]
\centering
\includegraphics[keepaspectratio]{Fig7.png}
\caption{A crystal dislocated from Knyahinya. Magnification 300. The encrusting material penetrates everywhere and fills the small cavities of the surface.}
\end{figure}
\clearpage
\begin{figure}[b]
\includegraphics[width=\textwidth,height=\textheight,keepaspectratio]{Fig8.png}
\caption{Table 2 --- A group of large crystals in a thin section of the Vouillé meteorite. Magnification 180. There are some large clumps of meteoritic iron. The opaque encrusting material fills all the interstices.}
\centering
\end{figure}
\clearpage
\begin{figure}[b]
\includegraphics[width=\textwidth,height=\textheight,keepaspectratio]{Fig9.png}
\caption{Table 2 --- Very thin section of the artificial enstatite produced by Mr. Daubrée through the fusion of peridot with iron. There is a large, almost circular, obtusely angled gap left by a removed crystal. Iron fills the interstices. Magnification 180.}
\centering
\end{figure}
\clearpage
\begin{figure}[b]
\includegraphics[width=\textwidth,height=\textheight,keepaspectratio]{Fig10.png}
\caption{Table 2 --- Thicker cut of the same artificial enstatite. Magnification 180.}
\centering
\end{figure}
\clearpage
\begin{figure}[b]
\includegraphics[width=\textwidth,height=\textheight,keepaspectratio]{Fig11.png}
\caption{Table 3 --- Transparent chondrule from the Vouillé meteorite showing a finely striated structure. A dislocated piece \emph{c} displays a columnar structure. --- \emph{a}, A tubiform filling of a fracture, isolated. Magnification 100. --- \emph{b}, The extremity of a tube formed by the encrusting material, bringing to light the channel. Magnification 500.}
\centering
\end{figure}
\clearpage
\rhead{Fragments of chondrules from Knyahinya, treated with acids. Magnification 300.}
\begin{figure}[b]
\centering
\includegraphics[keepaspectratio]{Fig12.png}
\caption{Larger crystals, on which smaller crystals are laid out.}
\end{figure}
\clearpage
\begin{figure}[b]
\centering
\includegraphics[keepaspectratio]{Fig13.png}
\caption{Portion of an Hahnian coral; articulated columnar layout.}
\end{figure}
\clearpage
\begin{figure}[b]
\centering
\includegraphics[keepaspectratio]{Fig14.png}
\caption{Arms of an Hahnian crinoid; articulated and branched layout.}
\end{figure}
\clearpage
\begin{figure}[b]
\centering
\includegraphics[keepaspectratio]{Fig15.png}
\caption{Table 2 --- Columnar and parallel disposition of crystals eroded and marked by encrusting opaque material.}
\end{figure}
\clearpage
\rhead{Explanation of the Figures}
\begin{figure}[b]
\includegraphics[width=\textwidth,height=\textheight,keepaspectratio]{Fig16.png}
\caption{Thin section of enstatite produced by Mr. Daubrée by melting lherzolite with iron. Radiant fibers in fields circumscribed by crystalline rods. Magnification 50.}
\centering
\end{figure}
\clearpage
\begin{figure}[b]
\centering
\includegraphics[keepaspectratio]{Fig17.png}
\caption{Figures 17 and 18 --- Two of these rods. Magnification 500. On one of these rods one sees figures resembling pore protuberances or lamina scars; on the other, pieces resembling crampons.}
\end{figure}
\begin{figure}[b]
\centering
\includegraphics[keepaspectratio]{Fig18.png}
\caption{Figures 17 and 18 --- Two of these rods. Magnification 500. On one of these rods one sees figures resembling pore protuberances or lamina scars; on the other, pieces resembling crampons.}
\end{figure}
\clearpage
\begin{figure}[b]
\centering
\includegraphics[keepaspectratio]{Fig19.png}
\caption{Figures 19, 20, and 21, Table 3 --- Groups of artificial enstatite as glaze, produced by Mr. [Stanislas-Étienne] Meunier. Magnification 500. Figure 19, Lateral articulation of the columns. Hahnian crinoid arm. Figure 20, Hahnian coral; scar of a budding channel. Figure 21, Stellar grouping.}
\end{figure}
\clearpage
\begin{figure}[b]
\centering
\includegraphics[keepaspectratio]{Fig20.png}
\caption{Figures 19, 20, and 21, Table 3 --- Groups of artificial enstatite as glaze, produced by Mr. [Stanislas-Étienne] Meunier. Magnification 500. Figure 19, Lateral articulation of the columns. Hahnian crinoid arm. Figure 20, Hahnian coral; scar of a budding channel. Figure 21, Stellar grouping.}
\end{figure}
\clearpage
\begin{figure}[b]
\centering
\includegraphics[keepaspectratio]{Fig21.png}
\caption{Figures 19, 20, and 21, Table 3 --- Groups of artificial enstatite as glaze, produced by Mr. [Stanislas-Étienne] Meunier. Magnification 500. Figure 19, Lateral articulation of the columns. Hahnian crinoid arm. Figure 20, Hahnian coral; scar of a budding channel. Figure 21, Stellar grouping.}
\end{figure}
\clearpage
\begin{figure}[b]
\centering
\includegraphics[keepaspectratio]{Fig22.png}
\caption{A fragment of enstatite drawn from a thin section of the ``Schillerfels'' of Baste in the Harz. Magnification 300. Columnar and articulated disposition rendered visible by the of shock of polishing, as in the fragment of the transparent chondrule from Vouillé, Figure 11.}
\end{figure}
\clearpage
\begin{figure}[b]
\includegraphics[width=\textwidth,height=\textheight,keepaspectratio]{Fig23.png}
\caption{Group of crystals in a section of the Knyahinya meteorite resembling an artificial product from the melting of lherzolite with soft iron. Magnification 50.}
\centering
\end{figure}
\clearpage
\end{document}
